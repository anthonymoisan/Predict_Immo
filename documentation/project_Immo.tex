\documentclass[11pt]{article}

    \usepackage[breakable]{tcolorbox}
    \usepackage{parskip} % Stop auto-indenting (to mimic markdown behaviour)
    

    % Basic figure setup, for now with no caption control since it's done
    % automatically by Pandoc (which extracts ![](path) syntax from Markdown).
    \usepackage{graphicx}
    % Maintain compatibility with old templates. Remove in nbconvert 6.0
    \let\Oldincludegraphics\includegraphics
    % Ensure that by default, figures have no caption (until we provide a
    % proper Figure object with a Caption API and a way to capture that
    % in the conversion process - todo).
    \usepackage{caption}
    \DeclareCaptionFormat{nocaption}{}
    \captionsetup{format=nocaption,aboveskip=0pt,belowskip=0pt}

    \usepackage{float}
    \floatplacement{figure}{H} % forces figures to be placed at the correct location
    \usepackage{xcolor} % Allow colors to be defined
    \usepackage{enumerate} % Needed for markdown enumerations to work
    \usepackage{geometry} % Used to adjust the document margins
    \usepackage{amsmath} % Equations
    \usepackage{amssymb} % Equations
    \usepackage{textcomp} % defines textquotesingle
    % Hack from http://tex.stackexchange.com/a/47451/13684:
    \AtBeginDocument{%
        \def\PYZsq{\textquotesingle}% Upright quotes in Pygmentized code
    }
    \usepackage{upquote} % Upright quotes for verbatim code
    \usepackage{eurosym} % defines \euro

    \usepackage{iftex}
    \ifPDFTeX
        \usepackage[T1]{fontenc}
        \IfFileExists{alphabeta.sty}{
              \usepackage{alphabeta}
          }{
              \usepackage[mathletters]{ucs}
              \usepackage[utf8x]{inputenc}
          }
    \else
        \usepackage{fontspec}
        \usepackage{unicode-math}
    \fi

    \usepackage{fancyvrb} % verbatim replacement that allows latex
    \usepackage{grffile} % extends the file name processing of package graphics 
                         % to support a larger range
    \makeatletter % fix for old versions of grffile with XeLaTeX
    \@ifpackagelater{grffile}{2019/11/01}
    {
      % Do nothing on new versions
    }
    {
      \def\Gread@@xetex#1{%
        \IfFileExists{"\Gin@base".bb}%
        {\Gread@eps{\Gin@base.bb}}%
        {\Gread@@xetex@aux#1}%
      }
    }
    \makeatother
    \usepackage[Export]{adjustbox} % Used to constrain images to a maximum size
    \adjustboxset{max size={0.9\linewidth}{0.9\paperheight}}

    % The hyperref package gives us a pdf with properly built
    % internal navigation ('pdf bookmarks' for the table of contents,
    % internal cross-reference links, web links for URLs, etc.)
    \usepackage{hyperref}
    % The default LaTeX title has an obnoxious amount of whitespace. By default,
    % titling removes some of it. It also provides customization options.
    \usepackage{titling}
    \usepackage{longtable} % longtable support required by pandoc >1.10
    \usepackage{booktabs}  % table support for pandoc > 1.12.2
    \usepackage{array}     % table support for pandoc >= 2.11.3
    \usepackage{calc}      % table minipage width calculation for pandoc >= 2.11.1
    \usepackage[inline]{enumitem} % IRkernel/repr support (it uses the enumerate* environment)
    \usepackage[normalem]{ulem} % ulem is needed to support strikethroughs (\sout)
                                % normalem makes italics be italics, not underlines
    \usepackage{mathrsfs}
    

    
    % Colors for the hyperref package
    \definecolor{urlcolor}{rgb}{0,.145,.698}
    \definecolor{linkcolor}{rgb}{.71,0.21,0.01}
    \definecolor{citecolor}{rgb}{.12,.54,.11}

    % ANSI colors
    \definecolor{ansi-black}{HTML}{3E424D}
    \definecolor{ansi-black-intense}{HTML}{282C36}
    \definecolor{ansi-red}{HTML}{E75C58}
    \definecolor{ansi-red-intense}{HTML}{B22B31}
    \definecolor{ansi-green}{HTML}{00A250}
    \definecolor{ansi-green-intense}{HTML}{007427}
    \definecolor{ansi-yellow}{HTML}{DDB62B}
    \definecolor{ansi-yellow-intense}{HTML}{B27D12}
    \definecolor{ansi-blue}{HTML}{208FFB}
    \definecolor{ansi-blue-intense}{HTML}{0065CA}
    \definecolor{ansi-magenta}{HTML}{D160C4}
    \definecolor{ansi-magenta-intense}{HTML}{A03196}
    \definecolor{ansi-cyan}{HTML}{60C6C8}
    \definecolor{ansi-cyan-intense}{HTML}{258F8F}
    \definecolor{ansi-white}{HTML}{C5C1B4}
    \definecolor{ansi-white-intense}{HTML}{A1A6B2}
    \definecolor{ansi-default-inverse-fg}{HTML}{FFFFFF}
    \definecolor{ansi-default-inverse-bg}{HTML}{000000}

    % common color for the border for error outputs.
    \definecolor{outerrorbackground}{HTML}{FFDFDF}

    % commands and environments needed by pandoc snippets
    % extracted from the output of `pandoc -s`
    \providecommand{\tightlist}{%
      \setlength{\itemsep}{0pt}\setlength{\parskip}{0pt}}
    \DefineVerbatimEnvironment{Highlighting}{Verbatim}{commandchars=\\\{\}}
    % Add ',fontsize=\small' for more characters per line
    \newenvironment{Shaded}{}{}
    \newcommand{\KeywordTok}[1]{\textcolor[rgb]{0.00,0.44,0.13}{\textbf{{#1}}}}
    \newcommand{\DataTypeTok}[1]{\textcolor[rgb]{0.56,0.13,0.00}{{#1}}}
    \newcommand{\DecValTok}[1]{\textcolor[rgb]{0.25,0.63,0.44}{{#1}}}
    \newcommand{\BaseNTok}[1]{\textcolor[rgb]{0.25,0.63,0.44}{{#1}}}
    \newcommand{\FloatTok}[1]{\textcolor[rgb]{0.25,0.63,0.44}{{#1}}}
    \newcommand{\CharTok}[1]{\textcolor[rgb]{0.25,0.44,0.63}{{#1}}}
    \newcommand{\StringTok}[1]{\textcolor[rgb]{0.25,0.44,0.63}{{#1}}}
    \newcommand{\CommentTok}[1]{\textcolor[rgb]{0.38,0.63,0.69}{\textit{{#1}}}}
    \newcommand{\OtherTok}[1]{\textcolor[rgb]{0.00,0.44,0.13}{{#1}}}
    \newcommand{\AlertTok}[1]{\textcolor[rgb]{1.00,0.00,0.00}{\textbf{{#1}}}}
    \newcommand{\FunctionTok}[1]{\textcolor[rgb]{0.02,0.16,0.49}{{#1}}}
    \newcommand{\RegionMarkerTok}[1]{{#1}}
    \newcommand{\ErrorTok}[1]{\textcolor[rgb]{1.00,0.00,0.00}{\textbf{{#1}}}}
    \newcommand{\NormalTok}[1]{{#1}}
    
    % Additional commands for more recent versions of Pandoc
    \newcommand{\ConstantTok}[1]{\textcolor[rgb]{0.53,0.00,0.00}{{#1}}}
    \newcommand{\SpecialCharTok}[1]{\textcolor[rgb]{0.25,0.44,0.63}{{#1}}}
    \newcommand{\VerbatimStringTok}[1]{\textcolor[rgb]{0.25,0.44,0.63}{{#1}}}
    \newcommand{\SpecialStringTok}[1]{\textcolor[rgb]{0.73,0.40,0.53}{{#1}}}
    \newcommand{\ImportTok}[1]{{#1}}
    \newcommand{\DocumentationTok}[1]{\textcolor[rgb]{0.73,0.13,0.13}{\textit{{#1}}}}
    \newcommand{\AnnotationTok}[1]{\textcolor[rgb]{0.38,0.63,0.69}{\textbf{\textit{{#1}}}}}
    \newcommand{\CommentVarTok}[1]{\textcolor[rgb]{0.38,0.63,0.69}{\textbf{\textit{{#1}}}}}
    \newcommand{\VariableTok}[1]{\textcolor[rgb]{0.10,0.09,0.49}{{#1}}}
    \newcommand{\ControlFlowTok}[1]{\textcolor[rgb]{0.00,0.44,0.13}{\textbf{{#1}}}}
    \newcommand{\OperatorTok}[1]{\textcolor[rgb]{0.40,0.40,0.40}{{#1}}}
    \newcommand{\BuiltInTok}[1]{{#1}}
    \newcommand{\ExtensionTok}[1]{{#1}}
    \newcommand{\PreprocessorTok}[1]{\textcolor[rgb]{0.74,0.48,0.00}{{#1}}}
    \newcommand{\AttributeTok}[1]{\textcolor[rgb]{0.49,0.56,0.16}{{#1}}}
    \newcommand{\InformationTok}[1]{\textcolor[rgb]{0.38,0.63,0.69}{\textbf{\textit{{#1}}}}}
    \newcommand{\WarningTok}[1]{\textcolor[rgb]{0.38,0.63,0.69}{\textbf{\textit{{#1}}}}}
    
    
    % Define a nice break command that doesn't care if a line doesn't already
    % exist.
    \def\br{\hspace*{\fill} \\* }
    % Math Jax compatibility definitions
    \def\gt{>}
    \def\lt{<}
    \let\Oldtex\TeX
    \let\Oldlatex\LaTeX
    \renewcommand{\TeX}{\textrm{\Oldtex}}
    \renewcommand{\LaTeX}{\textrm{\Oldlatex}}
    % Document parameters
    % Document title
    \title{project\_Immo}
    
    
    
    
    
% Pygments definitions
\makeatletter
\def\PY@reset{\let\PY@it=\relax \let\PY@bf=\relax%
    \let\PY@ul=\relax \let\PY@tc=\relax%
    \let\PY@bc=\relax \let\PY@ff=\relax}
\def\PY@tok#1{\csname PY@tok@#1\endcsname}
\def\PY@toks#1+{\ifx\relax#1\empty\else%
    \PY@tok{#1}\expandafter\PY@toks\fi}
\def\PY@do#1{\PY@bc{\PY@tc{\PY@ul{%
    \PY@it{\PY@bf{\PY@ff{#1}}}}}}}
\def\PY#1#2{\PY@reset\PY@toks#1+\relax+\PY@do{#2}}

\@namedef{PY@tok@w}{\def\PY@tc##1{\textcolor[rgb]{0.73,0.73,0.73}{##1}}}
\@namedef{PY@tok@c}{\let\PY@it=\textit\def\PY@tc##1{\textcolor[rgb]{0.24,0.48,0.48}{##1}}}
\@namedef{PY@tok@cp}{\def\PY@tc##1{\textcolor[rgb]{0.61,0.40,0.00}{##1}}}
\@namedef{PY@tok@k}{\let\PY@bf=\textbf\def\PY@tc##1{\textcolor[rgb]{0.00,0.50,0.00}{##1}}}
\@namedef{PY@tok@kp}{\def\PY@tc##1{\textcolor[rgb]{0.00,0.50,0.00}{##1}}}
\@namedef{PY@tok@kt}{\def\PY@tc##1{\textcolor[rgb]{0.69,0.00,0.25}{##1}}}
\@namedef{PY@tok@o}{\def\PY@tc##1{\textcolor[rgb]{0.40,0.40,0.40}{##1}}}
\@namedef{PY@tok@ow}{\let\PY@bf=\textbf\def\PY@tc##1{\textcolor[rgb]{0.67,0.13,1.00}{##1}}}
\@namedef{PY@tok@nb}{\def\PY@tc##1{\textcolor[rgb]{0.00,0.50,0.00}{##1}}}
\@namedef{PY@tok@nf}{\def\PY@tc##1{\textcolor[rgb]{0.00,0.00,1.00}{##1}}}
\@namedef{PY@tok@nc}{\let\PY@bf=\textbf\def\PY@tc##1{\textcolor[rgb]{0.00,0.00,1.00}{##1}}}
\@namedef{PY@tok@nn}{\let\PY@bf=\textbf\def\PY@tc##1{\textcolor[rgb]{0.00,0.00,1.00}{##1}}}
\@namedef{PY@tok@ne}{\let\PY@bf=\textbf\def\PY@tc##1{\textcolor[rgb]{0.80,0.25,0.22}{##1}}}
\@namedef{PY@tok@nv}{\def\PY@tc##1{\textcolor[rgb]{0.10,0.09,0.49}{##1}}}
\@namedef{PY@tok@no}{\def\PY@tc##1{\textcolor[rgb]{0.53,0.00,0.00}{##1}}}
\@namedef{PY@tok@nl}{\def\PY@tc##1{\textcolor[rgb]{0.46,0.46,0.00}{##1}}}
\@namedef{PY@tok@ni}{\let\PY@bf=\textbf\def\PY@tc##1{\textcolor[rgb]{0.44,0.44,0.44}{##1}}}
\@namedef{PY@tok@na}{\def\PY@tc##1{\textcolor[rgb]{0.41,0.47,0.13}{##1}}}
\@namedef{PY@tok@nt}{\let\PY@bf=\textbf\def\PY@tc##1{\textcolor[rgb]{0.00,0.50,0.00}{##1}}}
\@namedef{PY@tok@nd}{\def\PY@tc##1{\textcolor[rgb]{0.67,0.13,1.00}{##1}}}
\@namedef{PY@tok@s}{\def\PY@tc##1{\textcolor[rgb]{0.73,0.13,0.13}{##1}}}
\@namedef{PY@tok@sd}{\let\PY@it=\textit\def\PY@tc##1{\textcolor[rgb]{0.73,0.13,0.13}{##1}}}
\@namedef{PY@tok@si}{\let\PY@bf=\textbf\def\PY@tc##1{\textcolor[rgb]{0.64,0.35,0.47}{##1}}}
\@namedef{PY@tok@se}{\let\PY@bf=\textbf\def\PY@tc##1{\textcolor[rgb]{0.67,0.36,0.12}{##1}}}
\@namedef{PY@tok@sr}{\def\PY@tc##1{\textcolor[rgb]{0.64,0.35,0.47}{##1}}}
\@namedef{PY@tok@ss}{\def\PY@tc##1{\textcolor[rgb]{0.10,0.09,0.49}{##1}}}
\@namedef{PY@tok@sx}{\def\PY@tc##1{\textcolor[rgb]{0.00,0.50,0.00}{##1}}}
\@namedef{PY@tok@m}{\def\PY@tc##1{\textcolor[rgb]{0.40,0.40,0.40}{##1}}}
\@namedef{PY@tok@gh}{\let\PY@bf=\textbf\def\PY@tc##1{\textcolor[rgb]{0.00,0.00,0.50}{##1}}}
\@namedef{PY@tok@gu}{\let\PY@bf=\textbf\def\PY@tc##1{\textcolor[rgb]{0.50,0.00,0.50}{##1}}}
\@namedef{PY@tok@gd}{\def\PY@tc##1{\textcolor[rgb]{0.63,0.00,0.00}{##1}}}
\@namedef{PY@tok@gi}{\def\PY@tc##1{\textcolor[rgb]{0.00,0.52,0.00}{##1}}}
\@namedef{PY@tok@gr}{\def\PY@tc##1{\textcolor[rgb]{0.89,0.00,0.00}{##1}}}
\@namedef{PY@tok@ge}{\let\PY@it=\textit}
\@namedef{PY@tok@gs}{\let\PY@bf=\textbf}
\@namedef{PY@tok@gp}{\let\PY@bf=\textbf\def\PY@tc##1{\textcolor[rgb]{0.00,0.00,0.50}{##1}}}
\@namedef{PY@tok@go}{\def\PY@tc##1{\textcolor[rgb]{0.44,0.44,0.44}{##1}}}
\@namedef{PY@tok@gt}{\def\PY@tc##1{\textcolor[rgb]{0.00,0.27,0.87}{##1}}}
\@namedef{PY@tok@err}{\def\PY@bc##1{{\setlength{\fboxsep}{\string -\fboxrule}\fcolorbox[rgb]{1.00,0.00,0.00}{1,1,1}{\strut ##1}}}}
\@namedef{PY@tok@kc}{\let\PY@bf=\textbf\def\PY@tc##1{\textcolor[rgb]{0.00,0.50,0.00}{##1}}}
\@namedef{PY@tok@kd}{\let\PY@bf=\textbf\def\PY@tc##1{\textcolor[rgb]{0.00,0.50,0.00}{##1}}}
\@namedef{PY@tok@kn}{\let\PY@bf=\textbf\def\PY@tc##1{\textcolor[rgb]{0.00,0.50,0.00}{##1}}}
\@namedef{PY@tok@kr}{\let\PY@bf=\textbf\def\PY@tc##1{\textcolor[rgb]{0.00,0.50,0.00}{##1}}}
\@namedef{PY@tok@bp}{\def\PY@tc##1{\textcolor[rgb]{0.00,0.50,0.00}{##1}}}
\@namedef{PY@tok@fm}{\def\PY@tc##1{\textcolor[rgb]{0.00,0.00,1.00}{##1}}}
\@namedef{PY@tok@vc}{\def\PY@tc##1{\textcolor[rgb]{0.10,0.09,0.49}{##1}}}
\@namedef{PY@tok@vg}{\def\PY@tc##1{\textcolor[rgb]{0.10,0.09,0.49}{##1}}}
\@namedef{PY@tok@vi}{\def\PY@tc##1{\textcolor[rgb]{0.10,0.09,0.49}{##1}}}
\@namedef{PY@tok@vm}{\def\PY@tc##1{\textcolor[rgb]{0.10,0.09,0.49}{##1}}}
\@namedef{PY@tok@sa}{\def\PY@tc##1{\textcolor[rgb]{0.73,0.13,0.13}{##1}}}
\@namedef{PY@tok@sb}{\def\PY@tc##1{\textcolor[rgb]{0.73,0.13,0.13}{##1}}}
\@namedef{PY@tok@sc}{\def\PY@tc##1{\textcolor[rgb]{0.73,0.13,0.13}{##1}}}
\@namedef{PY@tok@dl}{\def\PY@tc##1{\textcolor[rgb]{0.73,0.13,0.13}{##1}}}
\@namedef{PY@tok@s2}{\def\PY@tc##1{\textcolor[rgb]{0.73,0.13,0.13}{##1}}}
\@namedef{PY@tok@sh}{\def\PY@tc##1{\textcolor[rgb]{0.73,0.13,0.13}{##1}}}
\@namedef{PY@tok@s1}{\def\PY@tc##1{\textcolor[rgb]{0.73,0.13,0.13}{##1}}}
\@namedef{PY@tok@mb}{\def\PY@tc##1{\textcolor[rgb]{0.40,0.40,0.40}{##1}}}
\@namedef{PY@tok@mf}{\def\PY@tc##1{\textcolor[rgb]{0.40,0.40,0.40}{##1}}}
\@namedef{PY@tok@mh}{\def\PY@tc##1{\textcolor[rgb]{0.40,0.40,0.40}{##1}}}
\@namedef{PY@tok@mi}{\def\PY@tc##1{\textcolor[rgb]{0.40,0.40,0.40}{##1}}}
\@namedef{PY@tok@il}{\def\PY@tc##1{\textcolor[rgb]{0.40,0.40,0.40}{##1}}}
\@namedef{PY@tok@mo}{\def\PY@tc##1{\textcolor[rgb]{0.40,0.40,0.40}{##1}}}
\@namedef{PY@tok@ch}{\let\PY@it=\textit\def\PY@tc##1{\textcolor[rgb]{0.24,0.48,0.48}{##1}}}
\@namedef{PY@tok@cm}{\let\PY@it=\textit\def\PY@tc##1{\textcolor[rgb]{0.24,0.48,0.48}{##1}}}
\@namedef{PY@tok@cpf}{\let\PY@it=\textit\def\PY@tc##1{\textcolor[rgb]{0.24,0.48,0.48}{##1}}}
\@namedef{PY@tok@c1}{\let\PY@it=\textit\def\PY@tc##1{\textcolor[rgb]{0.24,0.48,0.48}{##1}}}
\@namedef{PY@tok@cs}{\let\PY@it=\textit\def\PY@tc##1{\textcolor[rgb]{0.24,0.48,0.48}{##1}}}

\def\PYZbs{\char`\\}
\def\PYZus{\char`\_}
\def\PYZob{\char`\{}
\def\PYZcb{\char`\}}
\def\PYZca{\char`\^}
\def\PYZam{\char`\&}
\def\PYZlt{\char`\<}
\def\PYZgt{\char`\>}
\def\PYZsh{\char`\#}
\def\PYZpc{\char`\%}
\def\PYZdl{\char`\$}
\def\PYZhy{\char`\-}
\def\PYZsq{\char`\'}
\def\PYZdq{\char`\"}
\def\PYZti{\char`\~}
% for compatibility with earlier versions
\def\PYZat{@}
\def\PYZlb{[}
\def\PYZrb{]}
\makeatother


    % For linebreaks inside Verbatim environment from package fancyvrb. 
    \makeatletter
        \newbox\Wrappedcontinuationbox 
        \newbox\Wrappedvisiblespacebox 
        \newcommand*\Wrappedvisiblespace {\textcolor{red}{\textvisiblespace}} 
        \newcommand*\Wrappedcontinuationsymbol {\textcolor{red}{\llap{\tiny$\m@th\hookrightarrow$}}} 
        \newcommand*\Wrappedcontinuationindent {3ex } 
        \newcommand*\Wrappedafterbreak {\kern\Wrappedcontinuationindent\copy\Wrappedcontinuationbox} 
        % Take advantage of the already applied Pygments mark-up to insert 
        % potential linebreaks for TeX processing. 
        %        {, <, #, %, $, ' and ": go to next line. 
        %        _, }, ^, &, >, - and ~: stay at end of broken line. 
        % Use of \textquotesingle for straight quote. 
        \newcommand*\Wrappedbreaksatspecials {% 
            \def\PYGZus{\discretionary{\char`\_}{\Wrappedafterbreak}{\char`\_}}% 
            \def\PYGZob{\discretionary{}{\Wrappedafterbreak\char`\{}{\char`\{}}% 
            \def\PYGZcb{\discretionary{\char`\}}{\Wrappedafterbreak}{\char`\}}}% 
            \def\PYGZca{\discretionary{\char`\^}{\Wrappedafterbreak}{\char`\^}}% 
            \def\PYGZam{\discretionary{\char`\&}{\Wrappedafterbreak}{\char`\&}}% 
            \def\PYGZlt{\discretionary{}{\Wrappedafterbreak\char`\<}{\char`\<}}% 
            \def\PYGZgt{\discretionary{\char`\>}{\Wrappedafterbreak}{\char`\>}}% 
            \def\PYGZsh{\discretionary{}{\Wrappedafterbreak\char`\#}{\char`\#}}% 
            \def\PYGZpc{\discretionary{}{\Wrappedafterbreak\char`\%}{\char`\%}}% 
            \def\PYGZdl{\discretionary{}{\Wrappedafterbreak\char`\$}{\char`\$}}% 
            \def\PYGZhy{\discretionary{\char`\-}{\Wrappedafterbreak}{\char`\-}}% 
            \def\PYGZsq{\discretionary{}{\Wrappedafterbreak\textquotesingle}{\textquotesingle}}% 
            \def\PYGZdq{\discretionary{}{\Wrappedafterbreak\char`\"}{\char`\"}}% 
            \def\PYGZti{\discretionary{\char`\~}{\Wrappedafterbreak}{\char`\~}}% 
        } 
        % Some characters . , ; ? ! / are not pygmentized. 
        % This macro makes them "active" and they will insert potential linebreaks 
        \newcommand*\Wrappedbreaksatpunct {% 
            \lccode`\~`\.\lowercase{\def~}{\discretionary{\hbox{\char`\.}}{\Wrappedafterbreak}{\hbox{\char`\.}}}% 
            \lccode`\~`\,\lowercase{\def~}{\discretionary{\hbox{\char`\,}}{\Wrappedafterbreak}{\hbox{\char`\,}}}% 
            \lccode`\~`\;\lowercase{\def~}{\discretionary{\hbox{\char`\;}}{\Wrappedafterbreak}{\hbox{\char`\;}}}% 
            \lccode`\~`\:\lowercase{\def~}{\discretionary{\hbox{\char`\:}}{\Wrappedafterbreak}{\hbox{\char`\:}}}% 
            \lccode`\~`\?\lowercase{\def~}{\discretionary{\hbox{\char`\?}}{\Wrappedafterbreak}{\hbox{\char`\?}}}% 
            \lccode`\~`\!\lowercase{\def~}{\discretionary{\hbox{\char`\!}}{\Wrappedafterbreak}{\hbox{\char`\!}}}% 
            \lccode`\~`\/\lowercase{\def~}{\discretionary{\hbox{\char`\/}}{\Wrappedafterbreak}{\hbox{\char`\/}}}% 
            \catcode`\.\active
            \catcode`\,\active 
            \catcode`\;\active
            \catcode`\:\active
            \catcode`\?\active
            \catcode`\!\active
            \catcode`\/\active 
            \lccode`\~`\~ 	
        }
    \makeatother

    \let\OriginalVerbatim=\Verbatim
    \makeatletter
    \renewcommand{\Verbatim}[1][1]{%
        %\parskip\z@skip
        \sbox\Wrappedcontinuationbox {\Wrappedcontinuationsymbol}%
        \sbox\Wrappedvisiblespacebox {\FV@SetupFont\Wrappedvisiblespace}%
        \def\FancyVerbFormatLine ##1{\hsize\linewidth
            \vtop{\raggedright\hyphenpenalty\z@\exhyphenpenalty\z@
                \doublehyphendemerits\z@\finalhyphendemerits\z@
                \strut ##1\strut}%
        }%
        % If the linebreak is at a space, the latter will be displayed as visible
        % space at end of first line, and a continuation symbol starts next line.
        % Stretch/shrink are however usually zero for typewriter font.
        \def\FV@Space {%
            \nobreak\hskip\z@ plus\fontdimen3\font minus\fontdimen4\font
            \discretionary{\copy\Wrappedvisiblespacebox}{\Wrappedafterbreak}
            {\kern\fontdimen2\font}%
        }%
        
        % Allow breaks at special characters using \PYG... macros.
        \Wrappedbreaksatspecials
        % Breaks at punctuation characters . , ; ? ! and / need catcode=\active 	
        \OriginalVerbatim[#1,codes*=\Wrappedbreaksatpunct]%
    }
    \makeatother

    % Exact colors from NB
    \definecolor{incolor}{HTML}{303F9F}
    \definecolor{outcolor}{HTML}{D84315}
    \definecolor{cellborder}{HTML}{CFCFCF}
    \definecolor{cellbackground}{HTML}{F7F7F7}
    
    % prompt
    \makeatletter
    \newcommand{\boxspacing}{\kern\kvtcb@left@rule\kern\kvtcb@boxsep}
    \makeatother
    \newcommand{\prompt}[4]{
        {\ttfamily\llap{{\color{#2}[#3]:\hspace{3pt}#4}}\vspace{-\baselineskip}}
    }
    

    
    % Prevent overflowing lines due to hard-to-break entities
    \sloppy 
    % Setup hyperref package
    \hypersetup{
      breaklinks=true,  % so long urls are correctly broken across lines
      colorlinks=true,
      urlcolor=urlcolor,
      linkcolor=linkcolor,
      citecolor=citecolor,
      }
    % Slightly bigger margins than the latex defaults
    
    \geometry{verbose,tmargin=1in,bmargin=1in,lmargin=1in,rmargin=1in}
    
    

\begin{document}
    
    \maketitle
    
    

    
    \hypertarget{description-du-projet}{%
\section{Description du projet}\label{description-du-projet}}

Le projet s'intéresse au prix de l'immobilier sur Paris. Est-on en
mesure d'avoir une bonne prédiction sur la valeur mobilière d'un bien

    \hypertarget{lecture-des-jeux-de-donnuxe9es}{%
\section{Lecture des jeux de
données}\label{lecture-des-jeux-de-donnuxe9es}}

Les jeux de données sont disponibles sur
https://www.data.gouv.fr/fr/datasets/demandes-de-valeurs-foncieres-geolocalisees/

    \begin{tcolorbox}[breakable, size=fbox, boxrule=1pt, pad at break*=1mm,colback=cellbackground, colframe=cellborder]
\prompt{In}{incolor}{1}{\boxspacing}
\begin{Verbatim}[commandchars=\\\{\}]
\PY{k+kn}{import} \PY{n+nn}{pandas} \PY{k}{as} \PY{n+nn}{pd}
\PY{k+kn}{import} \PY{n+nn}{numpy} \PY{k}{as} \PY{n+nn}{np}

\PY{k+kn}{import} \PY{n+nn}{matplotlib}\PY{n+nn}{.}\PY{n+nn}{pyplot} \PY{k}{as} \PY{n+nn}{plt}
\PY{k+kn}{import} \PY{n+nn}{seaborn} \PY{k}{as} \PY{n+nn}{sns}
\PY{o}{\PYZpc{}}\PY{k}{matplotlib} inline
\end{Verbatim}
\end{tcolorbox}

    \begin{tcolorbox}[breakable, size=fbox, boxrule=1pt, pad at break*=1mm,colback=cellbackground, colframe=cellborder]
\prompt{In}{incolor}{2}{\boxspacing}
\begin{Verbatim}[commandchars=\\\{\}]
\PY{c+c1}{\PYZsh{} Fonction pour lire les donnnées en fonction du fichier}
\PY{k}{def} \PY{n+nf}{ReadFile}\PY{p}{(}\PY{n}{nomFile}\PY{p}{,} \PY{n}{delimiter} \PY{o}{=} \PY{l+s+s1}{\PYZsq{}}\PY{l+s+s1}{|}\PY{l+s+s1}{\PYZsq{}}\PY{p}{)}\PY{p}{:}
    \PY{c+c1}{\PYZsh{} lecture du fichier excel}
    \PY{n}{df} \PY{o}{=} \PY{n}{pd}\PY{o}{.}\PY{n}{read\PYZus{}csv}\PY{p}{(}\PY{n}{nomFile}\PY{p}{,} \PY{n}{delimiter} \PY{o}{=} \PY{n}{delimiter}\PY{p}{,} \PY{n}{low\PYZus{}memory} \PY{o}{=} \PY{k+kc}{False}\PY{p}{)}
    \PY{n+nb}{print}\PY{p}{(}\PY{l+s+s2}{\PYZdq{}}\PY{l+s+s2}{taille du jeu de donnees :}\PY{l+s+s2}{\PYZdq{}}\PY{p}{,} \PY{n}{df}\PY{o}{.}\PY{n}{shape}\PY{p}{)}
    \PY{k}{return} \PY{n}{df}
    
    
\end{Verbatim}
\end{tcolorbox}

    \begin{tcolorbox}[breakable, size=fbox, boxrule=1pt, pad at break*=1mm,colback=cellbackground, colframe=cellborder]
\prompt{In}{incolor}{3}{\boxspacing}
\begin{Verbatim}[commandchars=\\\{\}]
\PY{c+c1}{\PYZsh{} Fonction pour extraire les données à partir d\PYZsq{}un numéro de département}
\PY{k}{def} \PY{n+nf}{ExtractDepartement}\PY{p}{(}\PY{n}{df}\PY{p}{,} \PY{n}{numDep}\PY{p}{)}\PY{p}{:}
    \PY{n}{df}\PY{p}{[}\PY{l+s+s1}{\PYZsq{}}\PY{l+s+s1}{code\PYZus{}departement}\PY{l+s+s1}{\PYZsq{}}\PY{p}{]}\PY{o}{.}\PY{n}{astype}\PY{p}{(}\PY{n+nb}{str}\PY{p}{)}
    \PY{n}{df}\PY{p}{[}\PY{l+s+s1}{\PYZsq{}}\PY{l+s+s1}{Validation}\PY{l+s+s1}{\PYZsq{}}\PY{p}{]} \PY{o}{=} \PY{p}{(}\PY{n}{df}\PY{p}{[}\PY{l+s+s1}{\PYZsq{}}\PY{l+s+s1}{code\PYZus{}departement}\PY{l+s+s1}{\PYZsq{}}\PY{p}{]} \PY{o}{==} \PY{n}{numDep} \PY{p}{)}
    \PY{n}{dfDep} \PY{o}{=} \PY{n}{df}\PY{p}{[}\PY{n}{df}\PY{p}{[}\PY{l+s+s1}{\PYZsq{}}\PY{l+s+s1}{Validation}\PY{l+s+s1}{\PYZsq{}}\PY{p}{]}\PY{o}{==}\PY{k+kc}{True}\PY{p}{]}
    \PY{n}{dfDep} \PY{o}{=} \PY{n}{dfDep}\PY{o}{.}\PY{n}{drop}\PY{p}{(}\PY{l+s+s1}{\PYZsq{}}\PY{l+s+s1}{Validation}\PY{l+s+s1}{\PYZsq{}}\PY{p}{,} \PY{n}{axis}\PY{o}{=}\PY{l+m+mi}{1}\PY{p}{)}
    \PY{n+nb}{print}\PY{p}{(}\PY{l+s+s2}{\PYZdq{}}\PY{l+s+s2}{Departement : }\PY{l+s+si}{\PYZob{}0\PYZcb{}}\PY{l+s+s2}{\PYZdq{}}\PY{o}{.}\PY{n}{format}\PY{p}{(}\PY{n}{numDep}\PY{p}{)}\PY{p}{)}
    \PY{n+nb}{print}\PY{p}{(}\PY{l+s+s2}{\PYZdq{}}\PY{l+s+s2}{Taille du jeu de donnees}\PY{l+s+s2}{\PYZdq{}}\PY{p}{,} \PY{n}{dfDep}\PY{o}{.}\PY{n}{shape}\PY{p}{)}
    \PY{k}{return} \PY{n}{dfDep}       
\end{Verbatim}
\end{tcolorbox}

    \begin{tcolorbox}[breakable, size=fbox, boxrule=1pt, pad at break*=1mm,colback=cellbackground, colframe=cellborder]
\prompt{In}{incolor}{4}{\boxspacing}
\begin{Verbatim}[commandchars=\\\{\}]
\PY{n}{df1} \PY{o}{=} \PY{n}{ReadFile}\PY{p}{(}\PY{l+s+s2}{\PYZdq{}}\PY{l+s+s2}{../input/AvecCoordonneesGeo/full.csv}\PY{l+s+s2}{\PYZdq{}}\PY{p}{,} \PY{l+s+s1}{\PYZsq{}}\PY{l+s+s1}{,}\PY{l+s+s1}{\PYZsq{}}\PY{p}{)}
\end{Verbatim}
\end{tcolorbox}

    \begin{Verbatim}[commandchars=\\\{\}]
taille du jeu de donnees : (1429093, 40)
    \end{Verbatim}

    \begin{tcolorbox}[breakable, size=fbox, boxrule=1pt, pad at break*=1mm,colback=cellbackground, colframe=cellborder]
\prompt{In}{incolor}{5}{\boxspacing}
\begin{Verbatim}[commandchars=\\\{\}]
\PY{n}{df2} \PY{o}{=} \PY{n}{ReadFile}\PY{p}{(}\PY{l+s+s2}{\PYZdq{}}\PY{l+s+s2}{../input/AvecCoordonneesGeo/full2021.csv}\PY{l+s+s2}{\PYZdq{}}\PY{p}{,} \PY{l+s+s1}{\PYZsq{}}\PY{l+s+s1}{,}\PY{l+s+s1}{\PYZsq{}}\PY{p}{)}
\end{Verbatim}
\end{tcolorbox}

    \begin{Verbatim}[commandchars=\\\{\}]
taille du jeu de donnees : (4375223, 40)
    \end{Verbatim}

    \hypertarget{concatuxe9nation-des-donnuxe9es}{%
\subsection{Concaténation des
données}\label{concatuxe9nation-des-donnuxe9es}}

    \begin{tcolorbox}[breakable, size=fbox, boxrule=1pt, pad at break*=1mm,colback=cellbackground, colframe=cellborder]
\prompt{In}{incolor}{6}{\boxspacing}
\begin{Verbatim}[commandchars=\\\{\}]
\PY{c+c1}{\PYZsh{}Concaténation des deux jeux de données}
\PY{n}{df} \PY{o}{=} \PY{n}{pd}\PY{o}{.}\PY{n}{concat}\PY{p}{(}\PY{p}{[}\PY{n}{df1}\PY{p}{,}\PY{n}{df2}\PY{p}{]}\PY{p}{)}
\PY{c+c1}{\PYZsh{},keys=[\PYZsq{}2022\PYZsq{},\PYZsq{}2021\PYZsq{}])}
\PY{c+c1}{\PYZsh{}, names = [\PYZsq{}FileInput\PYZsq{}, \PYZsq{}RowId\PYZsq{}])}
\PY{n+nb}{print}\PY{p}{(}\PY{l+s+s2}{\PYZdq{}}\PY{l+s+s2}{taille suite à union :}\PY{l+s+s2}{\PYZdq{}}\PY{p}{,} \PY{n}{df}\PY{o}{.}\PY{n}{shape}\PY{p}{)}
\end{Verbatim}
\end{tcolorbox}

    \begin{Verbatim}[commandchars=\\\{\}]
taille suite à union : (5804316, 40)
    \end{Verbatim}

    \hypertarget{ruxe9duction-uxe0-un-duxe9partement}{%
\subsection{Réduction à un
département}\label{ruxe9duction-uxe0-un-duxe9partement}}

    \begin{tcolorbox}[breakable, size=fbox, boxrule=1pt, pad at break*=1mm,colback=cellbackground, colframe=cellborder]
\prompt{In}{incolor}{7}{\boxspacing}
\begin{Verbatim}[commandchars=\\\{\}]
\PY{n}{dfDep} \PY{o}{=} \PY{n}{ExtractDepartement}\PY{p}{(}\PY{n}{df}\PY{p}{,}\PY{l+s+s1}{\PYZsq{}}\PY{l+s+s1}{75}\PY{l+s+s1}{\PYZsq{}}\PY{p}{)}
\PY{n}{dfDepIni} \PY{o}{=} \PY{n}{dfDep}
\end{Verbatim}
\end{tcolorbox}

    \begin{Verbatim}[commandchars=\\\{\}]
Departement : 75
Taille du jeu de donnees (130696, 40)
    \end{Verbatim}

    \hypertarget{nettoyage-du-jeu-de-donnuxe9es}{%
\section{Nettoyage du jeu de
données}\label{nettoyage-du-jeu-de-donnuxe9es}}

    \hypertarget{nettoyage}{%
\subsection{Nettoyage}\label{nettoyage}}

    \begin{tcolorbox}[breakable, size=fbox, boxrule=1pt, pad at break*=1mm,colback=cellbackground, colframe=cellborder]
\prompt{In}{incolor}{8}{\boxspacing}
\begin{Verbatim}[commandchars=\\\{\}]
\PY{c+c1}{\PYZsh{}Suppression des lignes en doublon}
\PY{n}{dfDep}\PY{o}{.}\PY{n}{drop\PYZus{}duplicates}\PY{p}{(}\PY{n}{inplace}\PY{o}{=}\PY{k+kc}{True}\PY{p}{)}
\PY{n}{dfDep}\PY{o}{.}\PY{n}{shape}
\end{Verbatim}
\end{tcolorbox}

            \begin{tcolorbox}[breakable, size=fbox, boxrule=.5pt, pad at break*=1mm, opacityfill=0]
\prompt{Out}{outcolor}{8}{\boxspacing}
\begin{Verbatim}[commandchars=\\\{\}]
(119224, 40)
\end{Verbatim}
\end{tcolorbox}
        
    \begin{tcolorbox}[breakable, size=fbox, boxrule=1pt, pad at break*=1mm,colback=cellbackground, colframe=cellborder]
\prompt{In}{incolor}{9}{\boxspacing}
\begin{Verbatim}[commandchars=\\\{\}]
\PY{c+c1}{\PYZsh{}Suppression des longitudes et latitudes null}
\PY{n}{dfDep}\PY{o}{.}\PY{n}{drop}\PY{p}{(}\PY{n}{dfDep}\PY{p}{[}\PY{p}{(}\PY{n}{dfDep}\PY{p}{[}\PY{l+s+s1}{\PYZsq{}}\PY{l+s+s1}{longitude}\PY{l+s+s1}{\PYZsq{}}\PY{p}{]}\PY{o}{.}\PY{n}{isnull}\PY{p}{(}\PY{p}{)}\PY{p}{)} \PY{o}{|} \PY{p}{(}\PY{n}{dfDep}\PY{p}{[}\PY{l+s+s1}{\PYZsq{}}\PY{l+s+s1}{latitude}\PY{l+s+s1}{\PYZsq{}}\PY{p}{]}\PY{o}{.}\PY{n}{isnull}\PY{p}{(}\PY{p}{)}\PY{p}{)}\PY{p}{]}\PY{o}{.}\PY{n}{index}\PY{p}{,} \PY{n}{inplace}\PY{o}{=}\PY{k+kc}{True}\PY{p}{)}
\PY{n}{dfDep}\PY{o}{.}\PY{n}{shape}
\end{Verbatim}
\end{tcolorbox}

            \begin{tcolorbox}[breakable, size=fbox, boxrule=.5pt, pad at break*=1mm, opacityfill=0]
\prompt{Out}{outcolor}{9}{\boxspacing}
\begin{Verbatim}[commandchars=\\\{\}]
(119182, 40)
\end{Verbatim}
\end{tcolorbox}
        
    \begin{tcolorbox}[breakable, size=fbox, boxrule=1pt, pad at break*=1mm,colback=cellbackground, colframe=cellborder]
\prompt{In}{incolor}{10}{\boxspacing}
\begin{Verbatim}[commandchars=\\\{\}]
\PY{c+c1}{\PYZsh{}Suppression des données où la valeur foncière est null}
\PY{n}{dfDep}\PY{o}{.}\PY{n}{drop}\PY{p}{(}\PY{n}{dfDep}\PY{p}{[}\PY{n}{dfDep}\PY{p}{[}\PY{l+s+s1}{\PYZsq{}}\PY{l+s+s1}{valeur\PYZus{}fonciere}\PY{l+s+s1}{\PYZsq{}}\PY{p}{]}\PY{o}{.}\PY{n}{isnull}\PY{p}{(}\PY{p}{)} \PY{p}{]}\PY{o}{.}\PY{n}{index}\PY{p}{,} \PY{n}{inplace}\PY{o}{=}\PY{k+kc}{True}\PY{p}{)}
\PY{n}{dfDep}\PY{o}{.}\PY{n}{shape}
\end{Verbatim}
\end{tcolorbox}

            \begin{tcolorbox}[breakable, size=fbox, boxrule=.5pt, pad at break*=1mm, opacityfill=0]
\prompt{Out}{outcolor}{10}{\boxspacing}
\begin{Verbatim}[commandchars=\\\{\}]
(118329, 40)
\end{Verbatim}
\end{tcolorbox}
        
    \begin{tcolorbox}[breakable, size=fbox, boxrule=1pt, pad at break*=1mm,colback=cellbackground, colframe=cellborder]
\prompt{In}{incolor}{11}{\boxspacing}
\begin{Verbatim}[commandchars=\\\{\}]
\PY{c+c1}{\PYZsh{}Suppression des valeurs foncières \PYZlt{} 50KE et \PYZgt{}3000KE}
\PY{n}{dfDep}\PY{o}{.}\PY{n}{drop}\PY{p}{(}\PY{n}{dfDep}\PY{p}{[}\PY{n}{dfDep}\PY{p}{[}\PY{l+s+s1}{\PYZsq{}}\PY{l+s+s1}{valeur\PYZus{}fonciere}\PY{l+s+s1}{\PYZsq{}}\PY{p}{]}\PY{o}{\PYZlt{}}\PY{l+m+mi}{50000} \PY{p}{]}\PY{o}{.}\PY{n}{index}\PY{p}{,} \PY{n}{inplace}\PY{o}{=}\PY{k+kc}{True}\PY{p}{)}
\PY{n}{dfDep}\PY{o}{.}\PY{n}{drop}\PY{p}{(}\PY{n}{dfDep}\PY{p}{[}\PY{n}{dfDep}\PY{p}{[}\PY{l+s+s1}{\PYZsq{}}\PY{l+s+s1}{valeur\PYZus{}fonciere}\PY{l+s+s1}{\PYZsq{}}\PY{p}{]}\PY{o}{\PYZgt{}}\PY{l+m+mi}{3000000} \PY{p}{]}\PY{o}{.}\PY{n}{index}\PY{p}{,} \PY{n}{inplace}\PY{o}{=}\PY{k+kc}{True}\PY{p}{)}
\PY{n}{dfDep}\PY{o}{.}\PY{n}{shape}
\end{Verbatim}
\end{tcolorbox}

            \begin{tcolorbox}[breakable, size=fbox, boxrule=.5pt, pad at break*=1mm, opacityfill=0]
\prompt{Out}{outcolor}{11}{\boxspacing}
\begin{Verbatim}[commandchars=\\\{\}]
(99469, 40)
\end{Verbatim}
\end{tcolorbox}
        
    \begin{tcolorbox}[breakable, size=fbox, boxrule=1pt, pad at break*=1mm,colback=cellbackground, colframe=cellborder]
\prompt{In}{incolor}{12}{\boxspacing}
\begin{Verbatim}[commandchars=\\\{\}]
\PY{c+c1}{\PYZsh{}Suppression des variables qui sont nulles pour 80\PYZpc{} des valeurs}
\PY{n}{listVariables} \PY{o}{=} \PY{n}{dfDep}\PY{o}{.}\PY{n}{isnull}\PY{p}{(}\PY{p}{)}\PY{o}{.}\PY{n}{sum}\PY{p}{(}\PY{p}{)} \PY{o}{\PYZgt{}} \PY{p}{(}\PY{n}{dfDep}\PY{o}{.}\PY{n}{shape}\PY{p}{[}\PY{l+m+mi}{0}\PY{p}{]}\PY{o}{*}\PY{l+m+mf}{0.8}\PY{p}{)}
\PY{n}{listResultatsVarDrop} \PY{o}{=} \PY{p}{[}\PY{p}{]}
\PY{k}{for} \PY{n}{colname}\PY{p}{,} \PY{n}{serie} \PY{o+ow}{in} \PY{n}{listVariables}\PY{o}{.}\PY{n}{iteritems}\PY{p}{(}\PY{p}{)}\PY{p}{:}
    \PY{k}{if}\PY{p}{(}\PY{n}{serie} \PY{o}{==} \PY{k+kc}{True}\PY{p}{)}\PY{p}{:}
            \PY{n}{listResultatsVarDrop}\PY{o}{.}\PY{n}{append}\PY{p}{(}\PY{n}{colname}\PY{p}{)}
\PY{n}{listResultatsVarDrop}
\PY{n}{dfDep}\PY{o}{.}\PY{n}{drop}\PY{p}{(}\PY{n}{listResultatsVarDrop}\PY{p}{,} \PY{n}{inplace}\PY{o}{=}\PY{k+kc}{True}\PY{p}{,} \PY{n}{axis}\PY{o}{=}\PY{l+m+mi}{1}\PY{p}{)}
\PY{n}{dfDep}\PY{o}{.}\PY{n}{shape}
\end{Verbatim}
\end{tcolorbox}

            \begin{tcolorbox}[breakable, size=fbox, boxrule=.5pt, pad at break*=1mm, opacityfill=0]
\prompt{Out}{outcolor}{12}{\boxspacing}
\begin{Verbatim}[commandchars=\\\{\}]
(99469, 23)
\end{Verbatim}
\end{tcolorbox}
        
    \begin{tcolorbox}[breakable, size=fbox, boxrule=1pt, pad at break*=1mm,colback=cellbackground, colframe=cellborder]
\prompt{In}{incolor}{13}{\boxspacing}
\begin{Verbatim}[commandchars=\\\{\}]
\PY{c+c1}{\PYZsh{}Conversion des objets en string}
\PY{n}{dfDep}\PY{p}{[}\PY{l+s+s1}{\PYZsq{}}\PY{l+s+s1}{adresse\PYZus{}nom\PYZus{}voie}\PY{l+s+s1}{\PYZsq{}}\PY{p}{]} \PY{o}{=} \PY{n}{dfDep}\PY{p}{[}\PY{l+s+s1}{\PYZsq{}}\PY{l+s+s1}{adresse\PYZus{}nom\PYZus{}voie}\PY{l+s+s1}{\PYZsq{}}\PY{p}{]}\PY{o}{.}\PY{n}{astype}\PY{p}{(}\PY{l+s+s2}{\PYZdq{}}\PY{l+s+s2}{string}\PY{l+s+s2}{\PYZdq{}}\PY{p}{)}
\PY{n}{dfDep}\PY{p}{[}\PY{l+s+s1}{\PYZsq{}}\PY{l+s+s1}{adresse\PYZus{}numero}\PY{l+s+s1}{\PYZsq{}}\PY{p}{]} \PY{o}{=} \PY{n}{dfDep}\PY{p}{[}\PY{l+s+s1}{\PYZsq{}}\PY{l+s+s1}{adresse\PYZus{}numero}\PY{l+s+s1}{\PYZsq{}}\PY{p}{]}\PY{o}{.}\PY{n}{astype}\PY{p}{(}\PY{l+s+s2}{\PYZdq{}}\PY{l+s+s2}{string}\PY{l+s+s2}{\PYZdq{}}\PY{p}{)}
\PY{n}{dfDep}\PY{p}{[}\PY{l+s+s1}{\PYZsq{}}\PY{l+s+s1}{nom\PYZus{}commune}\PY{l+s+s1}{\PYZsq{}}\PY{p}{]} \PY{o}{=} \PY{n}{dfDep}\PY{p}{[}\PY{l+s+s1}{\PYZsq{}}\PY{l+s+s1}{nom\PYZus{}commune}\PY{l+s+s1}{\PYZsq{}}\PY{p}{]}\PY{o}{.}\PY{n}{astype}\PY{p}{(}\PY{l+s+s2}{\PYZdq{}}\PY{l+s+s2}{string}\PY{l+s+s2}{\PYZdq{}}\PY{p}{)}
\PY{n}{dfDep}\PY{p}{[}\PY{l+s+s1}{\PYZsq{}}\PY{l+s+s1}{adresse\PYZus{}complete}\PY{l+s+s1}{\PYZsq{}}\PY{p}{]}\PY{o}{=}\PY{n}{dfDep}\PY{p}{[}\PY{l+s+s1}{\PYZsq{}}\PY{l+s+s1}{adresse\PYZus{}numero}\PY{l+s+s1}{\PYZsq{}}\PY{p}{]}\PY{o}{+}\PY{l+s+s1}{\PYZsq{}}\PY{l+s+s1}{ }\PY{l+s+s1}{\PYZsq{}}\PY{o}{+}\PY{n}{dfDep}\PY{p}{[}\PY{l+s+s1}{\PYZsq{}}\PY{l+s+s1}{adresse\PYZus{}nom\PYZus{}voie}\PY{l+s+s1}{\PYZsq{}}\PY{p}{]}\PY{o}{+}\PY{l+s+s1}{\PYZsq{}}\PY{l+s+s1}{ , }\PY{l+s+s1}{\PYZsq{}}\PY{o}{+}\PY{n}{dfDep}\PY{p}{[}\PY{l+s+s1}{\PYZsq{}}\PY{l+s+s1}{nom\PYZus{}commune}\PY{l+s+s1}{\PYZsq{}}\PY{p}{]}
\end{Verbatim}
\end{tcolorbox}

    \begin{tcolorbox}[breakable, size=fbox, boxrule=1pt, pad at break*=1mm,colback=cellbackground, colframe=cellborder]
\prompt{In}{incolor}{14}{\boxspacing}
\begin{Verbatim}[commandchars=\\\{\}]
\PY{c+c1}{\PYZsh{}Suppression des données où le type de local est une dépendance}
\PY{n}{dfDep}\PY{o}{.}\PY{n}{drop}\PY{p}{(}\PY{n}{dfDep}\PY{p}{[}\PY{n}{dfDep}\PY{p}{[}\PY{l+s+s1}{\PYZsq{}}\PY{l+s+s1}{type\PYZus{}local}\PY{l+s+s1}{\PYZsq{}}\PY{p}{]}\PY{o}{==} \PY{l+s+s1}{\PYZsq{}}\PY{l+s+s1}{Dépendance}\PY{l+s+s1}{\PYZsq{}} \PY{p}{]}\PY{o}{.}\PY{n}{index}\PY{p}{,} \PY{n}{inplace}\PY{o}{=}\PY{k+kc}{True}\PY{p}{)}
\PY{n}{dfDep}\PY{o}{.}\PY{n}{shape}
\end{Verbatim}
\end{tcolorbox}

            \begin{tcolorbox}[breakable, size=fbox, boxrule=.5pt, pad at break*=1mm, opacityfill=0]
\prompt{Out}{outcolor}{14}{\boxspacing}
\begin{Verbatim}[commandchars=\\\{\}]
(59687, 24)
\end{Verbatim}
\end{tcolorbox}
        
    \begin{tcolorbox}[breakable, size=fbox, boxrule=1pt, pad at break*=1mm,colback=cellbackground, colframe=cellborder]
\prompt{In}{incolor}{15}{\boxspacing}
\begin{Verbatim}[commandchars=\\\{\}]
\PY{n}{dfDep}\PY{o}{.}\PY{n}{head}\PY{p}{(}\PY{l+m+mi}{10}\PY{p}{)}
\end{Verbatim}
\end{tcolorbox}

            \begin{tcolorbox}[breakable, size=fbox, boxrule=.5pt, pad at break*=1mm, opacityfill=0]
\prompt{Out}{outcolor}{15}{\boxspacing}
\begin{Verbatim}[commandchars=\\\{\}]
         id\_mutation date\_mutation  numero\_disposition nature\_mutation  \textbackslash{}
1379722  2022-514000    2022-01-04                   1           Vente
1379723  2022-514000    2022-01-04                   1           Vente
1379725  2022-514001    2022-01-06                   1           Vente
1379732  2022-514004    2022-01-05                   1           Vente
1379734  2022-514005    2022-01-05                   1           Vente
1379736  2022-514006    2022-01-07                   1           Vente
1379738  2022-514007    2022-01-06                   1           Vente
1379739  2022-514008    2022-01-04                   1           Vente
1379740  2022-514009    2022-01-04                   1           Vente
1379742  2022-514010    2022-01-06                   1           Vente

         valeur\_fonciere adresse\_numero         adresse\_nom\_voie  \textbackslash{}
1379722         580000.0           13.0             RUE DE SOFIA
1379723         580000.0           13.0             RUE DE SOFIA
1379725         605000.0           51.0              RUE CHARLOT
1379732         716250.0            6.0        RUE PAUL ESCUDIER
1379734         320000.0            4.0    RUE DU CHATEAU LANDON
1379736         320000.0          134.0              AV GAMBETTA
1379738         220000.0            9.0  RUE ELYSEE MENILMONTANT
1379739         280000.0           18.0            RUE DES HAIES
1379740         200000.0          195.0            RUE DE CRIMEE
1379742         677500.0           79.0      RUE DES GRAVILLIERS

        adresse\_code\_voie  code\_postal code\_commune  {\ldots} lot1\_surface\_carrez  \textbackslash{}
1379722              9002      75018.0        75118  {\ldots}                 NaN
1379723              9002      75018.0        75118  {\ldots}               61.00
1379725              1880      75003.0        75103  {\ldots}               40.66
1379732              7155      75009.0        75109  {\ldots}                 NaN
1379734              1924      75010.0        75110  {\ldots}                 NaN
1379736              3933      75020.0        75120  {\ldots}               32.52
1379738              3192      75020.0        75120  {\ldots}               34.53
1379739              4452      75020.0        75120  {\ldots}               24.59
1379740              2443      75019.0        75119  {\ldots}               27.64
1379742              4302      75003.0        75103  {\ldots}               57.12

        lot2\_numero nombre\_lots code\_type\_local   type\_local  \textbackslash{}
1379722          56           2             2.0  Appartement
1379723          58           3             2.0  Appartement
1379725         NaN           1             2.0  Appartement
1379732           3           3             2.0  Appartement
1379734          92           2             2.0  Appartement
1379736          50           2             2.0  Appartement
1379738          24           2             2.0  Appartement
1379739          83           2             2.0  Appartement
1379740          54           2             2.0  Appartement
1379742           4           2             2.0  Appartement

        surface\_reelle\_bati  nombre\_pieces\_principales  longitude   latitude  \textbackslash{}
1379722                20.0                        2.0   2.348168  48.884490
1379723                25.0                        2.0   2.348168  48.884490
1379725                42.0                        3.0   2.362871  48.863374
1379732                69.0                        3.0   2.332324  48.880353
1379734                33.0                        2.0   2.362613  48.879658
1379736                29.0                        1.0   2.405513  48.872782
1379738                36.0                        2.0   2.386648  48.869335
1379739                28.0                        2.0   2.400622  48.852508
1379740                27.0                        2.0   2.375845  48.891167
1379742                58.0                        3.0   2.353479  48.864674

                                          adresse\_complete
1379722       13.0 RUE DE SOFIA , Paris 18e Arrondissement
1379723       13.0 RUE DE SOFIA , Paris 18e Arrondissement
1379725         51.0 RUE CHARLOT , Paris 3e Arrondissement
1379732    6.0 RUE PAUL ESCUDIER , Paris 9e Arrondissement
1379734  4.0 RUE DU CHATEAU LANDON , Paris 10e Arrondis{\ldots}
1379736       134.0 AV GAMBETTA , Paris 20e Arrondissement
1379738  9.0 RUE ELYSEE MENILMONTANT , Paris 20e Arrond{\ldots}
1379739      18.0 RUE DES HAIES , Paris 20e Arrondissement
1379740     195.0 RUE DE CRIMEE , Paris 19e Arrondissement
1379742  79.0 RUE DES GRAVILLIERS , Paris 3e Arrondisse{\ldots}

[10 rows x 24 columns]
\end{Verbatim}
\end{tcolorbox}
        
    \begin{tcolorbox}[breakable, size=fbox, boxrule=1pt, pad at break*=1mm,colback=cellbackground, colframe=cellborder]
\prompt{In}{incolor}{16}{\boxspacing}
\begin{Verbatim}[commandchars=\\\{\}]
\PY{k}{def} \PY{n+nf}{AggregationSimilarData}\PY{p}{(}\PY{n}{df}\PY{p}{)}\PY{p}{:}
    
    \PY{c+c1}{\PYZsh{} Construction d\PYZsq{}un dictionnaire }
    \PY{c+c1}{\PYZsh{} où la clé est la chaine de caractère qui permet d\PYZsq{}indiquer que deux lignes sont similaires}
    \PY{c+c1}{\PYZsh{} où la valeur est l\PYZsq{}index dans le dataframe initial}
    \PY{n}{dict\PYZus{}similarData} \PY{o}{=} \PY{p}{\PYZob{}}\PY{p}{\PYZcb{}}
    \PY{k}{for} \PY{n}{index}\PY{p}{,}\PY{n}{series} \PY{o+ow}{in} \PY{n}{df}\PY{o}{.}\PY{n}{iterrows}\PY{p}{(}\PY{p}{)}\PY{p}{:}
        \PY{n}{keyRow} \PY{o}{=} \PY{n+nb}{str}\PY{p}{(}\PY{n}{series}\PY{p}{[}\PY{l+s+s1}{\PYZsq{}}\PY{l+s+s1}{date\PYZus{}mutation}\PY{l+s+s1}{\PYZsq{}}\PY{p}{]}\PY{p}{)}\PY{o}{+}\PY{l+s+s1}{\PYZsq{}}\PY{l+s+s1}{\PYZus{}}\PY{l+s+s1}{\PYZsq{}}\PY{o}{+}\PY{n+nb}{str}\PY{p}{(}\PY{n}{series}\PY{p}{[}\PY{l+s+s1}{\PYZsq{}}\PY{l+s+s1}{valeur\PYZus{}fonciere}\PY{l+s+s1}{\PYZsq{}}\PY{p}{]}\PY{p}{)}\PY{o}{+}\PY{l+s+s1}{\PYZsq{}}\PY{l+s+s1}{\PYZus{}}\PY{l+s+s1}{\PYZsq{}}\PY{o}{+}\PY{n}{series}\PY{p}{[}\PY{l+s+s1}{\PYZsq{}}\PY{l+s+s1}{adresse\PYZus{}complete}\PY{l+s+s1}{\PYZsq{}}\PY{p}{]}
        \PY{k}{if} \PY{n}{keyRow} \PY{o+ow}{in} \PY{n}{dict\PYZus{}similarData}\PY{p}{:}
            \PY{n}{listIndexSimilaire} \PY{o}{=} \PY{n}{dict\PYZus{}similarData}\PY{p}{[}\PY{n}{keyRow}\PY{p}{]}
            \PY{n}{listIndexSimilaire}\PY{o}{.}\PY{n}{append}\PY{p}{(}\PY{n}{index}\PY{p}{)}
        \PY{k}{else}\PY{p}{:}
            \PY{n}{listKeyRow} \PY{o}{=} \PY{n+nb}{list}\PY{p}{(}\PY{p}{)}\PY{p}{;}
            \PY{n}{listKeyRow}\PY{o}{.}\PY{n}{append}\PY{p}{(}\PY{n}{index}\PY{p}{)}
            \PY{n}{dict\PYZus{}similarData}\PY{p}{[}\PY{n}{keyRow}\PY{p}{]} \PY{o}{=} \PY{n}{listKeyRow}
    
    \PY{c+c1}{\PYZsh{}Suppression des valeurs dupliquées en prenant comme surface\PYZus{}reelle\PYZus{}bati le cumulé des surfaces}
    \PY{n}{listIndexASupprimer} \PY{o}{=} \PY{p}{[}\PY{p}{]}
    \PY{k}{for} \PY{n}{cle}\PY{p}{,}\PY{n}{listIndex} \PY{o+ow}{in} \PY{n}{dict\PYZus{}similarData}\PY{o}{.}\PY{n}{items}\PY{p}{(}\PY{p}{)}\PY{p}{:}
        \PY{k}{if}\PY{p}{(}\PY{n+nb}{len}\PY{p}{(}\PY{n}{listIndex}\PY{p}{)}\PY{o}{\PYZgt{}}\PY{l+m+mi}{1}\PY{p}{)}\PY{p}{:}
            \PY{n}{valSurfaceAgregee} \PY{o}{=} \PY{n}{df}\PY{o}{.}\PY{n}{at}\PY{p}{[}\PY{n}{listIndex}\PY{p}{[}\PY{l+m+mi}{0}\PY{p}{]}\PY{p}{,}\PY{l+s+s2}{\PYZdq{}}\PY{l+s+s2}{surface\PYZus{}reelle\PYZus{}bati}\PY{l+s+s2}{\PYZdq{}}\PY{p}{]}
            \PY{n}{val} \PY{o}{=} \PY{l+m+mi}{1}
            \PY{k}{while} \PY{p}{(}\PY{n}{val} \PY{o}{!=} \PY{n+nb}{len}\PY{p}{(}\PY{n}{listIndex}\PY{p}{)}\PY{p}{)}\PY{p}{:}
                \PY{n}{listIndexASupprimer}\PY{o}{.}\PY{n}{append}\PY{p}{(}\PY{n}{listIndex}\PY{p}{[}\PY{n}{val}\PY{p}{]}\PY{p}{)}
                \PY{n}{valSurfaceAgregee} \PY{o}{+}\PY{o}{=} \PY{n}{df}\PY{o}{.}\PY{n}{at}\PY{p}{[}\PY{n}{listIndex}\PY{p}{[}\PY{n}{val}\PY{p}{]}\PY{p}{,}\PY{l+s+s2}{\PYZdq{}}\PY{l+s+s2}{surface\PYZus{}reelle\PYZus{}bati}\PY{l+s+s2}{\PYZdq{}}\PY{p}{]}
                \PY{n}{val} \PY{o}{+}\PY{o}{=} \PY{l+m+mi}{1}
            \PY{n}{df}\PY{o}{.}\PY{n}{at}\PY{p}{[}\PY{n}{listIndex}\PY{p}{[}\PY{l+m+mi}{0}\PY{p}{]}\PY{p}{,}\PY{l+s+s2}{\PYZdq{}}\PY{l+s+s2}{surface\PYZus{}reelle\PYZus{}bati}\PY{l+s+s2}{\PYZdq{}}\PY{p}{]} \PY{o}{=} \PY{n}{valSurfaceAgregee}
    \PY{c+c1}{\PYZsh{}print(listIndexASupprimer)}
    \PY{n}{df}\PY{o}{.}\PY{n}{drop}\PY{p}{(}\PY{n}{listIndexASupprimer}\PY{p}{,} \PY{n}{inplace} \PY{o}{=} \PY{k+kc}{True}\PY{p}{,} \PY{n}{axis} \PY{o}{=} \PY{l+m+mi}{0}\PY{p}{)}
    \PY{n+nb}{print}\PY{p}{(}\PY{n}{df}\PY{o}{.}\PY{n}{shape}\PY{p}{)}
\end{Verbatim}
\end{tcolorbox}

    \hypertarget{gestion-des-doublons}{%
\subsection{Gestion des doublons}\label{gestion-des-doublons}}

    \begin{tcolorbox}[breakable, size=fbox, boxrule=1pt, pad at break*=1mm,colback=cellbackground, colframe=cellborder]
\prompt{In}{incolor}{17}{\boxspacing}
\begin{Verbatim}[commandchars=\\\{\}]
\PY{n}{AggregationSimilarData}\PY{p}{(}\PY{n}{dfDep}\PY{p}{)}
\end{Verbatim}
\end{tcolorbox}

    \begin{Verbatim}[commandchars=\\\{\}]
(54289, 24)
    \end{Verbatim}

    \hypertarget{gestion-des-variables-catuxe9gorielles}{%
\subsection{Gestion des variables
catégorielles}\label{gestion-des-variables-catuxe9gorielles}}

On regarde les valeurs uniques pour identifier les variables
catégorielles

    \begin{tcolorbox}[breakable, size=fbox, boxrule=1pt, pad at break*=1mm,colback=cellbackground, colframe=cellborder]
\prompt{In}{incolor}{18}{\boxspacing}
\begin{Verbatim}[commandchars=\\\{\}]
\PY{k}{for} \PY{n}{colname}\PY{p}{,} \PY{n}{serie} \PY{o+ow}{in} \PY{n}{dfDep}\PY{o}{.}\PY{n}{iteritems}\PY{p}{(}\PY{p}{)}\PY{p}{:}
    \PY{n+nb}{print}\PY{p}{(}\PY{n}{colname} \PY{o}{+} \PY{l+s+s2}{\PYZdq{}}\PY{l+s+s2}{ has }\PY{l+s+s2}{\PYZdq{}} \PY{o}{+} \PY{n+nb}{str}\PY{p}{(}\PY{n}{serie}\PY{o}{.}\PY{n}{drop\PYZus{}duplicates}\PY{p}{(}\PY{p}{)}\PY{o}{.}\PY{n}{shape}\PY{p}{[}\PY{l+m+mi}{0}\PY{p}{]}\PY{p}{)} \PY{o}{+} \PY{l+s+s2}{\PYZdq{}}\PY{l+s+s2}{ unique values.}\PY{l+s+s2}{\PYZdq{}}\PY{p}{)}
\end{Verbatim}
\end{tcolorbox}

    \begin{Verbatim}[commandchars=\\\{\}]
id\_mutation has 54084 unique values.
date\_mutation has 453 unique values.
numero\_disposition has 3 unique values.
nature\_mutation has 5 unique values.
valeur\_fonciere has 15118 unique values.
adresse\_numero has 376 unique values.
adresse\_nom\_voie has 3429 unique values.
adresse\_code\_voie has 3419 unique values.
code\_postal has 21 unique values.
code\_commune has 20 unique values.
nom\_commune has 20 unique values.
code\_departement has 1 unique values.
id\_parcelle has 25386 unique values.
lot1\_numero has 1872 unique values.
lot1\_surface\_carrez has 10423 unique values.
lot2\_numero has 1875 unique values.
nombre\_lots has 18 unique values.
code\_type\_local has 4 unique values.
type\_local has 4 unique values.
surface\_reelle\_bati has 497 unique values.
nombre\_pieces\_principales has 16 unique values.
longitude has 22986 unique values.
latitude has 21254 unique values.
adresse\_complete has 26739 unique values.
    \end{Verbatim}

    \begin{tcolorbox}[breakable, size=fbox, boxrule=1pt, pad at break*=1mm,colback=cellbackground, colframe=cellborder]
\prompt{In}{incolor}{19}{\boxspacing}
\begin{Verbatim}[commandchars=\\\{\}]
\PY{n}{dfDep}\PY{p}{[}\PY{l+s+s2}{\PYZdq{}}\PY{l+s+s2}{nature\PYZus{}mutation}\PY{l+s+s2}{\PYZdq{}}\PY{p}{]} \PY{o}{=} \PY{n}{pd}\PY{o}{.}\PY{n}{Categorical}\PY{p}{(}\PY{n}{dfDep}\PY{p}{[}\PY{l+s+s2}{\PYZdq{}}\PY{l+s+s2}{nature\PYZus{}mutation}\PY{l+s+s2}{\PYZdq{}}\PY{p}{]}\PY{p}{,} \PY{n}{ordered}\PY{o}{=}\PY{k+kc}{False}\PY{p}{)}
\PY{n}{dfDep}\PY{p}{[}\PY{l+s+s2}{\PYZdq{}}\PY{l+s+s2}{type\PYZus{}local}\PY{l+s+s2}{\PYZdq{}}\PY{p}{]} \PY{o}{=} \PY{n}{pd}\PY{o}{.}\PY{n}{Categorical}\PY{p}{(}\PY{n}{dfDep}\PY{p}{[}\PY{l+s+s2}{\PYZdq{}}\PY{l+s+s2}{type\PYZus{}local}\PY{l+s+s2}{\PYZdq{}}\PY{p}{]}\PY{p}{,} \PY{n}{ordered}\PY{o}{=}\PY{k+kc}{False}\PY{p}{)}
\PY{n}{dfDep}\PY{p}{[}\PY{l+s+s2}{\PYZdq{}}\PY{l+s+s2}{nombre\PYZus{}pieces\PYZus{}principales}\PY{l+s+s2}{\PYZdq{}}\PY{p}{]} \PY{o}{=} \PY{n}{pd}\PY{o}{.}\PY{n}{Categorical}\PY{p}{(}\PY{n}{dfDep}\PY{p}{[}\PY{l+s+s2}{\PYZdq{}}\PY{l+s+s2}{nombre\PYZus{}pieces\PYZus{}principales}\PY{l+s+s2}{\PYZdq{}}\PY{p}{]}\PY{p}{,} \PY{n}{ordered}\PY{o}{=}\PY{k+kc}{False}\PY{p}{)}
\PY{n}{dfDep}\PY{p}{[}\PY{l+s+s2}{\PYZdq{}}\PY{l+s+s2}{nom\PYZus{}commune}\PY{l+s+s2}{\PYZdq{}}\PY{p}{]} \PY{o}{=} \PY{n}{pd}\PY{o}{.}\PY{n}{Categorical}\PY{p}{(}\PY{n}{dfDep}\PY{p}{[}\PY{l+s+s2}{\PYZdq{}}\PY{l+s+s2}{nom\PYZus{}commune}\PY{l+s+s2}{\PYZdq{}}\PY{p}{]}\PY{p}{,} \PY{n}{ordered}\PY{o}{=}\PY{k+kc}{False}\PY{p}{)}
\end{Verbatim}
\end{tcolorbox}

    \begin{tcolorbox}[breakable, size=fbox, boxrule=1pt, pad at break*=1mm,colback=cellbackground, colframe=cellborder]
\prompt{In}{incolor}{20}{\boxspacing}
\begin{Verbatim}[commandchars=\\\{\}]
\PY{c+c1}{\PYZsh{}Suppression des variables qui semblent inutiles}
\PY{n}{dfDep}\PY{o}{.}\PY{n}{drop}\PY{p}{(}\PY{p}{[}\PY{l+s+s1}{\PYZsq{}}\PY{l+s+s1}{code\PYZus{}departement}\PY{l+s+s1}{\PYZsq{}}\PY{p}{,} \PY{l+s+s1}{\PYZsq{}}\PY{l+s+s1}{code\PYZus{}postal}\PY{l+s+s1}{\PYZsq{}}\PY{p}{,} \PY{l+s+s1}{\PYZsq{}}\PY{l+s+s1}{adresse\PYZus{}code\PYZus{}voie}\PY{l+s+s1}{\PYZsq{}}\PY{p}{,} \PY{l+s+s1}{\PYZsq{}}\PY{l+s+s1}{code\PYZus{}commune}\PY{l+s+s1}{\PYZsq{}}\PY{p}{,} \PY{l+s+s1}{\PYZsq{}}\PY{l+s+s1}{id\PYZus{}parcelle}\PY{l+s+s1}{\PYZsq{}}\PY{p}{,}\PY{l+s+s1}{\PYZsq{}}\PY{l+s+s1}{lot1\PYZus{}numero}\PY{l+s+s1}{\PYZsq{}}\PY{p}{,}\PY{l+s+s1}{\PYZsq{}}\PY{l+s+s1}{lot2\PYZus{}numero}\PY{l+s+s1}{\PYZsq{}}\PY{p}{,} \PY{l+s+s1}{\PYZsq{}}\PY{l+s+s1}{code\PYZus{}type\PYZus{}local}\PY{l+s+s1}{\PYZsq{}}\PY{p}{]}\PY{p}{,} \PY{n}{inplace}\PY{o}{=}\PY{k+kc}{True}\PY{p}{,} \PY{n}{axis}\PY{o}{=}\PY{l+m+mi}{1}\PY{p}{)}
\PY{n}{dfDep}\PY{o}{.}\PY{n}{shape}
\end{Verbatim}
\end{tcolorbox}

            \begin{tcolorbox}[breakable, size=fbox, boxrule=.5pt, pad at break*=1mm, opacityfill=0]
\prompt{Out}{outcolor}{20}{\boxspacing}
\begin{Verbatim}[commandchars=\\\{\}]
(54289, 16)
\end{Verbatim}
\end{tcolorbox}
        
    \hypertarget{typage-des-variables}{%
\subsection{Typage des variables}\label{typage-des-variables}}

    \begin{tcolorbox}[breakable, size=fbox, boxrule=1pt, pad at break*=1mm,colback=cellbackground, colframe=cellborder]
\prompt{In}{incolor}{21}{\boxspacing}
\begin{Verbatim}[commandchars=\\\{\}]
\PY{n}{dfDep}\PY{p}{[}\PY{l+s+s1}{\PYZsq{}}\PY{l+s+s1}{date\PYZus{}mutation}\PY{l+s+s1}{\PYZsq{}}\PY{p}{]} \PY{o}{=} \PY{n}{pd}\PY{o}{.}\PY{n}{to\PYZus{}datetime}\PY{p}{(}\PY{n}{dfDep}\PY{p}{[}\PY{l+s+s1}{\PYZsq{}}\PY{l+s+s1}{date\PYZus{}mutation}\PY{l+s+s1}{\PYZsq{}}\PY{p}{]}\PY{p}{,} \PY{n+nb}{format}\PY{o}{=}\PY{l+s+s1}{\PYZsq{}}\PY{l+s+s1}{\PYZpc{}}\PY{l+s+s1}{Y/}\PY{l+s+s1}{\PYZpc{}}\PY{l+s+s1}{m/}\PY{l+s+si}{\PYZpc{}d}\PY{l+s+s1}{\PYZsq{}}\PY{p}{)}
\end{Verbatim}
\end{tcolorbox}

    \begin{tcolorbox}[breakable, size=fbox, boxrule=1pt, pad at break*=1mm,colback=cellbackground, colframe=cellborder]
\prompt{In}{incolor}{22}{\boxspacing}
\begin{Verbatim}[commandchars=\\\{\}]
\PY{n}{dfDep}\PY{o}{.}\PY{n}{info}\PY{p}{(}\PY{p}{)}
\end{Verbatim}
\end{tcolorbox}

    \begin{Verbatim}[commandchars=\\\{\}]
<class 'pandas.core.frame.DataFrame'>
Int64Index: 54289 entries, 1379722 to 4375222
Data columns (total 16 columns):
 \#   Column                     Non-Null Count  Dtype
---  ------                     --------------  -----
 0   id\_mutation                54289 non-null  object
 1   date\_mutation              54289 non-null  datetime64[ns]
 2   numero\_disposition         54289 non-null  int64
 3   nature\_mutation            54289 non-null  category
 4   valeur\_fonciere            54289 non-null  float64
 5   adresse\_numero             54288 non-null  string
 6   adresse\_nom\_voie           54288 non-null  string
 7   nom\_commune                54289 non-null  category
 8   lot1\_surface\_carrez        32599 non-null  float64
 9   nombre\_lots                54289 non-null  int64
 10  type\_local                 53930 non-null  category
 11  surface\_reelle\_bati        53912 non-null  float64
 12  nombre\_pieces\_principales  53929 non-null  category
 13  longitude                  54289 non-null  float64
 14  latitude                   54289 non-null  float64
 15  adresse\_complete           54288 non-null  string
dtypes: category(4), datetime64[ns](1), float64(5), int64(2), object(1),
string(3)
memory usage: 5.6+ MB
    \end{Verbatim}

    \begin{tcolorbox}[breakable, size=fbox, boxrule=1pt, pad at break*=1mm,colback=cellbackground, colframe=cellborder]
\prompt{In}{incolor}{23}{\boxspacing}
\begin{Verbatim}[commandchars=\\\{\}]
\PY{c+c1}{\PYZsh{}Conversion des objets en string}
\PY{n}{dfDep}\PY{p}{[}\PY{l+s+s1}{\PYZsq{}}\PY{l+s+s1}{id\PYZus{}mutation}\PY{l+s+s1}{\PYZsq{}}\PY{p}{]} \PY{o}{=} \PY{n}{dfDep}\PY{p}{[}\PY{l+s+s1}{\PYZsq{}}\PY{l+s+s1}{id\PYZus{}mutation}\PY{l+s+s1}{\PYZsq{}}\PY{p}{]}\PY{o}{.}\PY{n}{astype}\PY{p}{(}\PY{l+s+s2}{\PYZdq{}}\PY{l+s+s2}{string}\PY{l+s+s2}{\PYZdq{}}\PY{p}{)}
\PY{n}{dfDep}\PY{o}{.}\PY{n}{info}\PY{p}{(}\PY{p}{)}
\end{Verbatim}
\end{tcolorbox}

    \begin{Verbatim}[commandchars=\\\{\}]
<class 'pandas.core.frame.DataFrame'>
Int64Index: 54289 entries, 1379722 to 4375222
Data columns (total 16 columns):
 \#   Column                     Non-Null Count  Dtype
---  ------                     --------------  -----
 0   id\_mutation                54289 non-null  string
 1   date\_mutation              54289 non-null  datetime64[ns]
 2   numero\_disposition         54289 non-null  int64
 3   nature\_mutation            54289 non-null  category
 4   valeur\_fonciere            54289 non-null  float64
 5   adresse\_numero             54288 non-null  string
 6   adresse\_nom\_voie           54288 non-null  string
 7   nom\_commune                54289 non-null  category
 8   lot1\_surface\_carrez        32599 non-null  float64
 9   nombre\_lots                54289 non-null  int64
 10  type\_local                 53930 non-null  category
 11  surface\_reelle\_bati        53912 non-null  float64
 12  nombre\_pieces\_principales  53929 non-null  category
 13  longitude                  54289 non-null  float64
 14  latitude                   54289 non-null  float64
 15  adresse\_complete           54288 non-null  string
dtypes: category(4), datetime64[ns](1), float64(5), int64(2), string(4)
memory usage: 5.6 MB
    \end{Verbatim}

    \hypertarget{exploration}{%
\section{Exploration}\label{exploration}}

\hypertarget{description-univariuxe9e}{%
\subsection{Description univariée}\label{description-univariuxe9e}}

\hypertarget{la-variable-de-temps}{%
\subsubsection{La variable de temps}\label{la-variable-de-temps}}

On effectue du feature engineering

    \begin{tcolorbox}[breakable, size=fbox, boxrule=1pt, pad at break*=1mm,colback=cellbackground, colframe=cellborder]
\prompt{In}{incolor}{24}{\boxspacing}
\begin{Verbatim}[commandchars=\\\{\}]
\PY{n}{dfDep}\PY{p}{[}\PY{l+s+s1}{\PYZsq{}}\PY{l+s+s1}{month}\PY{l+s+s1}{\PYZsq{}}\PY{p}{]}\PY{o}{=}\PY{n}{dfDep}\PY{p}{[}\PY{l+s+s2}{\PYZdq{}}\PY{l+s+s2}{date\PYZus{}mutation}\PY{l+s+s2}{\PYZdq{}}\PY{p}{]}\PY{o}{.}\PY{n}{apply}\PY{p}{(}\PY{k}{lambda} \PY{n}{x}\PY{p}{:} \PY{n}{x}\PY{o}{.}\PY{n}{month}\PY{p}{)}
\PY{n}{dfDep}\PY{p}{[}\PY{l+s+s1}{\PYZsq{}}\PY{l+s+s1}{day}\PY{l+s+s1}{\PYZsq{}}\PY{p}{]} \PY{o}{=} \PY{n}{dfDep}\PY{p}{[}\PY{l+s+s2}{\PYZdq{}}\PY{l+s+s2}{date\PYZus{}mutation}\PY{l+s+s2}{\PYZdq{}}\PY{p}{]}\PY{o}{.}\PY{n}{apply}\PY{p}{(}\PY{k}{lambda} \PY{n}{x}\PY{p}{:} \PY{n}{x}\PY{o}{.}\PY{n}{day}\PY{p}{)}
\PY{n}{dfDep}\PY{p}{[}\PY{l+s+s1}{\PYZsq{}}\PY{l+s+s1}{year}\PY{l+s+s1}{\PYZsq{}}\PY{p}{]} \PY{o}{=} \PY{n}{dfDep}\PY{p}{[}\PY{l+s+s2}{\PYZdq{}}\PY{l+s+s2}{date\PYZus{}mutation}\PY{l+s+s2}{\PYZdq{}}\PY{p}{]}\PY{o}{.}\PY{n}{apply}\PY{p}{(}\PY{k}{lambda} \PY{n}{x}\PY{p}{:} \PY{n}{x}\PY{o}{.}\PY{n}{year}\PY{p}{)}
\PY{n}{dfDep}\PY{p}{[}\PY{l+s+s2}{\PYZdq{}}\PY{l+s+s2}{month}\PY{l+s+s2}{\PYZdq{}}\PY{p}{]} \PY{o}{=} \PY{n}{pd}\PY{o}{.}\PY{n}{Categorical}\PY{p}{(}\PY{n}{dfDep}\PY{p}{[}\PY{l+s+s2}{\PYZdq{}}\PY{l+s+s2}{month}\PY{l+s+s2}{\PYZdq{}}\PY{p}{]}\PY{p}{,} \PY{n}{ordered}\PY{o}{=}\PY{k+kc}{True}\PY{p}{)}
\PY{n}{dfDep}\PY{p}{[}\PY{l+s+s2}{\PYZdq{}}\PY{l+s+s2}{day}\PY{l+s+s2}{\PYZdq{}}\PY{p}{]} \PY{o}{=} \PY{n}{pd}\PY{o}{.}\PY{n}{Categorical}\PY{p}{(}\PY{n}{dfDep}\PY{p}{[}\PY{l+s+s2}{\PYZdq{}}\PY{l+s+s2}{day}\PY{l+s+s2}{\PYZdq{}}\PY{p}{]}\PY{p}{,} \PY{n}{ordered}\PY{o}{=}\PY{k+kc}{True}\PY{p}{)}
\PY{n}{dfDep}\PY{p}{[}\PY{l+s+s2}{\PYZdq{}}\PY{l+s+s2}{year}\PY{l+s+s2}{\PYZdq{}}\PY{p}{]}\PY{o}{=} \PY{n}{pd}\PY{o}{.}\PY{n}{Categorical}\PY{p}{(}\PY{n}{dfDep}\PY{p}{[}\PY{l+s+s2}{\PYZdq{}}\PY{l+s+s2}{year}\PY{l+s+s2}{\PYZdq{}}\PY{p}{]}\PY{p}{,} \PY{n}{ordered}\PY{o}{=}\PY{k+kc}{True}\PY{p}{)}
\end{Verbatim}
\end{tcolorbox}

    On va représenter les variables catégorielles à travers des tableaux de
contingence ou des bars plots

    \begin{tcolorbox}[breakable, size=fbox, boxrule=1pt, pad at break*=1mm,colback=cellbackground, colframe=cellborder]
\prompt{In}{incolor}{25}{\boxspacing}
\begin{Verbatim}[commandchars=\\\{\}]
\PY{n}{dfDep}\PY{p}{[}\PY{l+s+s2}{\PYZdq{}}\PY{l+s+s2}{year}\PY{l+s+s2}{\PYZdq{}}\PY{p}{]}\PY{o}{.}\PY{n}{value\PYZus{}counts}\PY{p}{(}\PY{p}{)}
\end{Verbatim}
\end{tcolorbox}

            \begin{tcolorbox}[breakable, size=fbox, boxrule=.5pt, pad at break*=1mm, opacityfill=0]
\prompt{Out}{outcolor}{25}{\boxspacing}
\begin{Verbatim}[commandchars=\\\{\}]
2021    35393
2022    18896
Name: year, dtype: int64
\end{Verbatim}
\end{tcolorbox}
        
    On observe qu'on a pratiquement le double de données entre 2021 et 2022
ce qui est normal car on a une vision partielle de 2022 avec des données
jusqu'à Juin

    \begin{tcolorbox}[breakable, size=fbox, boxrule=1pt, pad at break*=1mm,colback=cellbackground, colframe=cellborder]
\prompt{In}{incolor}{26}{\boxspacing}
\begin{Verbatim}[commandchars=\\\{\}]
\PY{n}{dfDep}\PY{p}{[}\PY{l+s+s2}{\PYZdq{}}\PY{l+s+s2}{month}\PY{l+s+s2}{\PYZdq{}}\PY{p}{]}\PY{o}{.}\PY{n}{value\PYZus{}counts}\PY{p}{(}\PY{p}{)}\PY{o}{.}\PY{n}{sort\PYZus{}index}\PY{p}{(}\PY{p}{)}\PY{o}{.}\PY{n}{plot}\PY{p}{(}\PY{n}{kind}\PY{o}{=}\PY{l+s+s2}{\PYZdq{}}\PY{l+s+s2}{bar}\PY{l+s+s2}{\PYZdq{}}\PY{p}{)}
\PY{n}{plt}\PY{o}{.}\PY{n}{title}\PY{p}{(}\PY{l+s+s2}{\PYZdq{}}\PY{l+s+s2}{Distribution des dates par mois}\PY{l+s+s2}{\PYZdq{}}\PY{p}{)}
\PY{n}{plt}\PY{o}{.}\PY{n}{xlabel}\PY{p}{(}\PY{l+s+s2}{\PYZdq{}}\PY{l+s+s2}{Month}\PY{l+s+s2}{\PYZdq{}}\PY{p}{)}
\PY{n}{plt}\PY{o}{.}\PY{n}{ylabel}\PY{p}{(}\PY{l+s+s2}{\PYZdq{}}\PY{l+s+s2}{Count}\PY{l+s+s2}{\PYZdq{}}\PY{p}{)}
\PY{n}{plt}\PY{o}{.}\PY{n}{show}\PY{p}{(}\PY{p}{)}
\end{Verbatim}
\end{tcolorbox}

    \begin{center}
    \adjustimage{max size={0.9\linewidth}{0.9\paperheight}}{output_37_0.png}
    \end{center}
    { \hspace*{\fill} \\}
    
    On retrouve les données partielles. Par contre, il apparait difficile de
faire des conclusions sur les mois hormi que le mois d'août semble
relativement faible ce qui peut s'expliquer car les personnes sont en
vacances

    \begin{tcolorbox}[breakable, size=fbox, boxrule=1pt, pad at break*=1mm,colback=cellbackground, colframe=cellborder]
\prompt{In}{incolor}{27}{\boxspacing}
\begin{Verbatim}[commandchars=\\\{\}]
\PY{n}{dfDep}\PY{p}{[}\PY{l+s+s2}{\PYZdq{}}\PY{l+s+s2}{day}\PY{l+s+s2}{\PYZdq{}}\PY{p}{]}\PY{o}{.}\PY{n}{value\PYZus{}counts}\PY{p}{(}\PY{p}{)}\PY{o}{.}\PY{n}{sort\PYZus{}index}\PY{p}{(}\PY{p}{)}\PY{o}{.}\PY{n}{plot}\PY{p}{(}\PY{n}{kind}\PY{o}{=}\PY{l+s+s2}{\PYZdq{}}\PY{l+s+s2}{bar}\PY{l+s+s2}{\PYZdq{}}\PY{p}{)}
\PY{n}{plt}\PY{o}{.}\PY{n}{title}\PY{p}{(}\PY{l+s+s2}{\PYZdq{}}\PY{l+s+s2}{Distribution des dates par jour}\PY{l+s+s2}{\PYZdq{}}\PY{p}{)}
\PY{n}{plt}\PY{o}{.}\PY{n}{xlabel}\PY{p}{(}\PY{l+s+s2}{\PYZdq{}}\PY{l+s+s2}{Day}\PY{l+s+s2}{\PYZdq{}}\PY{p}{)}
\PY{n}{plt}\PY{o}{.}\PY{n}{ylabel}\PY{p}{(}\PY{l+s+s2}{\PYZdq{}}\PY{l+s+s2}{Count}\PY{l+s+s2}{\PYZdq{}}\PY{p}{)}
\PY{n}{plt}\PY{o}{.}\PY{n}{show}\PY{p}{(}\PY{p}{)}
\end{Verbatim}
\end{tcolorbox}

    \begin{center}
    \adjustimage{max size={0.9\linewidth}{0.9\paperheight}}{output_40_0.png}
    \end{center}
    { \hspace*{\fill} \\}
    
    Il semblerait qu'il y ait plus de ventes en milieu et fin de mois

    \hypertarget{la-cible-de-notre-moduxe8le}{%
\subsubsection{La cible de notre
modèle}\label{la-cible-de-notre-moduxe8le}}

    \begin{tcolorbox}[breakable, size=fbox, boxrule=1pt, pad at break*=1mm,colback=cellbackground, colframe=cellborder]
\prompt{In}{incolor}{28}{\boxspacing}
\begin{Verbatim}[commandchars=\\\{\}]
\PY{n}{dfDep}\PY{p}{[}\PY{l+s+s2}{\PYZdq{}}\PY{l+s+s2}{valeur\PYZus{}fonciere}\PY{l+s+s2}{\PYZdq{}}\PY{p}{]}\PY{o}{.}\PY{n}{describe}\PY{p}{(}\PY{p}{)}
\end{Verbatim}
\end{tcolorbox}

            \begin{tcolorbox}[breakable, size=fbox, boxrule=.5pt, pad at break*=1mm, opacityfill=0]
\prompt{Out}{outcolor}{28}{\boxspacing}
\begin{Verbatim}[commandchars=\\\{\}]
count    5.428900e+04
mean     6.115892e+05
std      4.902127e+05
min      5.000000e+04
25\%      2.920000e+05
50\%      4.585697e+05
75\%      7.550000e+05
max      3.000000e+06
Name: valeur\_fonciere, dtype: float64
\end{Verbatim}
\end{tcolorbox}
        
    \begin{tcolorbox}[breakable, size=fbox, boxrule=1pt, pad at break*=1mm,colback=cellbackground, colframe=cellborder]
\prompt{In}{incolor}{29}{\boxspacing}
\begin{Verbatim}[commandchars=\\\{\}]
\PY{n}{sns}\PY{o}{.}\PY{n}{boxplot}\PY{p}{(}\PY{n}{x}\PY{o}{=}\PY{n}{dfDep}\PY{p}{[}\PY{l+s+s1}{\PYZsq{}}\PY{l+s+s1}{valeur\PYZus{}fonciere}\PY{l+s+s1}{\PYZsq{}}\PY{p}{]}\PY{p}{)}
\end{Verbatim}
\end{tcolorbox}

            \begin{tcolorbox}[breakable, size=fbox, boxrule=.5pt, pad at break*=1mm, opacityfill=0]
\prompt{Out}{outcolor}{29}{\boxspacing}
\begin{Verbatim}[commandchars=\\\{\}]
<AxesSubplot:xlabel='valeur\_fonciere'>
\end{Verbatim}
\end{tcolorbox}
        
    \begin{center}
    \adjustimage{max size={0.9\linewidth}{0.9\paperheight}}{output_44_1.png}
    \end{center}
    { \hspace*{\fill} \\}
    
    On a clairement des problèmes d'échelle avec des outliers à supprimer

    \begin{tcolorbox}[breakable, size=fbox, boxrule=1pt, pad at break*=1mm,colback=cellbackground, colframe=cellborder]
\prompt{In}{incolor}{30}{\boxspacing}
\begin{Verbatim}[commandchars=\\\{\}]
\PY{c+c1}{\PYZsh{}Methode Remove outliers pour une loi biaisée}
\PY{k}{def} \PY{n+nf}{removeOutliers}\PY{p}{(}\PY{n}{variable}\PY{p}{)}\PY{p}{:}
    \PY{n+nb}{print}\PY{p}{(}\PY{l+s+s2}{\PYZdq{}}\PY{l+s+s2}{avant }\PY{l+s+s2}{\PYZdq{}}\PY{p}{,} \PY{n}{dfDep}\PY{o}{.}\PY{n}{shape}\PY{p}{)}
    \PY{n}{Q1} \PY{o}{=} \PY{n}{dfDep}\PY{p}{[}\PY{n}{variable}\PY{p}{]}\PY{o}{.}\PY{n}{quantile}\PY{p}{(}\PY{l+m+mf}{0.25}\PY{p}{)}
    \PY{n}{Q3} \PY{o}{=} \PY{n}{dfDep}\PY{p}{[}\PY{n}{variable}\PY{p}{]}\PY{o}{.}\PY{n}{quantile}\PY{p}{(}\PY{l+m+mf}{0.75}\PY{p}{)}
    \PY{n}{IQR} \PY{o}{=} \PY{n}{Q3} \PY{o}{\PYZhy{}} \PY{n}{Q1}
    \PY{n}{dfDep}\PY{o}{.}\PY{n}{drop}\PY{p}{(}\PY{n}{dfDep}\PY{p}{[}\PY{p}{(}\PY{n}{dfDep}\PY{p}{[}\PY{n}{variable}\PY{p}{]}\PY{o}{\PYZlt{}}\PY{n}{Q1} \PY{o}{\PYZhy{}} \PY{l+m+mf}{1.5}\PY{o}{*}\PY{n}{IQR}\PY{p}{)} \PY{o}{|} \PY{p}{(}\PY{n}{dfDep}\PY{p}{[}\PY{n}{variable}\PY{p}{]}\PY{o}{\PYZgt{}}\PY{n}{Q3} \PY{o}{+} \PY{l+m+mf}{1.5}\PY{o}{*}\PY{n}{IQR}\PY{p}{)}\PY{p}{]}\PY{o}{.}\PY{n}{index}\PY{p}{,} \PY{n}{inplace}\PY{o}{=}\PY{k+kc}{True}\PY{p}{)}
    \PY{n+nb}{print}\PY{p}{(}\PY{l+s+s2}{\PYZdq{}}\PY{l+s+s2}{après }\PY{l+s+s2}{\PYZdq{}}\PY{p}{,}\PY{n}{dfDep}\PY{o}{.}\PY{n}{shape}\PY{p}{)}
\end{Verbatim}
\end{tcolorbox}

    \begin{tcolorbox}[breakable, size=fbox, boxrule=1pt, pad at break*=1mm,colback=cellbackground, colframe=cellborder]
\prompt{In}{incolor}{31}{\boxspacing}
\begin{Verbatim}[commandchars=\\\{\}]
\PY{n}{removeOutliers}\PY{p}{(}\PY{l+s+s1}{\PYZsq{}}\PY{l+s+s1}{valeur\PYZus{}fonciere}\PY{l+s+s1}{\PYZsq{}}\PY{p}{)}
\end{Verbatim}
\end{tcolorbox}

    \begin{Verbatim}[commandchars=\\\{\}]
avant  (54289, 19)
après  (50446, 19)
    \end{Verbatim}

    \begin{tcolorbox}[breakable, size=fbox, boxrule=1pt, pad at break*=1mm,colback=cellbackground, colframe=cellborder]
\prompt{In}{incolor}{32}{\boxspacing}
\begin{Verbatim}[commandchars=\\\{\}]
\PY{c+c1}{\PYZsh{}Methode Remove outliers par une loi normale}
\PY{c+c1}{\PYZsh{}Mean = dfDep[\PYZsq{}valeur\PYZus{}fonciere\PYZsq{}].mean()}
\PY{c+c1}{\PYZsh{}StandardDeviation = dfDep[\PYZsq{}valeur\PYZus{}fonciere\PYZsq{}].std()}
\PY{c+c1}{\PYZsh{}dfDep.drop(dfDep[(dfDep[\PYZsq{}valeur\PYZus{}fonciere\PYZsq{}]\PYZlt{}Mean \PYZhy{} 3*StandardDeviation) | (dfDep[\PYZsq{}valeur\PYZus{}fonciere\PYZsq{}]\PYZgt{}Mean + 3*StandardDeviation)].index, inplace=True)}
\PY{c+c1}{\PYZsh{}dfDep.shape}
\end{Verbatim}
\end{tcolorbox}

    \begin{tcolorbox}[breakable, size=fbox, boxrule=1pt, pad at break*=1mm,colback=cellbackground, colframe=cellborder]
\prompt{In}{incolor}{33}{\boxspacing}
\begin{Verbatim}[commandchars=\\\{\}]
\PY{n}{sns}\PY{o}{.}\PY{n}{boxplot}\PY{p}{(}\PY{n}{x}\PY{o}{=}\PY{n}{dfDep}\PY{p}{[}\PY{l+s+s1}{\PYZsq{}}\PY{l+s+s1}{valeur\PYZus{}fonciere}\PY{l+s+s1}{\PYZsq{}}\PY{p}{]}\PY{p}{)}
\PY{n}{dfDep}\PY{p}{[}\PY{l+s+s1}{\PYZsq{}}\PY{l+s+s1}{valeur\PYZus{}fonciere}\PY{l+s+s1}{\PYZsq{}}\PY{p}{]}\PY{o}{.}\PY{n}{describe}\PY{p}{(}\PY{p}{)}
\end{Verbatim}
\end{tcolorbox}

            \begin{tcolorbox}[breakable, size=fbox, boxrule=.5pt, pad at break*=1mm, opacityfill=0]
\prompt{Out}{outcolor}{33}{\boxspacing}
\begin{Verbatim}[commandchars=\\\{\}]
count    5.044600e+04
mean     5.078709e+05
std      3.049878e+05
min      5.000000e+04
25\%      2.800000e+05
50\%      4.300000e+05
75\%      6.700000e+05
max      1.449300e+06
Name: valeur\_fonciere, dtype: float64
\end{Verbatim}
\end{tcolorbox}
        
    \begin{center}
    \adjustimage{max size={0.9\linewidth}{0.9\paperheight}}{output_49_1.png}
    \end{center}
    { \hspace*{\fill} \\}
    
    \hypertarget{les-autres-variables-quantitatives}{%
\subsubsection{Les autres variables
quantitatives}\label{les-autres-variables-quantitatives}}

    Regardons les variables abérrantes sur les max sur les m2 et mettons une
valeur max à 500 m2

    \begin{tcolorbox}[breakable, size=fbox, boxrule=1pt, pad at break*=1mm,colback=cellbackground, colframe=cellborder]
\prompt{In}{incolor}{34}{\boxspacing}
\begin{Verbatim}[commandchars=\\\{\}]
\PY{c+c1}{\PYZsh{}dfDep.drop(dfDep[dfDep[\PYZsq{}surface\PYZus{}reelle\PYZus{}bati\PYZsq{}]\PYZgt{}500].index, inplace=True, axis=0)}
\PY{c+c1}{\PYZsh{}dfDep.drop(dfDep[dfDep[\PYZsq{}lot1\PYZus{}surface\PYZus{}carrez\PYZsq{}]\PYZgt{}500].index, inplace=True, axis=0)}
\PY{c+c1}{\PYZsh{}dfDep.shape}
\PY{c+c1}{\PYZsh{}removeOutliers(\PYZsq{}surface\PYZus{}reelle\PYZus{}bati\PYZsq{})}
\PY{c+c1}{\PYZsh{}dfDep.describe()}
\end{Verbatim}
\end{tcolorbox}

    \begin{tcolorbox}[breakable, size=fbox, boxrule=1pt, pad at break*=1mm,colback=cellbackground, colframe=cellborder]
\prompt{In}{incolor}{35}{\boxspacing}
\begin{Verbatim}[commandchars=\\\{\}]
\PY{c+c1}{\PYZsh{}removeOutliers(\PYZsq{}lot1\PYZus{}surface\PYZus{}carrez\PYZsq{})}
\end{Verbatim}
\end{tcolorbox}

    \begin{tcolorbox}[breakable, size=fbox, boxrule=1pt, pad at break*=1mm,colback=cellbackground, colframe=cellborder]
\prompt{In}{incolor}{36}{\boxspacing}
\begin{Verbatim}[commandchars=\\\{\}]
\PY{n}{dfDep}\PY{o}{.}\PY{n}{describe}\PY{p}{(}\PY{p}{)}
\end{Verbatim}
\end{tcolorbox}

            \begin{tcolorbox}[breakable, size=fbox, boxrule=.5pt, pad at break*=1mm, opacityfill=0]
\prompt{Out}{outcolor}{36}{\boxspacing}
\begin{Verbatim}[commandchars=\\\{\}]
       numero\_disposition  valeur\_fonciere  lot1\_surface\_carrez   nombre\_lots  \textbackslash{}
count        50446.000000     5.044600e+04         30741.000000  50446.000000
mean             1.002795     5.078709e+05            45.776034      1.675534
std              0.053541     3.049878e+05            61.569145      0.824916
min              1.000000     5.000000e+04             1.000000      0.000000
25\%              1.000000     2.800000e+05            25.580000      1.000000
50\%              1.000000     4.300000e+05            38.750000      2.000000
75\%              1.000000     6.700000e+05            59.270000      2.000000
max              3.000000     1.449300e+06          7392.000000     34.000000

       surface\_reelle\_bati     longitude      latitude
count         50112.000000  50446.000000  50446.000000
mean             51.898567      2.342636     48.862166
std              66.661178      0.037447      0.020222
min               2.000000      2.255896     48.818759
25\%              28.000000      2.316022     48.845022
50\%              43.000000      2.345814     48.862759
75\%              65.000000      2.373166     48.880174
max            7842.000000      2.412825     48.900565
\end{Verbatim}
\end{tcolorbox}
        
    \begin{tcolorbox}[breakable, size=fbox, boxrule=1pt, pad at break*=1mm,colback=cellbackground, colframe=cellborder]
\prompt{In}{incolor}{37}{\boxspacing}
\begin{Verbatim}[commandchars=\\\{\}]
\PY{n}{dfDep}\PY{p}{[}\PY{l+s+s2}{\PYZdq{}}\PY{l+s+s2}{prix m2}\PY{l+s+s2}{\PYZdq{}}\PY{p}{]}\PY{o}{=}\PY{n}{dfDep}\PY{p}{[}\PY{l+s+s2}{\PYZdq{}}\PY{l+s+s2}{valeur\PYZus{}fonciere}\PY{l+s+s2}{\PYZdq{}}\PY{p}{]}\PY{o}{/}\PY{n}{dfDep}\PY{p}{[}\PY{l+s+s2}{\PYZdq{}}\PY{l+s+s2}{surface\PYZus{}reelle\PYZus{}bati}\PY{l+s+s2}{\PYZdq{}}\PY{p}{]}
\PY{n}{sns}\PY{o}{.}\PY{n}{boxplot}\PY{p}{(}\PY{n}{x}\PY{o}{=}\PY{n}{dfDep}\PY{p}{[}\PY{l+s+s1}{\PYZsq{}}\PY{l+s+s1}{prix m2}\PY{l+s+s1}{\PYZsq{}}\PY{p}{]}\PY{p}{)}
\PY{n}{dfDep}\PY{p}{[}\PY{l+s+s2}{\PYZdq{}}\PY{l+s+s2}{prix m2}\PY{l+s+s2}{\PYZdq{}}\PY{p}{]}\PY{o}{.}\PY{n}{describe}\PY{p}{(}\PY{p}{)}
\end{Verbatim}
\end{tcolorbox}

            \begin{tcolorbox}[breakable, size=fbox, boxrule=.5pt, pad at break*=1mm, opacityfill=0]
\prompt{Out}{outcolor}{37}{\boxspacing}
\begin{Verbatim}[commandchars=\\\{\}]
count     50112.000000
mean      10639.642826
std        3691.344154
min          24.323322
25\%        9000.000000
50\%       10557.634033
75\%       12117.835294
max      130000.000000
Name: prix m2, dtype: float64
\end{Verbatim}
\end{tcolorbox}
        
    \begin{center}
    \adjustimage{max size={0.9\linewidth}{0.9\paperheight}}{output_55_1.png}
    \end{center}
    { \hspace*{\fill} \\}
    
    \begin{tcolorbox}[breakable, size=fbox, boxrule=1pt, pad at break*=1mm,colback=cellbackground, colframe=cellborder]
\prompt{In}{incolor}{38}{\boxspacing}
\begin{Verbatim}[commandchars=\\\{\}]
\PY{n}{removeOutliers}\PY{p}{(}\PY{l+s+s2}{\PYZdq{}}\PY{l+s+s2}{prix m2}\PY{l+s+s2}{\PYZdq{}}\PY{p}{)}
\end{Verbatim}
\end{tcolorbox}

    \begin{Verbatim}[commandchars=\\\{\}]
avant  (50446, 20)
après  (47154, 20)
    \end{Verbatim}

    \begin{tcolorbox}[breakable, size=fbox, boxrule=1pt, pad at break*=1mm,colback=cellbackground, colframe=cellborder]
\prompt{In}{incolor}{39}{\boxspacing}
\begin{Verbatim}[commandchars=\\\{\}]
\PY{n}{dfDep}\PY{p}{[}\PY{l+s+s2}{\PYZdq{}}\PY{l+s+s2}{prix m2}\PY{l+s+s2}{\PYZdq{}}\PY{p}{]}\PY{o}{=}\PY{n}{dfDep}\PY{p}{[}\PY{l+s+s2}{\PYZdq{}}\PY{l+s+s2}{valeur\PYZus{}fonciere}\PY{l+s+s2}{\PYZdq{}}\PY{p}{]}\PY{o}{/}\PY{n}{dfDep}\PY{p}{[}\PY{l+s+s2}{\PYZdq{}}\PY{l+s+s2}{surface\PYZus{}reelle\PYZus{}bati}\PY{l+s+s2}{\PYZdq{}}\PY{p}{]}
\PY{n}{sns}\PY{o}{.}\PY{n}{boxplot}\PY{p}{(}\PY{n}{x}\PY{o}{=}\PY{n}{dfDep}\PY{p}{[}\PY{l+s+s1}{\PYZsq{}}\PY{l+s+s1}{prix m2}\PY{l+s+s1}{\PYZsq{}}\PY{p}{]}\PY{p}{)}
\PY{n}{dfDep}\PY{o}{.}\PY{n}{describe}\PY{p}{(}\PY{p}{)}
\end{Verbatim}
\end{tcolorbox}

            \begin{tcolorbox}[breakable, size=fbox, boxrule=.5pt, pad at break*=1mm, opacityfill=0]
\prompt{Out}{outcolor}{39}{\boxspacing}
\begin{Verbatim}[commandchars=\\\{\}]
       numero\_disposition  valeur\_fonciere  lot1\_surface\_carrez   nombre\_lots  \textbackslash{}
count        47154.000000     4.715400e+04         29010.000000  47154.000000
mean             1.002312     5.095385e+05            45.737030      1.679751
std              0.048899     2.988223e+05            62.355009      0.779361
min              1.000000     5.000000e+04             1.000000      0.000000
25\%              1.000000     2.870000e+05            25.840000      1.000000
50\%              1.000000     4.330000e+05            38.940000      2.000000
75\%              1.000000     6.686625e+05            59.220000      2.000000
max              3.000000     1.449300e+06          7392.000000     16.000000

       surface\_reelle\_bati     longitude      latitude       prix m2
count         46820.000000  47154.000000  47154.000000  46820.000000
mean             49.006279      2.343005     48.862179  10560.249890
std              28.651969      0.037685      0.020387   2317.756634
min               4.000000      2.255896     48.818759   4326.923077
25\%              28.000000      2.316104     48.844722   9146.073718
50\%              42.000000      2.346336     48.862890  10566.666667
75\%              64.000000      2.373824     48.880446  12000.000000
max             300.000000      2.412825     48.900565  16791.044776
\end{Verbatim}
\end{tcolorbox}
        
    \begin{center}
    \adjustimage{max size={0.9\linewidth}{0.9\paperheight}}{output_57_1.png}
    \end{center}
    { \hspace*{\fill} \\}
    
    \begin{tcolorbox}[breakable, size=fbox, boxrule=1pt, pad at break*=1mm,colback=cellbackground, colframe=cellborder]
\prompt{In}{incolor}{40}{\boxspacing}
\begin{Verbatim}[commandchars=\\\{\}]
\PY{n}{removeOutliers}\PY{p}{(}\PY{l+s+s2}{\PYZdq{}}\PY{l+s+s2}{lot1\PYZus{}surface\PYZus{}carrez}\PY{l+s+s2}{\PYZdq{}}\PY{p}{)}
\PY{n}{sns}\PY{o}{.}\PY{n}{boxplot}\PY{p}{(}\PY{n}{x}\PY{o}{=}\PY{n}{dfDep}\PY{p}{[}\PY{l+s+s1}{\PYZsq{}}\PY{l+s+s1}{lot1\PYZus{}surface\PYZus{}carrez}\PY{l+s+s1}{\PYZsq{}}\PY{p}{]}\PY{p}{)}
\PY{n}{dfDep}\PY{o}{.}\PY{n}{describe}\PY{p}{(}\PY{p}{)}
\end{Verbatim}
\end{tcolorbox}

    \begin{Verbatim}[commandchars=\\\{\}]
avant  (47154, 20)
après  (46482, 20)
    \end{Verbatim}

            \begin{tcolorbox}[breakable, size=fbox, boxrule=.5pt, pad at break*=1mm, opacityfill=0]
\prompt{Out}{outcolor}{40}{\boxspacing}
\begin{Verbatim}[commandchars=\\\{\}]
       numero\_disposition  valeur\_fonciere  lot1\_surface\_carrez   nombre\_lots  \textbackslash{}
count        46482.000000     4.648200e+04         28338.000000  46482.000000
mean             1.002302     4.996812e+05            43.209783      1.676864
std              0.048814     2.883507e+05            23.095905      0.774179
min              1.000000     5.000000e+04             1.000000      0.000000
25\%              1.000000     2.850000e+05            25.530000      1.000000
50\%              1.000000     4.290000e+05            38.160000      2.000000
75\%              1.000000     6.530150e+05            57.170000      2.000000
max              3.000000     1.449200e+06           109.270000     16.000000

       surface\_reelle\_bati     longitude      latitude       prix m2
count         46149.000000  46482.000000  46482.000000  46149.000000
mean             47.905610      2.343216     48.862159  10570.283401
std              27.152457      0.037587      0.020415   2317.846738
min               4.000000      2.255896     48.818759   4326.923077
25\%              28.000000      2.316626     48.844628   9155.172414
50\%              42.000000      2.346446     48.862869  10573.333333
75\%              63.000000      2.373929     48.880462  12000.000000
max             300.000000      2.412825     48.900565  16791.044776
\end{Verbatim}
\end{tcolorbox}
        
    \begin{center}
    \adjustimage{max size={0.9\linewidth}{0.9\paperheight}}{output_58_2.png}
    \end{center}
    { \hspace*{\fill} \\}
    
    Les données sur les variables quantitatives semblent cohérentes en terme
de grandeur suite à différentes suppressions des données atypiques.

    \begin{tcolorbox}[breakable, size=fbox, boxrule=1pt, pad at break*=1mm,colback=cellbackground, colframe=cellborder]
\prompt{In}{incolor}{41}{\boxspacing}
\begin{Verbatim}[commandchars=\\\{\}]
\PY{n}{dfDep}\PY{o}{.}\PY{n}{drop}\PY{p}{(}\PY{p}{[}\PY{l+s+s2}{\PYZdq{}}\PY{l+s+s2}{prix m2}\PY{l+s+s2}{\PYZdq{}}\PY{p}{]}\PY{p}{,}\PY{n}{inplace}\PY{o}{=}\PY{k+kc}{True}\PY{p}{,}\PY{n}{axis}\PY{o}{=}\PY{l+m+mi}{1}\PY{p}{)}
\end{Verbatim}
\end{tcolorbox}

    \hypertarget{les-variables-catuxe9gorielles}{%
\subsubsection{Les variables
catégorielles}\label{les-variables-catuxe9gorielles}}

    \begin{tcolorbox}[breakable, size=fbox, boxrule=1pt, pad at break*=1mm,colback=cellbackground, colframe=cellborder]
\prompt{In}{incolor}{42}{\boxspacing}
\begin{Verbatim}[commandchars=\\\{\}]
\PY{n}{dfDep}\PY{p}{[}\PY{l+s+s2}{\PYZdq{}}\PY{l+s+s2}{nature\PYZus{}mutation}\PY{l+s+s2}{\PYZdq{}}\PY{p}{]}\PY{o}{.}\PY{n}{value\PYZus{}counts}\PY{p}{(}\PY{p}{)}\PY{o}{.}\PY{n}{sort\PYZus{}index}\PY{p}{(}\PY{p}{)}\PY{o}{.}\PY{n}{plot}\PY{p}{(}\PY{n}{kind}\PY{o}{=}\PY{l+s+s2}{\PYZdq{}}\PY{l+s+s2}{bar}\PY{l+s+s2}{\PYZdq{}}\PY{p}{)}
\PY{n}{plt}\PY{o}{.}\PY{n}{title}\PY{p}{(}\PY{l+s+s2}{\PYZdq{}}\PY{l+s+s2}{Distribution selon les natures de mutation}\PY{l+s+s2}{\PYZdq{}}\PY{p}{)}
\PY{n}{plt}\PY{o}{.}\PY{n}{xlabel}\PY{p}{(}\PY{l+s+s2}{\PYZdq{}}\PY{l+s+s2}{nature mutation}\PY{l+s+s2}{\PYZdq{}}\PY{p}{)}
\PY{n}{plt}\PY{o}{.}\PY{n}{ylabel}\PY{p}{(}\PY{l+s+s2}{\PYZdq{}}\PY{l+s+s2}{Count}\PY{l+s+s2}{\PYZdq{}}\PY{p}{)}
\PY{n}{plt}\PY{o}{.}\PY{n}{show}\PY{p}{(}\PY{p}{)}
\end{Verbatim}
\end{tcolorbox}

    \begin{center}
    \adjustimage{max size={0.9\linewidth}{0.9\paperheight}}{output_62_0.png}
    \end{center}
    { \hspace*{\fill} \\}
    
    La plupart des biens sont des ventes. On va pouvoir supprimer cette
variable

    \begin{tcolorbox}[breakable, size=fbox, boxrule=1pt, pad at break*=1mm,colback=cellbackground, colframe=cellborder]
\prompt{In}{incolor}{43}{\boxspacing}
\begin{Verbatim}[commandchars=\\\{\}]
\PY{n}{dfDep}\PY{o}{.}\PY{n}{drop}\PY{p}{(}\PY{p}{[}\PY{l+s+s1}{\PYZsq{}}\PY{l+s+s1}{nature\PYZus{}mutation}\PY{l+s+s1}{\PYZsq{}}\PY{p}{]}\PY{p}{,} \PY{n}{inplace}\PY{o}{=}\PY{k+kc}{True}\PY{p}{,} \PY{n}{axis}\PY{o}{=}\PY{l+m+mi}{1}\PY{p}{)}
\end{Verbatim}
\end{tcolorbox}

    \begin{tcolorbox}[breakable, size=fbox, boxrule=1pt, pad at break*=1mm,colback=cellbackground, colframe=cellborder]
\prompt{In}{incolor}{44}{\boxspacing}
\begin{Verbatim}[commandchars=\\\{\}]
\PY{n}{dfDep}\PY{p}{[}\PY{l+s+s2}{\PYZdq{}}\PY{l+s+s2}{type\PYZus{}local}\PY{l+s+s2}{\PYZdq{}}\PY{p}{]}\PY{o}{.}\PY{n}{value\PYZus{}counts}\PY{p}{(}\PY{p}{)}\PY{o}{.}\PY{n}{sort\PYZus{}index}\PY{p}{(}\PY{p}{)}\PY{o}{.}\PY{n}{plot}\PY{p}{(}\PY{n}{kind}\PY{o}{=}\PY{l+s+s2}{\PYZdq{}}\PY{l+s+s2}{bar}\PY{l+s+s2}{\PYZdq{}}\PY{p}{)}
\PY{n}{plt}\PY{o}{.}\PY{n}{title}\PY{p}{(}\PY{l+s+s2}{\PYZdq{}}\PY{l+s+s2}{Distribution selon le type local}\PY{l+s+s2}{\PYZdq{}}\PY{p}{)}
\PY{n}{plt}\PY{o}{.}\PY{n}{xlabel}\PY{p}{(}\PY{l+s+s2}{\PYZdq{}}\PY{l+s+s2}{type local}\PY{l+s+s2}{\PYZdq{}}\PY{p}{)}
\PY{n}{plt}\PY{o}{.}\PY{n}{ylabel}\PY{p}{(}\PY{l+s+s2}{\PYZdq{}}\PY{l+s+s2}{Count}\PY{l+s+s2}{\PYZdq{}}\PY{p}{)}
\PY{n}{plt}\PY{o}{.}\PY{n}{show}\PY{p}{(}\PY{p}{)}
\end{Verbatim}
\end{tcolorbox}

    \begin{center}
    \adjustimage{max size={0.9\linewidth}{0.9\paperheight}}{output_65_0.png}
    \end{center}
    { \hspace*{\fill} \\}
    
    On peut se poser la question de la pertinence de cette variable car on a
essentiellement deux modalités qui jouent

    \begin{tcolorbox}[breakable, size=fbox, boxrule=1pt, pad at break*=1mm,colback=cellbackground, colframe=cellborder]
\prompt{In}{incolor}{45}{\boxspacing}
\begin{Verbatim}[commandchars=\\\{\}]
\PY{n}{dfDep}\PY{p}{[}\PY{l+s+s2}{\PYZdq{}}\PY{l+s+s2}{nombre\PYZus{}pieces\PYZus{}principales}\PY{l+s+s2}{\PYZdq{}}\PY{p}{]}\PY{o}{.}\PY{n}{value\PYZus{}counts}\PY{p}{(}\PY{p}{)}
\end{Verbatim}
\end{tcolorbox}

            \begin{tcolorbox}[breakable, size=fbox, boxrule=.5pt, pad at break*=1mm, opacityfill=0]
\prompt{Out}{outcolor}{45}{\boxspacing}
\begin{Verbatim}[commandchars=\\\{\}]
2.0     16205
1.0     12097
3.0     10428
4.0      3958
0.0      2512
5.0       846
6.0        89
7.0        13
8.0         6
9.0         1
10.0        1
13.0        1
15.0        1
17.0        1
11.0        0
Name: nombre\_pieces\_principales, dtype: int64
\end{Verbatim}
\end{tcolorbox}
        
    Ce champs apparait mal instancié avec des valeurs abberantes

    \begin{tcolorbox}[breakable, size=fbox, boxrule=1pt, pad at break*=1mm,colback=cellbackground, colframe=cellborder]
\prompt{In}{incolor}{46}{\boxspacing}
\begin{Verbatim}[commandchars=\\\{\}]
\PY{n}{dfDep}\PY{o}{.}\PY{n}{drop}\PY{p}{(}\PY{p}{[}\PY{l+s+s1}{\PYZsq{}}\PY{l+s+s1}{nombre\PYZus{}pieces\PYZus{}principales}\PY{l+s+s1}{\PYZsq{}}\PY{p}{]}\PY{p}{,} \PY{n}{inplace}\PY{o}{=}\PY{k+kc}{True}\PY{p}{,} \PY{n}{axis}\PY{o}{=}\PY{l+m+mi}{1}\PY{p}{)}
\end{Verbatim}
\end{tcolorbox}

    \begin{tcolorbox}[breakable, size=fbox, boxrule=1pt, pad at break*=1mm,colback=cellbackground, colframe=cellborder]
\prompt{In}{incolor}{47}{\boxspacing}
\begin{Verbatim}[commandchars=\\\{\}]
\PY{n}{dfDep}\PY{p}{[}\PY{l+s+s2}{\PYZdq{}}\PY{l+s+s2}{nom\PYZus{}commune}\PY{l+s+s2}{\PYZdq{}}\PY{p}{]}\PY{o}{.}\PY{n}{value\PYZus{}counts}\PY{p}{(}\PY{p}{)}
\end{Verbatim}
\end{tcolorbox}

            \begin{tcolorbox}[breakable, size=fbox, boxrule=.5pt, pad at break*=1mm, opacityfill=0]
\prompt{Out}{outcolor}{47}{\boxspacing}
\begin{Verbatim}[commandchars=\\\{\}]
Paris 18e Arrondissement    5247
Paris 15e Arrondissement    5173
Paris 11e Arrondissement    4162
Paris 17e Arrondissement    3960
Paris 20e Arrondissement    3318
Paris 16e Arrondissement    3209
Paris 19e Arrondissement    2845
Paris 12e Arrondissement    2785
Paris 10e Arrondissement    2691
Paris 14e Arrondissement    2493
Paris 13e Arrondissement    2410
Paris 9e Arrondissement     1811
Paris 5e Arrondissement     1181
Paris 3e Arrondissement      956
Paris 7e Arrondissement      882
Paris 6e Arrondissement      798
Paris 8e Arrondissement      743
Paris 2e Arrondissement      740
Paris 4e Arrondissement      676
Paris 1er Arrondissement     402
Name: nom\_commune, dtype: int64
\end{Verbatim}
\end{tcolorbox}
        
    \hypertarget{conclusion-sur-lanalyse-univariuxe9e}{%
\subsubsection{Conclusion sur l'analyse
univariée}\label{conclusion-sur-lanalyse-univariuxe9e}}

On a pu voir que : * le jeu de données est constituée d'un an et demi
d'historique * la target a été retravaillé pour éliminer les valeurs
aberrantes. Il reste néanmoins des points atypiques sur le boxplot * on
a rationnalisé certaines variables en enlevant des valeurs aberrantes ou
en les éliminant de l'analyse pour certaines varaibles catégorielles. *
on a introduit du feature ingeenering sur les dates

    \hypertarget{analyse-bivariuxe9e}{%
\subsection{Analyse bivariée}\label{analyse-bivariuxe9e}}

L'analyse bivariée va consister à regarder l'influence de différentes
variables sur la variable cible.

\hypertarget{les-variables-catuxe9gorielles}{%
\subsubsection{Les variables
catégorielles}\label{les-variables-catuxe9gorielles}}

    \begin{tcolorbox}[breakable, size=fbox, boxrule=1pt, pad at break*=1mm,colback=cellbackground, colframe=cellborder]
\prompt{In}{incolor}{48}{\boxspacing}
\begin{Verbatim}[commandchars=\\\{\}]
\PY{n}{sns}\PY{o}{.}\PY{n}{catplot}\PY{p}{(}\PY{n}{x}\PY{o}{=}\PY{l+s+s2}{\PYZdq{}}\PY{l+s+s2}{month}\PY{l+s+s2}{\PYZdq{}}\PY{p}{,} \PY{n}{y}\PY{o}{=}\PY{l+s+s2}{\PYZdq{}}\PY{l+s+s2}{valeur\PYZus{}fonciere}\PY{l+s+s2}{\PYZdq{}}\PY{p}{,} \PY{n}{kind}\PY{o}{=}\PY{l+s+s2}{\PYZdq{}}\PY{l+s+s2}{box}\PY{l+s+s2}{\PYZdq{}}\PY{p}{,} \PY{n}{data}\PY{o}{=}\PY{n}{dfDep}\PY{p}{)}
\end{Verbatim}
\end{tcolorbox}

            \begin{tcolorbox}[breakable, size=fbox, boxrule=.5pt, pad at break*=1mm, opacityfill=0]
\prompt{Out}{outcolor}{48}{\boxspacing}
\begin{Verbatim}[commandchars=\\\{\}]
<seaborn.axisgrid.FacetGrid at 0x16f219ac430>
\end{Verbatim}
\end{tcolorbox}
        
    \begin{center}
    \adjustimage{max size={0.9\linewidth}{0.9\paperheight}}{output_73_1.png}
    \end{center}
    { \hspace*{\fill} \\}
    
    Que cela soit en comparant month, year, day, les niveaux semblent
relativement identiques mais on a beaucoup de points atypiques

    \begin{tcolorbox}[breakable, size=fbox, boxrule=1pt, pad at break*=1mm,colback=cellbackground, colframe=cellborder]
\prompt{In}{incolor}{49}{\boxspacing}
\begin{Verbatim}[commandchars=\\\{\}]
\PY{n}{sns}\PY{o}{.}\PY{n}{catplot}\PY{p}{(}\PY{n}{x}\PY{o}{=}\PY{l+s+s2}{\PYZdq{}}\PY{l+s+s2}{type\PYZus{}local}\PY{l+s+s2}{\PYZdq{}}\PY{p}{,} \PY{n}{y}\PY{o}{=}\PY{l+s+s2}{\PYZdq{}}\PY{l+s+s2}{valeur\PYZus{}fonciere}\PY{l+s+s2}{\PYZdq{}}\PY{p}{,} \PY{n}{kind}\PY{o}{=}\PY{l+s+s2}{\PYZdq{}}\PY{l+s+s2}{box}\PY{l+s+s2}{\PYZdq{}}\PY{p}{,} \PY{n}{data}\PY{o}{=}\PY{n}{dfDep}\PY{p}{,} \PY{n}{orient}\PY{o}{=}\PY{l+s+s2}{\PYZdq{}}\PY{l+s+s2}{v}\PY{l+s+s2}{\PYZdq{}}\PY{p}{)}
\end{Verbatim}
\end{tcolorbox}

            \begin{tcolorbox}[breakable, size=fbox, boxrule=.5pt, pad at break*=1mm, opacityfill=0]
\prompt{Out}{outcolor}{49}{\boxspacing}
\begin{Verbatim}[commandchars=\\\{\}]
<seaborn.axisgrid.FacetGrid at 0x16f1f866ca0>
\end{Verbatim}
\end{tcolorbox}
        
    \begin{center}
    \adjustimage{max size={0.9\linewidth}{0.9\paperheight}}{output_75_1.png}
    \end{center}
    { \hspace*{\fill} \\}
    
    On peut constater que le prix des maisons est plus élevé que locaux
industriels qui sont eux-mêmes à un niveau équivalent par rapport aux
appartements en moyenne

    \begin{tcolorbox}[breakable, size=fbox, boxrule=1pt, pad at break*=1mm,colback=cellbackground, colframe=cellborder]
\prompt{In}{incolor}{50}{\boxspacing}
\begin{Verbatim}[commandchars=\\\{\}]
\PY{n}{sns}\PY{o}{.}\PY{n}{catplot}\PY{p}{(}\PY{n}{x}\PY{o}{=}\PY{l+s+s2}{\PYZdq{}}\PY{l+s+s2}{nom\PYZus{}commune}\PY{l+s+s2}{\PYZdq{}}\PY{p}{,} \PY{n}{y}\PY{o}{=}\PY{l+s+s2}{\PYZdq{}}\PY{l+s+s2}{valeur\PYZus{}fonciere}\PY{l+s+s2}{\PYZdq{}}\PY{p}{,} \PY{n}{kind}\PY{o}{=}\PY{l+s+s2}{\PYZdq{}}\PY{l+s+s2}{box}\PY{l+s+s2}{\PYZdq{}}\PY{p}{,} \PY{n}{orient}\PY{o}{=}\PY{l+s+s2}{\PYZdq{}}\PY{l+s+s2}{v}\PY{l+s+s2}{\PYZdq{}}\PY{p}{,}\PY{n}{data}\PY{o}{=}\PY{n}{dfDep}\PY{p}{)}
\end{Verbatim}
\end{tcolorbox}

            \begin{tcolorbox}[breakable, size=fbox, boxrule=.5pt, pad at break*=1mm, opacityfill=0]
\prompt{Out}{outcolor}{50}{\boxspacing}
\begin{Verbatim}[commandchars=\\\{\}]
<seaborn.axisgrid.FacetGrid at 0x16f234aa040>
\end{Verbatim}
\end{tcolorbox}
        
    \begin{center}
    \adjustimage{max size={0.9\linewidth}{0.9\paperheight}}{output_77_1.png}
    \end{center}
    { \hspace*{\fill} \\}
    
    Les arrondissements semblent avoir un effet sur les prix

    \hypertarget{les-variables-quantitatives}{%
\subsubsection{Les variables
quantitatives}\label{les-variables-quantitatives}}

    \begin{tcolorbox}[breakable, size=fbox, boxrule=1pt, pad at break*=1mm,colback=cellbackground, colframe=cellborder]
\prompt{In}{incolor}{51}{\boxspacing}
\begin{Verbatim}[commandchars=\\\{\}]
\PY{n}{dfDep}\PY{o}{.}\PY{n}{info}\PY{p}{(}\PY{p}{)}
\end{Verbatim}
\end{tcolorbox}

    \begin{Verbatim}[commandchars=\\\{\}]
<class 'pandas.core.frame.DataFrame'>
Int64Index: 46482 entries, 1379722 to 4375222
Data columns (total 17 columns):
 \#   Column               Non-Null Count  Dtype
---  ------               --------------  -----
 0   id\_mutation          46482 non-null  string
 1   date\_mutation        46482 non-null  datetime64[ns]
 2   numero\_disposition   46482 non-null  int64
 3   valeur\_fonciere      46482 non-null  float64
 4   adresse\_numero       46481 non-null  string
 5   adresse\_nom\_voie     46481 non-null  string
 6   nom\_commune          46482 non-null  category
 7   lot1\_surface\_carrez  28338 non-null  float64
 8   nombre\_lots          46482 non-null  int64
 9   type\_local           46159 non-null  category
 10  surface\_reelle\_bati  46149 non-null  float64
 11  longitude            46482 non-null  float64
 12  latitude             46482 non-null  float64
 13  adresse\_complete     46481 non-null  string
 14  month                46482 non-null  category
 15  day                  46482 non-null  category
 16  year                 46482 non-null  category
dtypes: category(5), datetime64[ns](1), float64(5), int64(2), string(4)
memory usage: 4.8 MB
    \end{Verbatim}

    \begin{tcolorbox}[breakable, size=fbox, boxrule=1pt, pad at break*=1mm,colback=cellbackground, colframe=cellborder]
\prompt{In}{incolor}{52}{\boxspacing}
\begin{Verbatim}[commandchars=\\\{\}]
\PY{n}{fig}\PY{p}{,}\PY{p}{(}\PY{n}{ax1}\PY{p}{,}\PY{n}{ax2}\PY{p}{,}\PY{n}{ax3}\PY{p}{)} \PY{o}{=} \PY{n}{plt}\PY{o}{.}\PY{n}{subplots}\PY{p}{(}\PY{n}{ncols}\PY{o}{=}\PY{l+m+mi}{3}\PY{p}{)}
\PY{n}{fig}\PY{o}{.}\PY{n}{set\PYZus{}size\PYZus{}inches}\PY{p}{(}\PY{l+m+mi}{14}\PY{p}{,}\PY{l+m+mi}{10}\PY{p}{)}
\PY{n}{sns}\PY{o}{.}\PY{n}{regplot}\PY{p}{(}\PY{n}{x}\PY{o}{=}\PY{l+s+s2}{\PYZdq{}}\PY{l+s+s2}{lot1\PYZus{}surface\PYZus{}carrez}\PY{l+s+s2}{\PYZdq{}}\PY{p}{,}\PY{n}{y}\PY{o}{=}\PY{l+s+s2}{\PYZdq{}}\PY{l+s+s2}{valeur\PYZus{}fonciere}\PY{l+s+s2}{\PYZdq{}}\PY{p}{,}\PY{n}{data}\PY{o}{=}\PY{n}{dfDep}\PY{p}{,} \PY{n}{ax}\PY{o}{=}\PY{n}{ax1}\PY{p}{)}
\PY{n}{sns}\PY{o}{.}\PY{n}{regplot}\PY{p}{(}\PY{n}{x}\PY{o}{=}\PY{l+s+s2}{\PYZdq{}}\PY{l+s+s2}{nombre\PYZus{}lots}\PY{l+s+s2}{\PYZdq{}}\PY{p}{,}\PY{n}{y}\PY{o}{=}\PY{l+s+s2}{\PYZdq{}}\PY{l+s+s2}{valeur\PYZus{}fonciere}\PY{l+s+s2}{\PYZdq{}}\PY{p}{,}\PY{n}{data}\PY{o}{=}\PY{n}{dfDep}\PY{p}{,}\PY{n}{ax}\PY{o}{=}\PY{n}{ax2}\PY{p}{)}
\PY{n}{sns}\PY{o}{.}\PY{n}{regplot}\PY{p}{(}\PY{n}{x}\PY{o}{=}\PY{l+s+s2}{\PYZdq{}}\PY{l+s+s2}{surface\PYZus{}reelle\PYZus{}bati}\PY{l+s+s2}{\PYZdq{}}\PY{p}{,}\PY{n}{y}\PY{o}{=}\PY{l+s+s2}{\PYZdq{}}\PY{l+s+s2}{valeur\PYZus{}fonciere}\PY{l+s+s2}{\PYZdq{}}\PY{p}{,}\PY{n}{data}\PY{o}{=}\PY{n}{dfDep}\PY{p}{,}\PY{n}{ax}\PY{o}{=}\PY{n}{ax3}\PY{p}{)}
\PY{c+c1}{\PYZsh{}sns.regplot(x=\PYZdq{}Prix m2\PYZdq{},y=\PYZdq{}valeur\PYZus{}fonciere\PYZdq{},data=dfDep,ax=ax4)}
\end{Verbatim}
\end{tcolorbox}

            \begin{tcolorbox}[breakable, size=fbox, boxrule=.5pt, pad at break*=1mm, opacityfill=0]
\prompt{Out}{outcolor}{52}{\boxspacing}
\begin{Verbatim}[commandchars=\\\{\}]
<AxesSubplot:xlabel='surface\_reelle\_bati', ylabel='valeur\_fonciere'>
\end{Verbatim}
\end{tcolorbox}
        
    \begin{center}
    \adjustimage{max size={0.9\linewidth}{0.9\paperheight}}{output_81_1.png}
    \end{center}
    { \hspace*{\fill} \\}
    
    La valeur foncière augmente avec la surface, le nombre de lots, la
surface réelle.

    \begin{tcolorbox}[breakable, size=fbox, boxrule=1pt, pad at break*=1mm,colback=cellbackground, colframe=cellborder]
\prompt{In}{incolor}{53}{\boxspacing}
\begin{Verbatim}[commandchars=\\\{\}]
\PY{n}{sns}\PY{o}{.}\PY{n}{heatmap}\PY{p}{(}\PY{n}{dfDep}\PY{o}{.}\PY{n}{corr}\PY{p}{(}\PY{p}{)}\PY{p}{,} \PY{n}{cmap}\PY{o}{=}\PY{l+s+s2}{\PYZdq{}}\PY{l+s+s2}{YlOrRd}\PY{l+s+s2}{\PYZdq{}}\PY{p}{)}
\PY{n}{plt}\PY{o}{.}\PY{n}{title}\PY{p}{(}\PY{l+s+s2}{\PYZdq{}}\PY{l+s+s2}{Corrélations des variables continues}\PY{l+s+s2}{\PYZdq{}}\PY{p}{)}
\PY{n}{plt}\PY{o}{.}\PY{n}{show}\PY{p}{(}\PY{p}{)}
\PY{n}{dfDep}\PY{o}{.}\PY{n}{corr}\PY{p}{(}\PY{p}{)}
\end{Verbatim}
\end{tcolorbox}

    \begin{center}
    \adjustimage{max size={0.9\linewidth}{0.9\paperheight}}{output_83_0.png}
    \end{center}
    { \hspace*{\fill} \\}
    
            \begin{tcolorbox}[breakable, size=fbox, boxrule=.5pt, pad at break*=1mm, opacityfill=0]
\prompt{Out}{outcolor}{53}{\boxspacing}
\begin{Verbatim}[commandchars=\\\{\}]
                     numero\_disposition  valeur\_fonciere  lot1\_surface\_carrez  \textbackslash{}
numero\_disposition             1.000000         0.024541             0.031832
valeur\_fonciere                0.024541         1.000000             0.909703
lot1\_surface\_carrez            0.031832         0.909703             1.000000
nombre\_lots                   -0.000242         0.227670             0.235166
surface\_reelle\_bati            0.031681         0.876837             0.915587
longitude                     -0.017176        -0.172863            -0.098811
latitude                       0.001994        -0.042408            -0.036007

                     nombre\_lots  surface\_reelle\_bati  longitude  latitude
numero\_disposition     -0.000242             0.031681  -0.017176  0.001994
valeur\_fonciere         0.227670             0.876837  -0.172863 -0.042408
lot1\_surface\_carrez     0.235166             0.915587  -0.098811 -0.036007
nombre\_lots             1.000000             0.227450   0.009932  0.003212
surface\_reelle\_bati     0.227450             1.000000  -0.106381 -0.041606
longitude               0.009932            -0.106381   1.000000  0.113772
latitude                0.003212            -0.041606   0.113772  1.000000
\end{Verbatim}
\end{tcolorbox}
        
    La valeur foncière est fortement corrélée à la surface carrez ou réelle,
faiblement au nombre de lots et aux coordonnées gps

    \hypertarget{conclusion-sur-lanalyse-bivariuxe9e}{%
\subsubsection{Conclusion sur l'analyse
bivariée}\label{conclusion-sur-lanalyse-bivariuxe9e}}

La target est sensible : * à la surface réelle ou carrez * au prix du m2
* à la commune Par contre, elle ne semble pas tellement sensible au
mois, jour, année. La difficulté vient surtout du nombre de points
atypiques importants.

    \hypertarget{analyse-multivariuxe9e}{%
\subsection{Analyse multivariée}\label{analyse-multivariuxe9e}}

    \begin{tcolorbox}[breakable, size=fbox, boxrule=1pt, pad at break*=1mm,colback=cellbackground, colframe=cellborder]
\prompt{In}{incolor}{54}{\boxspacing}
\begin{Verbatim}[commandchars=\\\{\}]
\PY{c+c1}{\PYZsh{}sns.relplot(x=\PYZdq{}lot1\PYZus{}surface\PYZus{}carrez\PYZdq{}, y=\PYZdq{}surface\PYZus{}reelle\PYZus{}bati\PYZdq{}, size=\PYZdq{}valeur\PYZus{}fonciere\PYZdq{}, sizes=(15, 100), data=dfDep);}
\PY{c+c1}{\PYZsh{}sns.catplot(x=\PYZdq{}lot1\PYZus{}surface\PYZus{}carrez\PYZdq{}, y=\PYZdq{}valeur\PYZus{}fonciere\PYZdq{}, hue=\PYZdq{}type\PYZus{}local\PYZdq{},kind=\PYZdq{}bar\PYZdq{},data=dfDep);}
\end{Verbatim}
\end{tcolorbox}

    \hypertarget{modelisation}{%
\section{Modelisation}\label{modelisation}}

    \hypertarget{preprocessing-pour-scikit-learn}{%
\subsection{Preprocessing pour
scikit-learn¶}\label{preprocessing-pour-scikit-learn}}

    \begin{tcolorbox}[breakable, size=fbox, boxrule=1pt, pad at break*=1mm,colback=cellbackground, colframe=cellborder]
\prompt{In}{incolor}{55}{\boxspacing}
\begin{Verbatim}[commandchars=\\\{\}]
\PY{n}{dfDep}\PY{o}{.}\PY{n}{info}\PY{p}{(}\PY{p}{)}
\end{Verbatim}
\end{tcolorbox}

    \begin{Verbatim}[commandchars=\\\{\}]
<class 'pandas.core.frame.DataFrame'>
Int64Index: 46482 entries, 1379722 to 4375222
Data columns (total 17 columns):
 \#   Column               Non-Null Count  Dtype
---  ------               --------------  -----
 0   id\_mutation          46482 non-null  string
 1   date\_mutation        46482 non-null  datetime64[ns]
 2   numero\_disposition   46482 non-null  int64
 3   valeur\_fonciere      46482 non-null  float64
 4   adresse\_numero       46481 non-null  string
 5   adresse\_nom\_voie     46481 non-null  string
 6   nom\_commune          46482 non-null  category
 7   lot1\_surface\_carrez  28338 non-null  float64
 8   nombre\_lots          46482 non-null  int64
 9   type\_local           46159 non-null  category
 10  surface\_reelle\_bati  46149 non-null  float64
 11  longitude            46482 non-null  float64
 12  latitude             46482 non-null  float64
 13  adresse\_complete     46481 non-null  string
 14  month                46482 non-null  category
 15  day                  46482 non-null  category
 16  year                 46482 non-null  category
dtypes: category(5), datetime64[ns](1), float64(5), int64(2), string(4)
memory usage: 4.8 MB
    \end{Verbatim}

    \hypertarget{gestion-des-donnuxe9es-manquantes}{%
\subsubsection{Gestion des données
manquantes}\label{gestion-des-donnuxe9es-manquantes}}

Les méthodes numériques d'apprentissage ne gèrent pas les NaN ou null
sur les valeurs numériques

    \begin{tcolorbox}[breakable, size=fbox, boxrule=1pt, pad at break*=1mm,colback=cellbackground, colframe=cellborder]
\prompt{In}{incolor}{56}{\boxspacing}
\begin{Verbatim}[commandchars=\\\{\}]
\PY{n}{dfDep}\PY{o}{.}\PY{n}{isna}\PY{p}{(}\PY{p}{)}\PY{o}{.}\PY{n}{sum}\PY{p}{(}\PY{p}{)}
\end{Verbatim}
\end{tcolorbox}

            \begin{tcolorbox}[breakable, size=fbox, boxrule=.5pt, pad at break*=1mm, opacityfill=0]
\prompt{Out}{outcolor}{56}{\boxspacing}
\begin{Verbatim}[commandchars=\\\{\}]
id\_mutation                0
date\_mutation              0
numero\_disposition         0
valeur\_fonciere            0
adresse\_numero             1
adresse\_nom\_voie           1
nom\_commune                0
lot1\_surface\_carrez    18144
nombre\_lots                0
type\_local               323
surface\_reelle\_bati      333
longitude                  0
latitude                   0
adresse\_complete           1
month                      0
day                        0
year                       0
dtype: int64
\end{Verbatim}
\end{tcolorbox}
        
    \begin{tcolorbox}[breakable, size=fbox, boxrule=1pt, pad at break*=1mm,colback=cellbackground, colframe=cellborder]
\prompt{In}{incolor}{57}{\boxspacing}
\begin{Verbatim}[commandchars=\\\{\}]
\PY{n}{dfDep}\PY{o}{.}\PY{n}{drop}\PY{p}{(}\PY{p}{[}\PY{l+s+s1}{\PYZsq{}}\PY{l+s+s1}{lot1\PYZus{}surface\PYZus{}carrez}\PY{l+s+s1}{\PYZsq{}}\PY{p}{]}\PY{p}{,}\PY{n}{inplace}\PY{o}{=}\PY{k+kc}{True}\PY{p}{,} \PY{n}{axis}\PY{o}{=}\PY{l+m+mi}{1}\PY{p}{)}
\PY{n}{dfDep}\PY{o}{.}\PY{n}{drop}\PY{p}{(}\PY{n}{dfDep}\PY{p}{[}\PY{n}{dfDep}\PY{p}{[}\PY{l+s+s1}{\PYZsq{}}\PY{l+s+s1}{surface\PYZus{}reelle\PYZus{}bati}\PY{l+s+s1}{\PYZsq{}}\PY{p}{]}\PY{o}{.}\PY{n}{isna}\PY{p}{(}\PY{p}{)}\PY{p}{]}\PY{o}{.}\PY{n}{index}\PY{p}{,} \PY{n}{inplace}\PY{o}{=}\PY{k+kc}{True}\PY{p}{,} \PY{n}{axis}\PY{o}{=}\PY{l+m+mi}{0}\PY{p}{)}
\end{Verbatim}
\end{tcolorbox}

    \begin{tcolorbox}[breakable, size=fbox, boxrule=1pt, pad at break*=1mm,colback=cellbackground, colframe=cellborder]
\prompt{In}{incolor}{58}{\boxspacing}
\begin{Verbatim}[commandchars=\\\{\}]
\PY{n}{dfDep}\PY{o}{.}\PY{n}{isna}\PY{p}{(}\PY{p}{)}\PY{o}{.}\PY{n}{sum}\PY{p}{(}\PY{p}{)}
\end{Verbatim}
\end{tcolorbox}

            \begin{tcolorbox}[breakable, size=fbox, boxrule=.5pt, pad at break*=1mm, opacityfill=0]
\prompt{Out}{outcolor}{58}{\boxspacing}
\begin{Verbatim}[commandchars=\\\{\}]
id\_mutation            0
date\_mutation          0
numero\_disposition     0
valeur\_fonciere        0
adresse\_numero         0
adresse\_nom\_voie       0
nom\_commune            0
nombre\_lots            0
type\_local             0
surface\_reelle\_bati    0
longitude              0
latitude               0
adresse\_complete       0
month                  0
day                    0
year                   0
dtype: int64
\end{Verbatim}
\end{tcolorbox}
        
    \hypertarget{construction-des-ensembles-x-et-y-uxe0-partir-du-dataframe}{%
\subsubsection{Construction des ensembles X et y à partir du
dataframe}\label{construction-des-ensembles-x-et-y-uxe0-partir-du-dataframe}}

    On peut enlever l'id qui est un champ purement technique ainsi que la
date et tous les éléments de type string

    \begin{tcolorbox}[breakable, size=fbox, boxrule=1pt, pad at break*=1mm,colback=cellbackground, colframe=cellborder]
\prompt{In}{incolor}{59}{\boxspacing}
\begin{Verbatim}[commandchars=\\\{\}]
\PY{n}{dfDep}\PY{p}{[}\PY{l+s+s2}{\PYZdq{}}\PY{l+s+s2}{id\PYZus{}mutation}\PY{l+s+s2}{\PYZdq{}}\PY{p}{]}\PY{o}{.}\PY{n}{drop\PYZus{}duplicates}\PY{p}{(}\PY{n}{inplace}\PY{o}{=}\PY{k+kc}{True}\PY{p}{)}
\PY{n+nb}{print}\PY{p}{(}\PY{n}{dfDep}\PY{o}{.}\PY{n}{shape}\PY{p}{)}
\end{Verbatim}
\end{tcolorbox}

    \begin{Verbatim}[commandchars=\\\{\}]
(46149, 16)
    \end{Verbatim}

    \begin{tcolorbox}[breakable, size=fbox, boxrule=1pt, pad at break*=1mm,colback=cellbackground, colframe=cellborder]
\prompt{In}{incolor}{60}{\boxspacing}
\begin{Verbatim}[commandchars=\\\{\}]
\PY{n}{X} \PY{o}{=} \PY{n}{dfDep}\PY{o}{.}\PY{n}{drop}\PY{p}{(}\PY{p}{[}\PY{l+s+s2}{\PYZdq{}}\PY{l+s+s2}{valeur\PYZus{}fonciere}\PY{l+s+s2}{\PYZdq{}}\PY{p}{,}\PY{l+s+s2}{\PYZdq{}}\PY{l+s+s2}{id\PYZus{}mutation}\PY{l+s+s2}{\PYZdq{}}\PY{p}{,} \PY{l+s+s2}{\PYZdq{}}\PY{l+s+s2}{date\PYZus{}mutation}\PY{l+s+s2}{\PYZdq{}}\PY{p}{,} \PY{l+s+s2}{\PYZdq{}}\PY{l+s+s2}{numero\PYZus{}disposition}\PY{l+s+s2}{\PYZdq{}}\PY{p}{,} \PY{l+s+s2}{\PYZdq{}}\PY{l+s+s2}{adresse\PYZus{}numero}\PY{l+s+s2}{\PYZdq{}}\PY{p}{,} \PY{l+s+s2}{\PYZdq{}}\PY{l+s+s2}{adresse\PYZus{}nom\PYZus{}voie}\PY{l+s+s2}{\PYZdq{}}\PY{p}{,}\PY{l+s+s2}{\PYZdq{}}\PY{l+s+s2}{adresse\PYZus{}complete}\PY{l+s+s2}{\PYZdq{}}\PY{p}{]}\PY{p}{,} \PY{n}{axis} \PY{o}{=} \PY{l+m+mi}{1}\PY{p}{)}
\PY{n}{y} \PY{o}{=} \PY{n}{dfDep}\PY{p}{[}\PY{l+s+s2}{\PYZdq{}}\PY{l+s+s2}{valeur\PYZus{}fonciere}\PY{l+s+s2}{\PYZdq{}}\PY{p}{]}
\PY{n+nb}{print}\PY{p}{(}\PY{l+s+sa}{f}\PY{l+s+s2}{\PYZdq{}}\PY{l+s+s2}{Shape de X : }\PY{l+s+si}{\PYZob{}}\PY{n}{X}\PY{o}{.}\PY{n}{shape}\PY{l+s+si}{\PYZcb{}}\PY{l+s+s2}{\PYZdq{}}\PY{p}{)}
\PY{n+nb}{print}\PY{p}{(}\PY{l+s+sa}{f}\PY{l+s+s2}{\PYZdq{}}\PY{l+s+s2}{Shape de y : }\PY{l+s+si}{\PYZob{}}\PY{n}{y}\PY{o}{.}\PY{n}{shape}\PY{l+s+si}{\PYZcb{}}\PY{l+s+s2}{\PYZdq{}}\PY{p}{)}
\PY{n}{X}\PY{o}{.}\PY{n}{head}\PY{p}{(}\PY{l+m+mi}{5}\PY{p}{)}
\end{Verbatim}
\end{tcolorbox}

    \begin{Verbatim}[commandchars=\\\{\}]
Shape de X : (46149, 9)
Shape de y : (46149,)
    \end{Verbatim}

            \begin{tcolorbox}[breakable, size=fbox, boxrule=.5pt, pad at break*=1mm, opacityfill=0]
\prompt{Out}{outcolor}{60}{\boxspacing}
\begin{Verbatim}[commandchars=\\\{\}]
                      nom\_commune  nombre\_lots   type\_local  \textbackslash{}
1379722  Paris 18e Arrondissement            2  Appartement
1379725   Paris 3e Arrondissement            1  Appartement
1379732   Paris 9e Arrondissement            3  Appartement
1379734  Paris 10e Arrondissement            2  Appartement
1379736  Paris 20e Arrondissement            2  Appartement

         surface\_reelle\_bati  longitude   latitude month day  year
1379722                 45.0   2.348168  48.884490     1   4  2022
1379725                 42.0   2.362871  48.863374     1   6  2022
1379732                 69.0   2.332324  48.880353     1   5  2022
1379734                 33.0   2.362613  48.879658     1   5  2022
1379736                 29.0   2.405513  48.872782     1   7  2022
\end{Verbatim}
\end{tcolorbox}
        
    \hypertarget{preprocessing-sur-les-variables-catuxe9gorielles}{%
\subsubsection{Preprocessing sur les variables
catégorielles}\label{preprocessing-sur-les-variables-catuxe9gorielles}}

    \begin{tcolorbox}[breakable, size=fbox, boxrule=1pt, pad at break*=1mm,colback=cellbackground, colframe=cellborder]
\prompt{In}{incolor}{61}{\boxspacing}
\begin{Verbatim}[commandchars=\\\{\}]
\PY{n}{categorical\PYZus{}features} \PY{o}{=} \PY{n}{X}\PY{o}{.}\PY{n}{columns}\PY{p}{[}\PY{n}{X}\PY{o}{.}\PY{n}{dtypes} \PY{o}{==} \PY{l+s+s2}{\PYZdq{}}\PY{l+s+s2}{category}\PY{l+s+s2}{\PYZdq{}}\PY{p}{]}\PY{o}{.}\PY{n}{tolist}\PY{p}{(}\PY{p}{)}
\PY{n+nb}{print}\PY{p}{(}\PY{n}{categorical\PYZus{}features}\PY{p}{)}
\end{Verbatim}
\end{tcolorbox}

    \begin{Verbatim}[commandchars=\\\{\}]
['nom\_commune', 'type\_local', 'month', 'day', 'year']
    \end{Verbatim}

    Scikit-learn ne reconnait pas les objets de type DataFrame directement,
notamment les types catégoriels. Il faut donc préparer nos données afin
que les méthodes de scikit-learn puissent les interpréter. Scikit learn
requiert un encodage numérique des ces variables. Nous allons donc
devoir encoder nos variables explicatives catégorielles à l'aide de
variables indicatrices.

    \begin{tcolorbox}[breakable, size=fbox, boxrule=1pt, pad at break*=1mm,colback=cellbackground, colframe=cellborder]
\prompt{In}{incolor}{62}{\boxspacing}
\begin{Verbatim}[commandchars=\\\{\}]
\PY{n}{df\PYZus{}dummies} \PY{o}{=}  \PY{n}{pd}\PY{o}{.}\PY{n}{get\PYZus{}dummies}\PY{p}{(}\PY{n}{X}\PY{p}{[}\PY{n}{categorical\PYZus{}features}\PY{p}{]}\PY{p}{,} \PY{n}{drop\PYZus{}first}\PY{o}{=}\PY{k+kc}{True}\PY{p}{)}
\PY{n}{X} \PY{o}{=} \PY{n}{pd}\PY{o}{.}\PY{n}{concat}\PY{p}{(}\PY{p}{[}\PY{n}{X}\PY{o}{.}\PY{n}{drop}\PY{p}{(}\PY{n}{categorical\PYZus{}features}\PY{p}{,} \PY{n}{axis}\PY{o}{=}\PY{l+m+mi}{1}\PY{p}{)}\PY{p}{,} \PY{n}{df\PYZus{}dummies}\PY{p}{]}\PY{p}{,} \PY{n}{axis}\PY{o}{=}\PY{l+m+mi}{1}\PY{p}{)}
\PY{n}{X}\PY{o}{.}\PY{n}{head}\PY{p}{(}\PY{l+m+mi}{5}\PY{p}{)}
\end{Verbatim}
\end{tcolorbox}

            \begin{tcolorbox}[breakable, size=fbox, boxrule=.5pt, pad at break*=1mm, opacityfill=0]
\prompt{Out}{outcolor}{62}{\boxspacing}
\begin{Verbatim}[commandchars=\\\{\}]
         nombre\_lots  surface\_reelle\_bati  longitude   latitude  \textbackslash{}
1379722            2                 45.0   2.348168  48.884490
1379725            1                 42.0   2.362871  48.863374
1379732            3                 69.0   2.332324  48.880353
1379734            2                 33.0   2.362613  48.879658
1379736            2                 29.0   2.405513  48.872782

         nom\_commune\_Paris 11e Arrondissement  \textbackslash{}
1379722                                     0
1379725                                     0
1379732                                     0
1379734                                     0
1379736                                     0

         nom\_commune\_Paris 12e Arrondissement  \textbackslash{}
1379722                                     0
1379725                                     0
1379732                                     0
1379734                                     0
1379736                                     0

         nom\_commune\_Paris 13e Arrondissement  \textbackslash{}
1379722                                     0
1379725                                     0
1379732                                     0
1379734                                     0
1379736                                     0

         nom\_commune\_Paris 14e Arrondissement  \textbackslash{}
1379722                                     0
1379725                                     0
1379732                                     0
1379734                                     0
1379736                                     0

         nom\_commune\_Paris 15e Arrondissement  \textbackslash{}
1379722                                     0
1379725                                     0
1379732                                     0
1379734                                     0
1379736                                     0

         nom\_commune\_Paris 16e Arrondissement  {\ldots}  day\_23  day\_24  day\_25  \textbackslash{}
1379722                                     0  {\ldots}       0       0       0
1379725                                     0  {\ldots}       0       0       0
1379732                                     0  {\ldots}       0       0       0
1379734                                     0  {\ldots}       0       0       0
1379736                                     0  {\ldots}       0       0       0

         day\_26  day\_27  day\_28  day\_29  day\_30  day\_31  year\_2022
1379722       0       0       0       0       0       0          1
1379725       0       0       0       0       0       0          1
1379732       0       0       0       0       0       0          1
1379734       0       0       0       0       0       0          1
1379736       0       0       0       0       0       0          1

[5 rows x 67 columns]
\end{Verbatim}
\end{tcolorbox}
        
    \hypertarget{train-test}{%
\subsection{Train, Test}\label{train-test}}

Nous utilisons scikit-learn pour faire le traitement et étant donné la
volumétrie du jeu de données, nous allons prendre 80\% pour le train et
20\% pour le test

    \begin{tcolorbox}[breakable, size=fbox, boxrule=1pt, pad at break*=1mm,colback=cellbackground, colframe=cellborder]
\prompt{In}{incolor}{63}{\boxspacing}
\begin{Verbatim}[commandchars=\\\{\}]
\PY{k+kn}{from} \PY{n+nn}{sklearn}\PY{n+nn}{.}\PY{n+nn}{model\PYZus{}selection} \PY{k+kn}{import} \PY{n}{train\PYZus{}test\PYZus{}split}
\PY{n}{X\PYZus{}train}\PY{p}{,} \PY{n}{X\PYZus{}test}\PY{p}{,} \PY{n}{y\PYZus{}train}\PY{p}{,} \PY{n}{y\PYZus{}test} \PY{o}{=} \PY{n}{train\PYZus{}test\PYZus{}split}\PY{p}{(}\PY{n}{X}\PY{p}{,} \PY{n}{y}\PY{p}{,} \PY{n}{test\PYZus{}size}\PY{o}{=}\PY{l+m+mf}{0.2}\PY{p}{,} \PY{n}{random\PYZus{}state}\PY{o}{=}\PY{l+m+mi}{777}\PY{p}{)}
\PY{n+nb}{print}\PY{p}{(}\PY{l+s+sa}{f}\PY{l+s+s2}{\PYZdq{}}\PY{l+s+s2}{Shape du X\PYZus{}train : }\PY{l+s+si}{\PYZob{}}\PY{n}{X\PYZus{}train}\PY{o}{.}\PY{n}{shape}\PY{l+s+si}{\PYZcb{}}\PY{l+s+s2}{\PYZdq{}}\PY{p}{)}
\PY{n+nb}{print}\PY{p}{(}\PY{l+s+sa}{f}\PY{l+s+s2}{\PYZdq{}}\PY{l+s+s2}{Shape du y\PYZus{}train : }\PY{l+s+si}{\PYZob{}}\PY{n}{y\PYZus{}train}\PY{o}{.}\PY{n}{shape}\PY{l+s+si}{\PYZcb{}}\PY{l+s+s2}{\PYZdq{}}\PY{p}{)}
\PY{n+nb}{print}\PY{p}{(}\PY{l+s+sa}{f}\PY{l+s+s2}{\PYZdq{}}\PY{l+s+s2}{Shape du X\PYZus{}test : }\PY{l+s+si}{\PYZob{}}\PY{n}{X\PYZus{}test}\PY{o}{.}\PY{n}{shape}\PY{l+s+si}{\PYZcb{}}\PY{l+s+s2}{\PYZdq{}}\PY{p}{)}
\PY{n+nb}{print}\PY{p}{(}\PY{l+s+sa}{f}\PY{l+s+s2}{\PYZdq{}}\PY{l+s+s2}{Shape du y\PYZus{}test : }\PY{l+s+si}{\PYZob{}}\PY{n}{y\PYZus{}test}\PY{o}{.}\PY{n}{shape}\PY{l+s+si}{\PYZcb{}}\PY{l+s+s2}{\PYZdq{}}\PY{p}{)}
\end{Verbatim}
\end{tcolorbox}

    \begin{Verbatim}[commandchars=\\\{\}]
Shape du X\_train : (36919, 67)
Shape du y\_train : (36919,)
Shape du X\_test : (9230, 67)
Shape du y\_test : (9230,)
    \end{Verbatim}

    \hypertarget{preprocessing-sur-les-variables-numuxe9riques}{%
\subsection{Preprocessing sur les variables
numériques}\label{preprocessing-sur-les-variables-numuxe9riques}}

    \begin{tcolorbox}[breakable, size=fbox, boxrule=1pt, pad at break*=1mm,colback=cellbackground, colframe=cellborder]
\prompt{In}{incolor}{64}{\boxspacing}
\begin{Verbatim}[commandchars=\\\{\}]
\PY{n}{numerical\PYZus{}features} \PY{o}{=} \PY{n}{dfDep}\PY{o}{.}\PY{n}{columns}\PY{p}{[}\PY{p}{(}\PY{n}{dfDep}\PY{o}{.}\PY{n}{dtypes} \PY{o}{==} \PY{l+s+s2}{\PYZdq{}}\PY{l+s+s2}{int64}\PY{l+s+s2}{\PYZdq{}}\PY{p}{)}\PY{p}{]}\PY{o}{.}\PY{n}{tolist}\PY{p}{(}\PY{p}{)} \PY{o}{+} \PY{n}{dfDep}\PY{o}{.}\PY{n}{columns}\PY{p}{[}\PY{p}{(}\PY{n}{dfDep}\PY{o}{.}\PY{n}{dtypes} \PY{o}{==} \PY{l+s+s2}{\PYZdq{}}\PY{l+s+s2}{float64}\PY{l+s+s2}{\PYZdq{}}\PY{p}{)}\PY{p}{]}\PY{o}{.}\PY{n}{tolist}\PY{p}{(}\PY{p}{)}
\PY{n+nb}{print}\PY{p}{(}\PY{n}{numerical\PYZus{}features}\PY{p}{)}
\end{Verbatim}
\end{tcolorbox}

    \begin{Verbatim}[commandchars=\\\{\}]
['numero\_disposition', 'nombre\_lots', 'valeur\_fonciere', 'surface\_reelle\_bati',
'longitude', 'latitude']
    \end{Verbatim}

    Certaines méthodes d'apprentissage sont sensibles aux problèmes
d'échelle sur les valeurs numériques. En preprocessing, on standardise
les variables numériques en retranchant leur moyenne et en divisant par
l'écart type via Scikit-learn. On réalise ce traitement sur l'ensemble
d'apprentissage et on applique cette standardisation sur l'ensemble de
test.

    \begin{tcolorbox}[breakable, size=fbox, boxrule=1pt, pad at break*=1mm,colback=cellbackground, colframe=cellborder]
\prompt{In}{incolor}{65}{\boxspacing}
\begin{Verbatim}[commandchars=\\\{\}]
\PY{k+kn}{from} \PY{n+nn}{sklearn}\PY{n+nn}{.}\PY{n+nn}{preprocessing} \PY{k+kn}{import} \PY{n}{StandardScaler}
\PY{n}{scaler} \PY{o}{=} \PY{n}{StandardScaler}\PY{p}{(}\PY{p}{)}

\PY{n}{X\PYZus{}train\PYZus{}scaled} \PY{o}{=} \PY{n}{scaler}\PY{o}{.}\PY{n}{fit\PYZus{}transform}\PY{p}{(}\PY{n}{X\PYZus{}train}\PY{p}{)}
\PY{n}{X\PYZus{}test\PYZus{}scaled} \PY{o}{=} \PY{n}{scaler}\PY{o}{.}\PY{n}{transform}\PY{p}{(}\PY{n}{X\PYZus{}test}\PY{p}{)}
\end{Verbatim}
\end{tcolorbox}

    \hypertarget{un-moduxe8le-simple-la-ruxe9gression-linuxe9aire}{%
\subsection{Un modèle simple : la régression
linéaire}\label{un-moduxe8le-simple-la-ruxe9gression-linuxe9aire}}

Un premier modèle qui nous servira de \emph{baseline}.

Nous allons aussi introduire l'instanciation sur les données
\emph{train}, et nous validerons \textbf{ENSUITE} sur les données
\emph{test}.

\hypertarget{moduxe8le-de-regression-sur-traintest}{%
\subsubsection{Modèle de regression sur
Train/Test}\label{moduxe8le-de-regression-sur-traintest}}

\[y =\sum_{i=1}^{n} a_i \times x_i + b\]

    \begin{tcolorbox}[breakable, size=fbox, boxrule=1pt, pad at break*=1mm,colback=cellbackground, colframe=cellborder]
\prompt{In}{incolor}{66}{\boxspacing}
\begin{Verbatim}[commandchars=\\\{\}]
\PY{k+kn}{from} \PY{n+nn}{sklearn} \PY{k+kn}{import} \PY{n}{linear\PYZus{}model}
\PY{n}{reg} \PY{o}{=} \PY{n}{linear\PYZus{}model}\PY{o}{.}\PY{n}{LinearRegression}\PY{p}{(}\PY{p}{)}

\PY{n}{reg}\PY{o}{.}\PY{n}{fit}\PY{p}{(}\PY{n}{X\PYZus{}train\PYZus{}scaled}\PY{p}{,} \PY{n}{y\PYZus{}train}\PY{p}{)}
\PY{n}{y\PYZus{}trainPred} \PY{o}{=} \PY{n}{reg}\PY{o}{.}\PY{n}{predict}\PY{p}{(}\PY{n}{X\PYZus{}train\PYZus{}scaled}\PY{p}{)}
\PY{n}{y\PYZus{}testPred} \PY{o}{=} \PY{n}{reg}\PY{o}{.}\PY{n}{predict}\PY{p}{(}\PY{n}{X\PYZus{}test\PYZus{}scaled}\PY{p}{)}
\PY{n+nb}{print}\PY{p}{(}\PY{l+s+sa}{f}\PY{l+s+s2}{\PYZdq{}}\PY{l+s+s2}{Score sur le train : }\PY{l+s+si}{\PYZob{}}\PY{n}{reg}\PY{o}{.}\PY{n}{score}\PY{p}{(}\PY{n}{X\PYZus{}train\PYZus{}scaled}\PY{p}{,}\PY{n}{y\PYZus{}train}\PY{p}{)}\PY{l+s+si}{\PYZcb{}}\PY{l+s+s2}{\PYZdq{}}\PY{p}{)}
\PY{n+nb}{print}\PY{p}{(}\PY{l+s+sa}{f}\PY{l+s+s2}{\PYZdq{}}\PY{l+s+s2}{Score sur le test : }\PY{l+s+si}{\PYZob{}}\PY{n}{reg}\PY{o}{.}\PY{n}{score}\PY{p}{(}\PY{n}{X\PYZus{}test\PYZus{}scaled}\PY{p}{,}\PY{n}{y\PYZus{}test}\PY{p}{)}\PY{l+s+si}{\PYZcb{}}\PY{l+s+s2}{\PYZdq{}}\PY{p}{)}
\end{Verbatim}
\end{tcolorbox}

    \begin{Verbatim}[commandchars=\\\{\}]
Score sur le train : 0.8017175728977979
Score sur le test : 0.8011631121444063
    \end{Verbatim}

    La régression linéaire donne des résultats et il n'y a pas de phénomène
de sur-apprentissage.

    \hypertarget{coefficients-de-la-ruxe9gression-linuxe9aire}{%
\subsection{Coefficients de la régression
linéaire}\label{coefficients-de-la-ruxe9gression-linuxe9aire}}

Un des avantages de la régression linéaire est que nous pouvons obtenir
les coefficients associés à chacune des variables. Nous pouvons voir les
coefficients qui ont un impact sur le nombre de vélos loués.

Regardons ces coefficients :

    \begin{tcolorbox}[breakable, size=fbox, boxrule=1pt, pad at break*=1mm,colback=cellbackground, colframe=cellborder]
\prompt{In}{incolor}{67}{\boxspacing}
\begin{Verbatim}[commandchars=\\\{\}]
\PY{n}{coefficients} \PY{o}{=} \PY{n}{pd}\PY{o}{.}\PY{n}{Series}\PY{p}{(}\PY{n}{reg}\PY{o}{.}\PY{n}{coef\PYZus{}}\PY{o}{.}\PY{n}{flatten}\PY{p}{(}\PY{p}{)}\PY{p}{,} \PY{n}{index}\PY{o}{=}\PY{n}{X}\PY{o}{.}\PY{n}{columns}\PY{p}{)}\PY{o}{.}\PY{n}{sort\PYZus{}values}\PY{p}{(}\PY{n}{ascending}\PY{o}{=}\PY{k+kc}{False}\PY{p}{)}
\PY{n}{coefficients}
\end{Verbatim}
\end{tcolorbox}

            \begin{tcolorbox}[breakable, size=fbox, boxrule=.5pt, pad at break*=1mm, opacityfill=0]
\prompt{Out}{outcolor}{67}{\boxspacing}
\begin{Verbatim}[commandchars=\\\{\}]
surface\_reelle\_bati                                    249572.870033
nom\_commune\_Paris 6e Arrondissement                     16466.302201
latitude                                                14105.244694
nom\_commune\_Paris 5e Arrondissement                     13676.283050
nom\_commune\_Paris 7e Arrondissement                     13343.601169
                                                           {\ldots}
nom\_commune\_Paris 15e Arrondissement                    -5217.255155
nom\_commune\_Paris 19e Arrondissement                   -11347.416919
nom\_commune\_Paris 18e Arrondissement                   -12725.454998
longitude                                              -26494.427558
type\_local\_Local industriel. commercial ou assimilé    -27947.820593
Length: 67, dtype: float64
\end{Verbatim}
\end{tcolorbox}
        
    \begin{tcolorbox}[breakable, size=fbox, boxrule=1pt, pad at break*=1mm,colback=cellbackground, colframe=cellborder]
\prompt{In}{incolor}{68}{\boxspacing}
\begin{Verbatim}[commandchars=\\\{\}]
\PY{n+nb}{print}\PY{p}{(}\PY{l+s+sa}{f}\PY{l+s+s2}{\PYZdq{}}\PY{l+s+s2}{ordonnee à l}\PY{l+s+s2}{\PYZsq{}}\PY{l+s+s2}{origine : }\PY{l+s+si}{\PYZob{}}\PY{n}{reg}\PY{o}{.}\PY{n}{intercept\PYZus{}}\PY{l+s+si}{\PYZcb{}}\PY{l+s+s2}{\PYZdq{}}\PY{p}{)}
\end{Verbatim}
\end{tcolorbox}

    \begin{Verbatim}[commandchars=\\\{\}]
ordonnee à l'origine : 499534.7393496575
    \end{Verbatim}

    \begin{tcolorbox}[breakable, size=fbox, boxrule=1pt, pad at break*=1mm,colback=cellbackground, colframe=cellborder]
\prompt{In}{incolor}{69}{\boxspacing}
\begin{Verbatim}[commandchars=\\\{\}]
\PY{n}{coefficients}\PY{p}{[}\PY{n}{np}\PY{o}{.}\PY{n}{abs}\PY{p}{(}\PY{n}{coefficients}\PY{p}{)}\PY{o}{\PYZgt{}}\PY{l+m+mi}{10000}\PY{p}{]}\PY{o}{.}\PY{n}{plot}\PY{p}{(}\PY{n}{kind}\PY{o}{=}\PY{l+s+s2}{\PYZdq{}}\PY{l+s+s2}{bar}\PY{l+s+s2}{\PYZdq{}}\PY{p}{)}
\PY{n}{plt}\PY{o}{.}\PY{n}{title}\PY{p}{(}\PY{l+s+s2}{\PYZdq{}}\PY{l+s+s2}{Regression lineaire coefficient}\PY{l+s+s2}{\PYZdq{}}\PY{p}{)}
\PY{n}{plt}\PY{o}{.}\PY{n}{ylabel}\PY{p}{(}\PY{l+s+s2}{\PYZdq{}}\PY{l+s+s2}{Coefficient value}\PY{l+s+s2}{\PYZdq{}}\PY{p}{)}
\PY{n}{plt}\PY{o}{.}\PY{n}{show}\PY{p}{(}\PY{p}{)}
\end{Verbatim}
\end{tcolorbox}

    \begin{center}
    \adjustimage{max size={0.9\linewidth}{0.9\paperheight}}{output_115_0.png}
    \end{center}
    { \hspace*{\fill} \\}
    
    On retrouve des éléments de l'exploration. Certaines communes tirent le
prix vers le base comme le 18ème, 19ème au contraire du 6ème, \ldots{}
L'élément le plus prépondérant est la surface réellement bati. La
lattitude et la longitude s'opposent en termes d'effet.

    \hypertarget{evaluation-de-la-ruxe9gression-avec-diffuxe9rentes-muxe9triques}{%
\subsubsection{Evaluation de la régression avec différentes
métriques}\label{evaluation-de-la-ruxe9gression-avec-diffuxe9rentes-muxe9triques}}

Nous allons regarder quelques métriques associées aux problématiques de
régression : * L'erreur maximum entre la prédiction et la réalité * La
moyenne des erreurs absolus entre la prédiction et la réalité * La
moyenne des erreurs au carré entre la prédiction et la réalité (MSE) *
Le score R2 qui est le coefficient de détermination en comparant MSE et
la variance. Fonction renvoyée par la méthode score de Scikit Learn

    \begin{tcolorbox}[breakable, size=fbox, boxrule=1pt, pad at break*=1mm,colback=cellbackground, colframe=cellborder]
\prompt{In}{incolor}{70}{\boxspacing}
\begin{Verbatim}[commandchars=\\\{\}]
\PY{k+kn}{from} \PY{n+nn}{sklearn} \PY{k+kn}{import} \PY{n}{metrics}


\PY{k}{def} \PY{n+nf}{regression\PYZus{}metrics}\PY{p}{(}\PY{n}{y}\PY{p}{,} \PY{n}{y\PYZus{}pred}\PY{p}{)}\PY{p}{:}
    \PY{k}{return} \PY{n}{pd}\PY{o}{.}\PY{n}{DataFrame}\PY{p}{(}
        \PY{p}{\PYZob{}}
            \PY{l+s+s2}{\PYZdq{}}\PY{l+s+s2}{max\PYZus{}error}\PY{l+s+s2}{\PYZdq{}}\PY{p}{:} \PY{n}{metrics}\PY{o}{.}\PY{n}{max\PYZus{}error}\PY{p}{(}\PY{n}{y\PYZus{}true}\PY{o}{=}\PY{n}{y}\PY{p}{,} \PY{n}{y\PYZus{}pred}\PY{o}{=}\PY{n}{y\PYZus{}pred}\PY{p}{)}\PY{p}{,}
            \PY{l+s+s2}{\PYZdq{}}\PY{l+s+s2}{mean\PYZus{}absolute\PYZus{}error}\PY{l+s+s2}{\PYZdq{}}\PY{p}{:} \PY{n}{metrics}\PY{o}{.}\PY{n}{mean\PYZus{}absolute\PYZus{}error}\PY{p}{(}\PY{n}{y\PYZus{}true}\PY{o}{=}\PY{n}{y}\PY{p}{,} \PY{n}{y\PYZus{}pred}\PY{o}{=}\PY{n}{y\PYZus{}pred}\PY{p}{)}\PY{p}{,}
            \PY{l+s+s2}{\PYZdq{}}\PY{l+s+s2}{mean\PYZus{}squared\PYZus{}error}\PY{l+s+s2}{\PYZdq{}}\PY{p}{:} \PY{n}{metrics}\PY{o}{.}\PY{n}{mean\PYZus{}squared\PYZus{}error}\PY{p}{(}\PY{n}{y\PYZus{}true}\PY{o}{=}\PY{n}{y}\PY{p}{,} \PY{n}{y\PYZus{}pred}\PY{o}{=}\PY{n}{y\PYZus{}pred}\PY{p}{)}\PY{p}{,}
            \PY{l+s+s2}{\PYZdq{}}\PY{l+s+s2}{r2\PYZus{}score}\PY{l+s+s2}{\PYZdq{}}\PY{p}{:} \PY{n}{metrics}\PY{o}{.}\PY{n}{r2\PYZus{}score}\PY{p}{(}\PY{n}{y\PYZus{}true}\PY{o}{=}\PY{n}{y}\PY{p}{,} \PY{n}{y\PYZus{}pred}\PY{o}{=}\PY{n}{y\PYZus{}pred}\PY{p}{)}
        \PY{p}{\PYZcb{}}\PY{p}{,}
        \PY{n}{index}\PY{o}{=}\PY{p}{[}\PY{l+m+mi}{0}\PY{p}{]}\PY{p}{)}
\end{Verbatim}
\end{tcolorbox}

    \begin{tcolorbox}[breakable, size=fbox, boxrule=1pt, pad at break*=1mm,colback=cellbackground, colframe=cellborder]
\prompt{In}{incolor}{71}{\boxspacing}
\begin{Verbatim}[commandchars=\\\{\}]
\PY{n+nb}{print}\PY{p}{(}\PY{l+s+s2}{\PYZdq{}}\PY{l+s+s2}{Regression metrics for train data}\PY{l+s+s2}{\PYZdq{}}\PY{p}{)}
\PY{n+nb}{print}\PY{p}{(}\PY{n}{regression\PYZus{}metrics}\PY{p}{(}\PY{n}{y\PYZus{}train}\PY{p}{,} \PY{n}{y\PYZus{}trainPred}\PY{p}{)}\PY{p}{)}
\PY{n+nb}{print}\PY{p}{(}\PY{l+s+s2}{\PYZdq{}}\PY{l+s+s2}{Regression metrics for test data}\PY{l+s+s2}{\PYZdq{}}\PY{p}{)}
\PY{n+nb}{print}\PY{p}{(}\PY{n}{regression\PYZus{}metrics}\PY{p}{(}\PY{n}{y\PYZus{}test}\PY{p}{,} \PY{n}{y\PYZus{}testPred}\PY{p}{)}\PY{p}{)}
\end{Verbatim}
\end{tcolorbox}

    \begin{Verbatim}[commandchars=\\\{\}]
Regression metrics for train data
      max\_error  mean\_absolute\_error  mean\_squared\_error  r2\_score
0  1.596970e+06         84687.636312        1.653548e+10  0.801718
Regression metrics for test data
      max\_error  mean\_absolute\_error  mean\_squared\_error  r2\_score
0  1.321491e+06         85090.635792        1.620885e+10  0.801163
    \end{Verbatim}

    Le modèle de regression linéaire n'est pas très bon quelque soit la
métrique retenue.

    \hypertarget{arbre-de-duxe9cision-et-visions-ensemblistes}{%
\subsection{Arbre de décision et visions
ensemblistes}\label{arbre-de-duxe9cision-et-visions-ensemblistes}}

\hypertarget{arbre-de-duxe9cision}{%
\subsubsection{Arbre de décision}\label{arbre-de-duxe9cision}}

    \begin{tcolorbox}[breakable, size=fbox, boxrule=1pt, pad at break*=1mm,colback=cellbackground, colframe=cellborder]
\prompt{In}{incolor}{72}{\boxspacing}
\begin{Verbatim}[commandchars=\\\{\}]
\PY{k+kn}{from} \PY{n+nn}{sklearn}\PY{n+nn}{.}\PY{n+nn}{tree} \PY{k+kn}{import} \PY{n}{DecisionTreeRegressor}
\PY{n}{decisionTree} \PY{o}{=} \PY{n}{DecisionTreeRegressor}\PY{p}{(}\PY{p}{)}
\PY{n}{decisionTree}\PY{o}{.}\PY{n}{fit}\PY{p}{(}\PY{n}{X\PYZus{}train\PYZus{}scaled}\PY{p}{,} \PY{n}{y\PYZus{}train}\PY{p}{)}
\PY{n}{y\PYZus{}trainPred} \PY{o}{=} \PY{n}{decisionTree}\PY{o}{.}\PY{n}{predict}\PY{p}{(}\PY{n}{X\PYZus{}train\PYZus{}scaled}\PY{p}{)}
\PY{n}{y\PYZus{}testPred} \PY{o}{=} \PY{n}{decisionTree}\PY{o}{.}\PY{n}{predict}\PY{p}{(}\PY{n}{X\PYZus{}test\PYZus{}scaled}\PY{p}{)}
\PY{n+nb}{print}\PY{p}{(}\PY{l+s+sa}{f}\PY{l+s+s2}{\PYZdq{}}\PY{l+s+s2}{Score sur le train de l}\PY{l+s+s2}{\PYZsq{}}\PY{l+s+s2}{arbre de décision : }\PY{l+s+si}{\PYZob{}}\PY{n}{decisionTree}\PY{o}{.}\PY{n}{score}\PY{p}{(}\PY{n}{X\PYZus{}train\PYZus{}scaled}\PY{p}{,}\PY{n}{y\PYZus{}train}\PY{p}{)}\PY{l+s+si}{\PYZcb{}}\PY{l+s+s2}{\PYZdq{}}\PY{p}{)}
\PY{n+nb}{print}\PY{p}{(}\PY{l+s+sa}{f}\PY{l+s+s2}{\PYZdq{}}\PY{l+s+s2}{Score sur le test de l}\PY{l+s+s2}{\PYZsq{}}\PY{l+s+s2}{arbre de décision : }\PY{l+s+si}{\PYZob{}}\PY{n}{decisionTree}\PY{o}{.}\PY{n}{score}\PY{p}{(}\PY{n}{X\PYZus{}test\PYZus{}scaled}\PY{p}{,}\PY{n}{y\PYZus{}test}\PY{p}{)}\PY{l+s+si}{\PYZcb{}}\PY{l+s+s2}{\PYZdq{}}\PY{p}{)}
\end{Verbatim}
\end{tcolorbox}

    \begin{Verbatim}[commandchars=\\\{\}]
Score sur le train de l'arbre de décision : 0.9998805881649797
Score sur le test de l'arbre de décision : 0.7432610064158165
    \end{Verbatim}

    \begin{tcolorbox}[breakable, size=fbox, boxrule=1pt, pad at break*=1mm,colback=cellbackground, colframe=cellborder]
\prompt{In}{incolor}{73}{\boxspacing}
\begin{Verbatim}[commandchars=\\\{\}]
\PY{n+nb}{print}\PY{p}{(}\PY{l+s+s2}{\PYZdq{}}\PY{l+s+s2}{Regression metrics with Decision Tree for train data}\PY{l+s+s2}{\PYZdq{}}\PY{p}{)}
\PY{n+nb}{print}\PY{p}{(}\PY{n}{regression\PYZus{}metrics}\PY{p}{(}\PY{n}{y\PYZus{}train}\PY{p}{,} \PY{n}{y\PYZus{}trainPred}\PY{p}{)}\PY{p}{)}
\PY{n+nb}{print}\PY{p}{(}\PY{l+s+s2}{\PYZdq{}}\PY{l+s+s2}{Regression metrics with Decision Tree for test data}\PY{l+s+s2}{\PYZdq{}}\PY{p}{)}
\PY{n+nb}{print}\PY{p}{(}\PY{n}{regression\PYZus{}metrics}\PY{p}{(}\PY{n}{y\PYZus{}test}\PY{p}{,} \PY{n}{y\PYZus{}testPred}\PY{p}{)}\PY{p}{)}
\end{Verbatim}
\end{tcolorbox}

    \begin{Verbatim}[commandchars=\\\{\}]
Regression metrics with Decision Tree for train data
   max\_error  mean\_absolute\_error  mean\_squared\_error  r2\_score
0   300000.0            64.969443        9.958180e+06  0.999881
Regression metrics with Decision Tree for test data
   max\_error  mean\_absolute\_error  mean\_squared\_error  r2\_score
0   946000.0         97475.687469        2.092894e+10  0.743261
    \end{Verbatim}

    On est dans un cas de surapprentissage puisque l'arbre de décision
``fit'' à l'ensemble de train mais ne se généralise pas bien sur
l'ensemble de test. Néanmoins la performance est moins bonne que la
régression linéaire

    \begin{tcolorbox}[breakable, size=fbox, boxrule=1pt, pad at break*=1mm,colback=cellbackground, colframe=cellborder]
\prompt{In}{incolor}{74}{\boxspacing}
\begin{Verbatim}[commandchars=\\\{\}]
\PY{n+nb}{print}\PY{p}{(}\PY{l+s+s2}{\PYZdq{}}\PY{l+s+s2}{Feature importances : }\PY{l+s+se}{\PYZbs{}n}\PY{l+s+si}{\PYZob{}\PYZcb{}}\PY{l+s+s2}{\PYZdq{}}\PY{o}{.}\PY{n}{format}\PY{p}{(}\PY{n}{decisionTree}\PY{o}{.}\PY{n}{feature\PYZus{}importances\PYZus{}}\PY{p}{)}\PY{p}{)}
\end{Verbatim}
\end{tcolorbox}

    \begin{Verbatim}[commandchars=\\\{\}]
Feature importances :
[7.24203099e-03 8.29786055e-01 4.18925665e-02 4.66348159e-02
 6.67695212e-04 1.18150928e-04 7.76163497e-04 4.08262587e-04
 7.84889388e-04 7.01459265e-04 3.81124735e-04 2.63807564e-04
 7.69301917e-05 4.66695105e-04 8.91297788e-04 2.30626244e-04
 2.90281190e-04 4.79320086e-04 8.02850151e-04 1.13558807e-03
 8.98383096e-04 5.95663078e-04 3.73737532e-04 6.87450761e-03
 8.14283687e-04 1.81425646e-03 1.50775843e-03 1.97679818e-03
 1.64040197e-03 2.27642287e-03 1.86109849e-03 7.31854835e-04
 1.45719587e-03 1.88363398e-03 1.12424481e-03 1.54502644e-03
 1.16679892e-03 9.22287403e-04 8.28115801e-04 1.32674090e-03
 1.16443121e-03 1.12699776e-03 1.36499948e-03 1.37794852e-03
 1.42391302e-03 1.10168192e-03 8.97735692e-04 1.07600888e-03
 1.22417400e-03 1.40406454e-03 1.57954947e-03 1.38401471e-03
 1.04567914e-03 1.10180426e-03 1.35720554e-03 9.28337909e-04
 1.38411118e-03 1.30660494e-03 1.13217292e-03 9.89857433e-04
 1.01727895e-03 1.20337627e-03 1.38172305e-03 1.43790008e-03
 8.86376639e-04 1.46483530e-03 2.58739624e-03]
    \end{Verbatim}

    \begin{tcolorbox}[breakable, size=fbox, boxrule=1pt, pad at break*=1mm,colback=cellbackground, colframe=cellborder]
\prompt{In}{incolor}{75}{\boxspacing}
\begin{Verbatim}[commandchars=\\\{\}]
\PY{k}{def} \PY{n+nf}{plot\PYZus{}feature\PYZus{}importances}\PY{p}{(}\PY{n}{model}\PY{p}{)}\PY{p}{:}
    \PY{n}{n\PYZus{}features} \PY{o}{=} \PY{n}{X}\PY{o}{.}\PY{n}{shape}\PY{p}{[}\PY{l+m+mi}{1}\PY{p}{]}
    \PY{n}{plt}\PY{o}{.}\PY{n}{barh}\PY{p}{(}\PY{n+nb}{range}\PY{p}{(}\PY{n}{n\PYZus{}features}\PY{p}{)}\PY{p}{,} \PY{n}{model}\PY{o}{.}\PY{n}{feature\PYZus{}importances\PYZus{}}\PY{p}{,} \PY{n}{align} \PY{o}{=} \PY{l+s+s1}{\PYZsq{}}\PY{l+s+s1}{center}\PY{l+s+s1}{\PYZsq{}}\PY{p}{)}
    \PY{n}{plt}\PY{o}{.}\PY{n}{yticks}\PY{p}{(}\PY{n}{np}\PY{o}{.}\PY{n}{arange}\PY{p}{(}\PY{n}{n\PYZus{}features}\PY{p}{)}\PY{p}{,} \PY{n}{X}\PY{o}{.}\PY{n}{columns}\PY{p}{)}
    \PY{n}{plt}\PY{o}{.}\PY{n}{xlabel}\PY{p}{(}\PY{l+s+s2}{\PYZdq{}}\PY{l+s+s2}{Feature Importance}\PY{l+s+s2}{\PYZdq{}}\PY{p}{)}
    \PY{n}{plt}\PY{o}{.}\PY{n}{ylabel}\PY{p}{(}\PY{l+s+s2}{\PYZdq{}}\PY{l+s+s2}{Feature}\PY{l+s+s2}{\PYZdq{}}\PY{p}{)}
    \PY{n}{plt}\PY{o}{.}\PY{n}{ylim}\PY{p}{(}\PY{o}{\PYZhy{}}\PY{l+m+mi}{1}\PY{p}{,}\PY{n}{n\PYZus{}features}\PY{p}{)}
\end{Verbatim}
\end{tcolorbox}

    \begin{tcolorbox}[breakable, size=fbox, boxrule=1pt, pad at break*=1mm,colback=cellbackground, colframe=cellborder]
\prompt{In}{incolor}{76}{\boxspacing}
\begin{Verbatim}[commandchars=\\\{\}]
\PY{n}{plot\PYZus{}feature\PYZus{}importances}\PY{p}{(}\PY{n}{decisionTree}\PY{p}{)}
\end{Verbatim}
\end{tcolorbox}

    \begin{center}
    \adjustimage{max size={0.9\linewidth}{0.9\paperheight}}{output_127_0.png}
    \end{center}
    { \hspace*{\fill} \\}
    
    \begin{tcolorbox}[breakable, size=fbox, boxrule=1pt, pad at break*=1mm,colback=cellbackground, colframe=cellborder]
\prompt{In}{incolor}{77}{\boxspacing}
\begin{Verbatim}[commandchars=\\\{\}]
\PY{n}{featuresImportance} \PY{o}{=} \PY{n}{pd}\PY{o}{.}\PY{n}{Series}\PY{p}{(}\PY{n}{decisionTree}\PY{o}{.}\PY{n}{feature\PYZus{}importances\PYZus{}}\PY{o}{.}\PY{n}{flatten}\PY{p}{(}\PY{p}{)}\PY{p}{,} \PY{n}{index}\PY{o}{=}\PY{n}{X}\PY{o}{.}\PY{n}{columns}\PY{p}{)}\PY{o}{.}\PY{n}{sort\PYZus{}values}\PY{p}{(}\PY{n}{ascending}\PY{o}{=}\PY{k+kc}{False}\PY{p}{)}
\PY{n}{featuresImportance}\PY{p}{[}\PY{p}{(}\PY{n}{featuresImportance}\PY{p}{)}\PY{o}{\PYZgt{}}\PY{l+m+mf}{0.03}\PY{p}{]}\PY{o}{.}\PY{n}{plot}\PY{p}{(}\PY{n}{kind}\PY{o}{=}\PY{l+s+s2}{\PYZdq{}}\PY{l+s+s2}{bar}\PY{l+s+s2}{\PYZdq{}}\PY{p}{)}
\PY{n}{plt}\PY{o}{.}\PY{n}{title}\PY{p}{(}\PY{l+s+s2}{\PYZdq{}}\PY{l+s+s2}{Feature}\PY{l+s+s2}{\PYZdq{}}\PY{p}{)}
\PY{n}{plt}\PY{o}{.}\PY{n}{ylabel}\PY{p}{(}\PY{l+s+s2}{\PYZdq{}}\PY{l+s+s2}{Feature Importance}\PY{l+s+s2}{\PYZdq{}}\PY{p}{)}
\PY{n}{plt}\PY{o}{.}\PY{n}{show}\PY{p}{(}\PY{p}{)}
\end{Verbatim}
\end{tcolorbox}

    \begin{center}
    \adjustimage{max size={0.9\linewidth}{0.9\paperheight}}{output_128_0.png}
    \end{center}
    { \hspace*{\fill} \\}
    
    On retrouve la surface réelle et la position géographique du bien.

    \begin{tcolorbox}[breakable, size=fbox, boxrule=1pt, pad at break*=1mm,colback=cellbackground, colframe=cellborder]
\prompt{In}{incolor}{78}{\boxspacing}
\begin{Verbatim}[commandchars=\\\{\}]
\PY{k}{for} \PY{n}{depth} \PY{o+ow}{in} \PY{n+nb}{range}\PY{p}{(}\PY{l+m+mi}{5}\PY{p}{,}\PY{l+m+mi}{20}\PY{p}{)}\PY{p}{:}
    \PY{n}{decisionTreeMaxDepth} \PY{o}{=} \PY{n}{DecisionTreeRegressor}\PY{p}{(}\PY{n}{max\PYZus{}depth}\PY{o}{=}\PY{n}{depth}\PY{p}{)}
    \PY{n}{decisionTreeMaxDepth}\PY{o}{.}\PY{n}{fit}\PY{p}{(}\PY{n}{X\PYZus{}train\PYZus{}scaled}\PY{p}{,} \PY{n}{y\PYZus{}train}\PY{p}{)}
    \PY{n+nb}{print}\PY{p}{(}\PY{l+s+sa}{f}\PY{l+s+s2}{\PYZdq{}}\PY{l+s+s2}{Max depth : }\PY{l+s+si}{\PYZob{}}\PY{n}{depth}\PY{l+s+si}{\PYZcb{}}\PY{l+s+s2}{\PYZdq{}}\PY{p}{)}
    \PY{n+nb}{print}\PY{p}{(}\PY{l+s+sa}{f}\PY{l+s+s2}{\PYZdq{}}\PY{l+s+s2}{Score sur le train de l}\PY{l+s+s2}{\PYZsq{}}\PY{l+s+s2}{arbre de décision : }\PY{l+s+si}{\PYZob{}}\PY{n}{decisionTreeMaxDepth}\PY{o}{.}\PY{n}{score}\PY{p}{(}\PY{n}{X\PYZus{}train\PYZus{}scaled}\PY{p}{,}\PY{n}{y\PYZus{}train}\PY{p}{)}\PY{l+s+si}{\PYZcb{}}\PY{l+s+s2}{\PYZdq{}}\PY{p}{)}
    \PY{n+nb}{print}\PY{p}{(}\PY{l+s+sa}{f}\PY{l+s+s2}{\PYZdq{}}\PY{l+s+s2}{Score sur le test de l}\PY{l+s+s2}{\PYZsq{}}\PY{l+s+s2}{arbre de décision : }\PY{l+s+si}{\PYZob{}}\PY{n}{decisionTreeMaxDepth}\PY{o}{.}\PY{n}{score}\PY{p}{(}\PY{n}{X\PYZus{}test\PYZus{}scaled}\PY{p}{,}\PY{n}{y\PYZus{}test}\PY{p}{)}\PY{l+s+si}{\PYZcb{}}\PY{l+s+s2}{\PYZdq{}}\PY{p}{)}
\end{Verbatim}
\end{tcolorbox}

    \begin{Verbatim}[commandchars=\\\{\}]
Max depth : 5
Score sur le train de l'arbre de décision : 0.8287761133628597
Score sur le test de l'arbre de décision : 0.8216097745011889
Max depth : 6
Score sur le train de l'arbre de décision : 0.8375911045161
Score sur le test de l'arbre de décision : 0.8298781113397137
Max depth : 7
Score sur le train de l'arbre de décision : 0.8475715134258025
Score sur le test de l'arbre de décision : 0.8345602372823613
Max depth : 8
Score sur le train de l'arbre de décision : 0.8571520065388156
Score sur le test de l'arbre de décision : 0.839275099223095
Max depth : 9
Score sur le train de l'arbre de décision : 0.8667514157812605
Score sur le test de l'arbre de décision : 0.8379927811502408
Max depth : 10
Score sur le train de l'arbre de décision : 0.8771788808029217
Score sur le test de l'arbre de décision : 0.8378138986698369
Max depth : 11
Score sur le train de l'arbre de décision : 0.8884427238026404
Score sur le test de l'arbre de décision : 0.8332184373980339
Max depth : 12
Score sur le train de l'arbre de décision : 0.9002835113213346
Score sur le test de l'arbre de décision : 0.827860871670689
Max depth : 13
Score sur le train de l'arbre de décision : 0.9126784232754481
Score sur le test de l'arbre de décision : 0.8184371899148659
Max depth : 14
Score sur le train de l'arbre de décision : 0.9248986124917595
Score sur le test de l'arbre de décision : 0.8073907282473716
Max depth : 15
Score sur le train de l'arbre de décision : 0.9360622025943398
Score sur le test de l'arbre de décision : 0.8011019844075465
Max depth : 16
Score sur le train de l'arbre de décision : 0.9465575116161347
Score sur le test de l'arbre de décision : 0.7943883184687182
Max depth : 17
Score sur le train de l'arbre de décision : 0.9559007041620089
Score sur le test de l'arbre de décision : 0.7843571997819065
Max depth : 18
Score sur le train de l'arbre de décision : 0.9638673313927707
Score sur le test de l'arbre de décision : 0.7741319430495943
Max depth : 19
Score sur le train de l'arbre de décision : 0.9706319172890191
Score sur le test de l'arbre de décision : 0.7741276562077122
    \end{Verbatim}

    On observe assez vite le surapprentissage lorsqu'on augmente la
profondeur de l'arbre

Avantages : * On peut contrôler la complexité de l'arbre en jouant sur
des paramètres avec la profondeur ou des stratégies d'élagage *
Interprétabilité des décisions * Pas de problématique de prise en compte
des échelles différentes entre les variables (même si dans notre cas,
nous travaillons sur des données standardisées)

Inconvénient majeur : * Même en jouant sur la complexité de l'arbre, un
arbre tend au surapprentissage et fournit de piètre performance de
généralisation

    \hypertarget{random-forest}{%
\subsubsection{Random Forest}\label{random-forest}}

    \begin{tcolorbox}[breakable, size=fbox, boxrule=1pt, pad at break*=1mm,colback=cellbackground, colframe=cellborder]
\prompt{In}{incolor}{79}{\boxspacing}
\begin{Verbatim}[commandchars=\\\{\}]
\PY{k+kn}{from} \PY{n+nn}{sklearn}\PY{n+nn}{.}\PY{n+nn}{ensemble} \PY{k+kn}{import} \PY{n}{RandomForestRegressor}
\PY{n}{nbTree} \PY{o}{=} \PY{l+m+mi}{100}
\PY{n+nb}{print}\PY{p}{(}\PY{l+s+sa}{f}\PY{l+s+s2}{\PYZdq{}}\PY{l+s+s2}{Nombre d}\PY{l+s+s2}{\PYZsq{}}\PY{l+s+s2}{arbres considérés : }\PY{l+s+si}{\PYZob{}}\PY{n}{nbTree}\PY{l+s+si}{\PYZcb{}}\PY{l+s+s2}{\PYZdq{}}\PY{p}{)}
\PY{k}{for} \PY{n}{depth} \PY{o+ow}{in} \PY{p}{[}\PY{l+m+mi}{5}\PY{p}{,}\PY{l+m+mi}{10}\PY{p}{,}\PY{l+m+mi}{15}\PY{p}{,}\PY{l+m+mi}{20}\PY{p}{,}\PY{l+m+mi}{30}\PY{p}{,} \PY{l+m+mi}{40}\PY{p}{]}\PY{p}{:}
    \PY{n}{randomForest} \PY{o}{=} \PY{n}{RandomForestRegressor}\PY{p}{(}\PY{n}{n\PYZus{}estimators}\PY{o}{=}\PY{n}{nbTree}\PY{p}{,} \PY{n}{random\PYZus{}state}\PY{o}{=}\PY{l+m+mi}{2}\PY{p}{,} \PY{n}{max\PYZus{}depth}\PY{o}{=}\PY{n}{depth}\PY{p}{)}
    \PY{n}{randomForest}\PY{o}{.}\PY{n}{fit}\PY{p}{(}\PY{n}{X\PYZus{}train\PYZus{}scaled}\PY{p}{,} \PY{n}{y\PYZus{}train}\PY{p}{)}
    \PY{n+nb}{print}\PY{p}{(}\PY{l+s+sa}{f}\PY{l+s+s2}{\PYZdq{}}\PY{l+s+s2}{\PYZhy{}\PYZhy{}\PYZhy{} Max depth : }\PY{l+s+si}{\PYZob{}}\PY{n}{depth}\PY{l+s+si}{\PYZcb{}}\PY{l+s+s2}{\PYZdq{}}\PY{p}{)}
    \PY{n+nb}{print}\PY{p}{(}\PY{l+s+sa}{f}\PY{l+s+s2}{\PYZdq{}}\PY{l+s+s2}{\PYZhy{}\PYZhy{}\PYZhy{}\PYZhy{}\PYZhy{}\PYZhy{}\PYZhy{}\PYZhy{}\PYZhy{}Score sur le train avec RandomForest : }\PY{l+s+si}{\PYZob{}}\PY{n}{randomForest}\PY{o}{.}\PY{n}{score}\PY{p}{(}\PY{n}{X\PYZus{}train\PYZus{}scaled}\PY{p}{,}\PY{n}{y\PYZus{}train}\PY{p}{)}\PY{l+s+si}{\PYZcb{}}\PY{l+s+s2}{\PYZdq{}}\PY{p}{)}
    \PY{n+nb}{print}\PY{p}{(}\PY{l+s+sa}{f}\PY{l+s+s2}{\PYZdq{}}\PY{l+s+s2}{\PYZhy{}\PYZhy{}\PYZhy{}\PYZhy{}\PYZhy{}\PYZhy{}\PYZhy{}\PYZhy{}\PYZhy{}Score sur le test avec RandomForest : }\PY{l+s+si}{\PYZob{}}\PY{n}{randomForest}\PY{o}{.}\PY{n}{score}\PY{p}{(}\PY{n}{X\PYZus{}test\PYZus{}scaled}\PY{p}{,}\PY{n}{y\PYZus{}test}\PY{p}{)}\PY{l+s+si}{\PYZcb{}}\PY{l+s+s2}{\PYZdq{}}\PY{p}{)}
\end{Verbatim}
\end{tcolorbox}

    \begin{Verbatim}[commandchars=\\\{\}]
Nombre d'arbres considérés : 100
--- Max depth : 5
---------Score sur le train avec RandomForest : 0.8359706488563999
---------Score sur le test avec RandomForest : 0.8289525406733299
--- Max depth : 10
---------Score sur le train avec RandomForest : 0.8896458202070148
---------Score sur le test avec RandomForest : 0.8595626102602516
--- Max depth : 15
---------Score sur le train avec RandomForest : 0.9407613378235083
---------Score sur le test avec RandomForest : 0.86188618552186
--- Max depth : 20
---------Score sur le train avec RandomForest : 0.9681499184443108
---------Score sur le test avec RandomForest : 0.8596451915853982
--- Max depth : 30
---------Score sur le train avec RandomForest : 0.979733060427418
---------Score sur le test avec RandomForest : 0.858242085415685
--- Max depth : 40
---------Score sur le train avec RandomForest : 0.9803220072381146
---------Score sur le test avec RandomForest : 0.8581958142919872
    \end{Verbatim}

    On observe avec Random Forest une amélioration du score fonction de la
profondeur considérées avec un surapprentissage de plus en plus
important.

    \hypertarget{gridsearch-et-validation-croisuxe9e}{%
\subsubsection{GridSearch et Validation
croisée}\label{gridsearch-et-validation-croisuxe9e}}

Nous allons creuser un peu plus loin afin d'améliorer RandomForest en
optimisant les hyperparamètres du modèle. Pour ce faire nous allons
procéder par validation croisée avec 5 plis sur l'ensemble
d'apprentissage. A l'aide de celle-ci, nous allons chercher quel(s)
paramètre(s) nous donne(nt) le meilleur score et enfin nous évaluerons
la qualité du modèle sur le jeu de données test.

Les paramètres que nous allons chercher à optimiser dans RandomForest
sont : * le paramètre max\_depth qui correspond à la profondeur de
l'arbre * le nombre d'arbres à considérer dans la forêt * le nombre de
features maximale à considérer

    \begin{tcolorbox}[breakable, size=fbox, boxrule=1pt, pad at break*=1mm,colback=cellbackground, colframe=cellborder]
\prompt{In}{incolor}{80}{\boxspacing}
\begin{Verbatim}[commandchars=\\\{\}]
\PY{k+kn}{from} \PY{n+nn}{sklearn}\PY{n+nn}{.}\PY{n+nn}{model\PYZus{}selection} \PY{k+kn}{import} \PY{n}{GridSearchCV}
\PY{c+c1}{\PYZsh{} grille de valeurs}
\PY{n}{params} \PY{o}{=} \PY{p}{[}\PY{p}{\PYZob{}}\PY{l+s+s2}{\PYZdq{}}\PY{l+s+s2}{max\PYZus{}depth}\PY{l+s+s2}{\PYZdq{}}\PY{p}{:} \PY{p}{[}\PY{l+m+mi}{10}\PY{p}{,}\PY{l+m+mi}{15}\PY{p}{,}\PY{l+m+mi}{20}\PY{p}{]}\PY{p}{,} \PY{l+s+s2}{\PYZdq{}}\PY{l+s+s2}{n\PYZus{}estimators}\PY{l+s+s2}{\PYZdq{}}\PY{p}{:} \PY{p}{[}\PY{l+m+mi}{100}\PY{p}{,}\PY{l+m+mi}{200}\PY{p}{,}\PY{l+m+mi}{300}\PY{p}{,}\PY{l+m+mi}{500}\PY{p}{]}\PY{p}{,} \PY{l+s+s2}{\PYZdq{}}\PY{l+s+s2}{max\PYZus{}features}\PY{l+s+s2}{\PYZdq{}}\PY{p}{:} \PY{p}{[}\PY{l+m+mi}{12}\PY{p}{,} \PY{l+m+mi}{15}\PY{p}{,} \PY{l+m+mi}{20}\PY{p}{,} \PY{l+m+mi}{25}\PY{p}{]}\PY{p}{\PYZcb{}}\PY{p}{]}

\PY{n}{gridSearchCV} \PY{o}{=} \PY{n}{GridSearchCV}\PY{p}{(}
    \PY{n}{RandomForestRegressor}\PY{p}{(}\PY{p}{)}\PY{p}{,}
    \PY{n}{params}\PY{p}{,}
    \PY{n}{cv}\PY{o}{=}\PY{l+m+mi}{5}\PY{p}{,}
    \PY{n}{n\PYZus{}jobs}\PY{o}{=}\PY{o}{\PYZhy{}}\PY{l+m+mi}{1}\PY{p}{,}
    \PY{n}{return\PYZus{}train\PYZus{}score}\PY{o}{=}\PY{k+kc}{True}\PY{p}{)}
\PY{n}{gridSearchCV}\PY{o}{.}\PY{n}{fit}\PY{p}{(}\PY{n}{X\PYZus{}train\PYZus{}scaled}\PY{p}{,} \PY{n}{y\PYZus{}train}\PY{p}{)}
\end{Verbatim}
\end{tcolorbox}

            \begin{tcolorbox}[breakable, size=fbox, boxrule=.5pt, pad at break*=1mm, opacityfill=0]
\prompt{Out}{outcolor}{80}{\boxspacing}
\begin{Verbatim}[commandchars=\\\{\}]
GridSearchCV(cv=5, estimator=RandomForestRegressor(), n\_jobs=-1,
             param\_grid=[\{'max\_depth': [10, 15, 20],
                          'max\_features': [12, 15, 20, 25],
                          'n\_estimators': [100, 200, 300, 500]\}],
             return\_train\_score=True)
\end{Verbatim}
\end{tcolorbox}
        
    \begin{tcolorbox}[breakable, size=fbox, boxrule=1pt, pad at break*=1mm,colback=cellbackground, colframe=cellborder]
\prompt{In}{incolor}{81}{\boxspacing}
\begin{Verbatim}[commandchars=\\\{\}]
\PY{n+nb}{print}\PY{p}{(}\PY{l+s+s2}{\PYZdq{}}\PY{l+s+s2}{Score sur le test : }\PY{l+s+si}{\PYZob{}:.2f\PYZcb{}}\PY{l+s+s2}{\PYZdq{}}\PY{o}{.}\PY{n}{format}\PY{p}{(}\PY{n}{gridSearchCV}\PY{o}{.}\PY{n}{score}\PY{p}{(}\PY{n}{X\PYZus{}test\PYZus{}scaled}\PY{p}{,}\PY{n}{y\PYZus{}test}\PY{p}{)}\PY{p}{)}\PY{p}{)}
\end{Verbatim}
\end{tcolorbox}

    \begin{Verbatim}[commandchars=\\\{\}]
Score sur le test : 0.86
    \end{Verbatim}

    \begin{tcolorbox}[breakable, size=fbox, boxrule=1pt, pad at break*=1mm,colback=cellbackground, colframe=cellborder]
\prompt{In}{incolor}{82}{\boxspacing}
\begin{Verbatim}[commandchars=\\\{\}]
\PY{n+nb}{print}\PY{p}{(}\PY{l+s+s2}{\PYZdq{}}\PY{l+s+s2}{Best parameters : }\PY{l+s+si}{\PYZob{}\PYZcb{}}\PY{l+s+s2}{\PYZdq{}}\PY{o}{.}\PY{n}{format}\PY{p}{(}\PY{n}{gridSearchCV}\PY{o}{.}\PY{n}{best\PYZus{}params\PYZus{}}\PY{p}{)}\PY{p}{)}
\PY{n+nb}{print}\PY{p}{(}\PY{l+s+s2}{\PYZdq{}}\PY{l+s+s2}{Best cross\PYZhy{}validation score : }\PY{l+s+si}{\PYZob{}:.2f\PYZcb{}}\PY{l+s+s2}{\PYZdq{}}\PY{o}{.}\PY{n}{format}\PY{p}{(}\PY{n}{gridSearchCV}\PY{o}{.}\PY{n}{best\PYZus{}score\PYZus{}}\PY{p}{)}\PY{p}{)}
\end{Verbatim}
\end{tcolorbox}

    \begin{Verbatim}[commandchars=\\\{\}]
Best parameters : \{'max\_depth': 15, 'max\_features': 25, 'n\_estimators': 300\}
Best cross-validation score : 0.86
    \end{Verbatim}

    \begin{tcolorbox}[breakable, size=fbox, boxrule=1pt, pad at break*=1mm,colback=cellbackground, colframe=cellborder]
\prompt{In}{incolor}{83}{\boxspacing}
\begin{Verbatim}[commandchars=\\\{\}]
\PY{n+nb}{print}\PY{p}{(}\PY{l+s+s2}{\PYZdq{}}\PY{l+s+s2}{Best estimator:}\PY{l+s+se}{\PYZbs{}n}\PY{l+s+si}{\PYZob{}\PYZcb{}}\PY{l+s+s2}{\PYZdq{}}\PY{o}{.}\PY{n}{format}\PY{p}{(}\PY{n}{gridSearchCV}\PY{o}{.}\PY{n}{best\PYZus{}estimator\PYZus{}}\PY{p}{)}\PY{p}{)}
\end{Verbatim}
\end{tcolorbox}

    \begin{Verbatim}[commandchars=\\\{\}]
Best estimator:
RandomForestRegressor(max\_depth=15, max\_features=25, n\_estimators=300)
    \end{Verbatim}

    \begin{tcolorbox}[breakable, size=fbox, boxrule=1pt, pad at break*=1mm,colback=cellbackground, colframe=cellborder]
\prompt{In}{incolor}{84}{\boxspacing}
\begin{Verbatim}[commandchars=\\\{\}]
\PY{n}{featuresImportance} \PY{o}{=} \PY{n}{pd}\PY{o}{.}\PY{n}{Series}\PY{p}{(}\PY{n}{gridSearchCV}\PY{o}{.}\PY{n}{best\PYZus{}estimator\PYZus{}}\PY{o}{.}\PY{n}{feature\PYZus{}importances\PYZus{}}\PY{o}{.}\PY{n}{flatten}\PY{p}{(}\PY{p}{)}\PY{p}{,} \PY{n}{index}\PY{o}{=}\PY{n}{X}\PY{o}{.}\PY{n}{columns}\PY{p}{)}\PY{o}{.}\PY{n}{sort\PYZus{}values}\PY{p}{(}\PY{n}{ascending}\PY{o}{=}\PY{k+kc}{False}\PY{p}{)}
\PY{n}{featuresImportance}\PY{p}{[}\PY{p}{(}\PY{n}{featuresImportance}\PY{p}{)}\PY{o}{\PYZgt{}}\PY{l+m+mf}{0.03}\PY{p}{]}\PY{o}{.}\PY{n}{plot}\PY{p}{(}\PY{n}{kind}\PY{o}{=}\PY{l+s+s2}{\PYZdq{}}\PY{l+s+s2}{bar}\PY{l+s+s2}{\PYZdq{}}\PY{p}{)}
\PY{n}{plt}\PY{o}{.}\PY{n}{title}\PY{p}{(}\PY{l+s+s2}{\PYZdq{}}\PY{l+s+s2}{Feature}\PY{l+s+s2}{\PYZdq{}}\PY{p}{)}
\PY{n}{plt}\PY{o}{.}\PY{n}{ylabel}\PY{p}{(}\PY{l+s+s2}{\PYZdq{}}\PY{l+s+s2}{Feature Importance}\PY{l+s+s2}{\PYZdq{}}\PY{p}{)}
\PY{n}{plt}\PY{o}{.}\PY{n}{show}\PY{p}{(}\PY{p}{)}
\end{Verbatim}
\end{tcolorbox}

    \begin{center}
    \adjustimage{max size={0.9\linewidth}{0.9\paperheight}}{output_140_0.png}
    \end{center}
    { \hspace*{\fill} \\}
    
    A travers une validation croisée et un grid search, on obtient un
paramétrage via Random Forest et on peut visualiser les variables qui
ont de l'importance. On retrouve des variables explicatives en lien avec
notre analyse exploratoire. On est aussi dans un cas où il n'y a pas de
surapprentissage.

    \begin{tcolorbox}[breakable, size=fbox, boxrule=1pt, pad at break*=1mm,colback=cellbackground, colframe=cellborder]
\prompt{In}{incolor}{85}{\boxspacing}
\begin{Verbatim}[commandchars=\\\{\}]
\PY{n}{y\PYZus{}testPred} \PY{o}{=} \PY{n}{gridSearchCV}\PY{o}{.}\PY{n}{best\PYZus{}estimator\PYZus{}}\PY{o}{.}\PY{n}{predict}\PY{p}{(}\PY{n}{X\PYZus{}test\PYZus{}scaled}\PY{p}{)}
\PY{n+nb}{print}\PY{p}{(}\PY{l+s+s2}{\PYZdq{}}\PY{l+s+s2}{Regression metrics pour la forêt aléatoire optimisée for test data}\PY{l+s+s2}{\PYZdq{}}\PY{p}{)}
\PY{n+nb}{print}\PY{p}{(}\PY{n}{regression\PYZus{}metrics}\PY{p}{(}\PY{n}{y\PYZus{}test}\PY{p}{,} \PY{n}{y\PYZus{}testPred}\PY{p}{)}\PY{p}{)}
\end{Verbatim}
\end{tcolorbox}

    \begin{Verbatim}[commandchars=\\\{\}]
Regression metrics pour la forêt aléatoire optimisée for test data
       max\_error  mean\_absolute\_error  mean\_squared\_error  r2\_score
0  667391.637514         71635.656938        1.122499e+10  0.862301
    \end{Verbatim}

    \begin{tcolorbox}[breakable, size=fbox, boxrule=1pt, pad at break*=1mm,colback=cellbackground, colframe=cellborder]
\prompt{In}{incolor}{86}{\boxspacing}
\begin{Verbatim}[commandchars=\\\{\}]
\PY{n}{y\PYZus{}trainPred} \PY{o}{=} \PY{n}{gridSearchCV}\PY{o}{.}\PY{n}{best\PYZus{}estimator\PYZus{}}\PY{o}{.}\PY{n}{predict}\PY{p}{(}\PY{n}{X\PYZus{}train\PYZus{}scaled}\PY{p}{)}
\PY{n+nb}{print}\PY{p}{(}\PY{l+s+s2}{\PYZdq{}}\PY{l+s+s2}{Regression metrics pour la forêt aléatoire optimisée for train data}\PY{l+s+s2}{\PYZdq{}}\PY{p}{)}
\PY{n+nb}{print}\PY{p}{(}\PY{n}{regression\PYZus{}metrics}\PY{p}{(}\PY{n}{y\PYZus{}train}\PY{p}{,} \PY{n}{y\PYZus{}trainPred}\PY{p}{)}\PY{p}{)}
\end{Verbatim}
\end{tcolorbox}

    \begin{Verbatim}[commandchars=\\\{\}]
Regression metrics pour la forêt aléatoire optimisée for train data
       max\_error  mean\_absolute\_error  mean\_squared\_error  r2\_score
0  535465.091525          54852.90272        6.189316e+09  0.925782
    \end{Verbatim}

    \hypertarget{sauvegarde-du-moduxe8le}{%
\subsection{Sauvegarde du modèle}\label{sauvegarde-du-moduxe8le}}

    \begin{tcolorbox}[breakable, size=fbox, boxrule=1pt, pad at break*=1mm,colback=cellbackground, colframe=cellborder]
\prompt{In}{incolor}{87}{\boxspacing}
\begin{Verbatim}[commandchars=\\\{\}]
\PY{c+c1}{\PYZsh{}from joblib import dump, load}
\end{Verbatim}
\end{tcolorbox}

    \begin{tcolorbox}[breakable, size=fbox, boxrule=1pt, pad at break*=1mm,colback=cellbackground, colframe=cellborder]
\prompt{In}{incolor}{88}{\boxspacing}
\begin{Verbatim}[commandchars=\\\{\}]
\PY{c+c1}{\PYZsh{}dump(gridSearchCV.best\PYZus{}estimator\PYZus{}.predict, \PYZsq{}sauvegardeModele.joblib\PYZsq{}) }
\end{Verbatim}
\end{tcolorbox}

    \begin{tcolorbox}[breakable, size=fbox, boxrule=1pt, pad at break*=1mm,colback=cellbackground, colframe=cellborder]
\prompt{In}{incolor}{89}{\boxspacing}
\begin{Verbatim}[commandchars=\\\{\}]
\PY{c+c1}{\PYZsh{}clf = load(\PYZsq{}sauvegardeModele.joblib\PYZsq{}) }
\end{Verbatim}
\end{tcolorbox}

    \begin{tcolorbox}[breakable, size=fbox, boxrule=1pt, pad at break*=1mm,colback=cellbackground, colframe=cellborder]
\prompt{In}{incolor}{90}{\boxspacing}
\begin{Verbatim}[commandchars=\\\{\}]
\PY{c+c1}{\PYZsh{}clf.predict(X\PYZus{}test\PYZus{}scaled)}
\end{Verbatim}
\end{tcolorbox}

    \begin{tcolorbox}[breakable, size=fbox, boxrule=1pt, pad at break*=1mm,colback=cellbackground, colframe=cellborder]
\prompt{In}{incolor}{91}{\boxspacing}
\begin{Verbatim}[commandchars=\\\{\}]
\PY{c+c1}{\PYZsh{}dfDepIni.to\PYZus{}csv(\PYZsq{}../input/AvecCoordonneesGeo/dep75.csv\PYZsq{})}
\end{Verbatim}
\end{tcolorbox}

    \hypertarget{conclusion}{%
\section{Conclusion}\label{conclusion}}

\hypertarget{sur-le-travail-ruxe9alisuxe9}{%
\subsection{Sur le travail réalisé}\label{sur-le-travail-ruxe9alisuxe9}}

\begin{itemize}
\tightlist
\item
  L'analyse univariée et multivariée ont permis de mettre en évidence
  des liens entre les variables explicatives et à expliquer
\item
  Le featuring Ingeenering a été un travail réalisé sur les dates pour
  essayer de voir les liens avec la variable à prédire.
\item
  Les modèles linéaires donne des résultats pas très intéressants sur
  certaines métriques
\item
  Un modèle basé sur des arbres de décision permet d'obtenir des
  meilleurs résultats par rapport à la regression linéaire. Une
  optimisation des paramètres a pu être mise en oeuvre via validation
  croisée et grille de recherche
\end{itemize}

\hypertarget{sur-les-perspectives}{%
\subsection{Sur les perspectives}\label{sur-les-perspectives}}

\begin{itemize}
\tightlist
\item
  Sur le code : la mise en place de Pipe avec l'utilisation de
  OneHotEncoder et StandardScaler.
\item
  Sur les modèles : tester d'autres modèles pour améliorer la prévision.
  On peut penser à une régression polynomiale ou boosting d'arbres de
  régression, ou des modèles traitant spécifiquement de séries
  temporelles.
\item
  Un traitement des points atypiques a été réalisé avec aussi de
  l'imputation mais il faudrait passer plus de temps sur la
  compréhension des données.
\end{itemize}


    % Add a bibliography block to the postdoc
    
    
    
\end{document}
